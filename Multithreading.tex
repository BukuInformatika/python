\documentclass [12pt,a4paper,notitlepage,oneside,bahasa]{article}
\usepackage[left=3.00 cm, right=2.00 cm, bottom=2.00 cm, top=3.00 cm]{geometry}
\begin{document}
\title{\textbf Multithreading}
\maketitle

Menjalankan beberapa\textit{ thread} mirip dengan menjalankan beberapa program yang berbeda secara bersamaan, namun dengan manfaat berikut :
\begin{itemize}
	\item Beberapa \textit{thread} dalam proses berbagi ruang data yang sama dengan benang induk dan karena dapat saling berbagi informasi atau berkomunikasi satu sama lain dengan lebih muda daripada jika prosesnya terpisah \par
	\item \textit{thread} terkadang disebut proses ringan dan tidak membutuhkan banyak memori atas, mereka lebih murah daripada proses.
\end{itemize}

Sebuah \textit{thread} memiliki permulaan, urutan eksekusi dan sebuah kesimpulan. Ini memiliki pointer perintah yang melacak dari mana dalam konteksnya saat ini berjalan. \par
\begin{itemize}
	\item Hal ini dapat dilakukan sebelum pre-\textit{empted} (\textit{inturrepted})
	\item Untuk sementara dapat ditunda sementara \textit{thread} lainnya yang sedang berjalan ini disebut unggul. 
\end{itemize}
\noindent 
\section{Memulai Thread Baru}
\noindent 
\hspace*{0.5in} Untuk melakukan \textit{thread} lain, perlu memanggil metode berikut yang tersedia dimodul \textit{thread} :
\noindent 
\begin{center}{\fontsize{9pt}{9pt}\selectfont Thread.start  \_  new   \_  thread (function, args [, kwargs] )}\end{center}
Pemanggilan metode ini memungkinkan cara cepat dan tepat untuk membuat \textit{thread} baru di linux dan window.
Pemanggilan metode segera kembali dan anak  \textit{thread} dimulai dan fungsi pemanggilan dengan daftar \textit{args} telah berlalu. Saat fungsi kembali ujung \textit{thread} akan berakhir.
Disini, \textit{args }adalah tupel argumen. Gunakan tupel kosong untuk memanggil fungsi tanpa melewati argumen. \textit{Kwargs} adalah kamus opsional argumen kata kunci.
\begin{center}{\fontsize{9pt}{9pt}\selectfont Thread.start  \_  new   \_  thread (function, args [, kwargs] )}\end{center}
Pemanggilan metode ini memungkinkan cara cepat dan tepat untuk membuat \textit{thread} baru di linux dan window.
Pemanggilan metode segera kembali dan anak  \textit{thread} dimulai dan fungsi pemanggilan dengan daftar \textit{args} telah berlalu. Saat fungsi kembali ujung \textit{thread} akan berakhir.
Disini, \textit{args }adalah tupel argumen. Gunakan tupel kosong untuk memanggil fungsi tanpa melewati argumen. \textit{Kwargs} adalah kamus opsional argumen kata kunci.
\noindent 
\par
\noindent 


%%%%%%%%%%%%  Start New Page here %%%%%%%%%%%%%%


\newpage

\vspace{12pt}
\vspace{12pt}
\noindent 
Contoh : 
\begin{verbatim}
#!/usr/bin/python

import thread
import time

# Define a function for the thread
def print_time( threadName, delay):
   count = 0
   while count < 5:
      time.sleep(delay)
      count += 1
      print "%s: %s" % ( threadName, time.ctime(time.time()) )

# Create two threads as follows
try:
   thread.start_new_thread( print_time, ("Thread-1", 2, ) )
   thread.start_new_thread( print_time, ("Thread-2", 4, ) )
except:
   print "Error: unable to start thread"

while 1:
   pass
\end{verbatim}
	
	%%%%%%%%%%%%  Start New Page here %%%%%%%%%%%%%%
	
	
	\newpage
	
	Bila kode diatas dieksekusi, maka menghasilkan hasil sebagai berikut :} \par
\vspace{10pt}
\noindent 
\begin{center}{\fontsize{10pt}{10pt}\selectfont Thread-1 : Thu Jan 22 15:42:17 2009}\end{center} \par
\noindent 
\begin{center}{\fontsize{10pt}{10pt}\selectfont Thread-1 : Thu Jan 22 15:42:19 2009}\end{center} \par
\noindent 
\begin{center}{\fontsize{10pt}{10pt}\selectfont Thread-2 : Thu Jan 22 15:42:19 2009}\end{center} \par
\noindent 
\begin{center}{\fontsize{10pt}{10pt}\selectfont Thread-1 : Thu Jan 22 15:42:21 2009}\end{center} \par
\noindent 
\begin{center}{\fontsize{10pt}{10pt}\selectfont Thread-2 : Thu Jan 22 15:42:23 2009}\end{center} \par
\noindent 
\begin{center}{\fontsize{10pt}{10pt}\selectfont Thread-1 : Thu Jan 22 15:42:23 2009}\end{center} \par
\noindent 
\begin{center}{\fontsize{10pt}{10pt}\selectfont Thread-1~:  Thu Jan 22 15:42:23 2009}\end{center} \par
\noindent 
\begin{center}{\fontsize{10pt}{10pt}\selectfont Thread-1 : Thu Jan 22 15:42:25 2009}\end{center} \par
\noindent 
\begin{center}{\fontsize{10pt}{10pt}\selectfont Thread-2 : Thu Jan 22 15:42:27 2009}\end{center} \par
\noindent 
\begin{center}{\fontsize{10pt}{10pt}\selectfont Thread-2 : Thu Jan 22 15:42:31 2009}\end{center} \par
\noindent 
\begin{center}{\fontsize{10pt}{10pt}\selectfont Thread-2 : Thu Jan 22 15:42:35 2009}\end{center} \par

\vspace{12pt}
Meskipun sangat efektif untuk benang tingkat rendah, namun modul \textit{thread} sangat terbatas dibandingkan dengan modul yang baru. \par
\vspace{12pt}
\section{ Modul Threading } \par

Modul threading yang lebih baru disertakan dengan Python 2.4 memberikan jauh lebih kuat, dukungan tingkat tinggi untuk \textit{thread}\textit{ }dari modul\textit{ }\textit{thread}\textit{ }dibahas pada bagian sebelumnya. \par
The \textit{thread}\textit{ing }modul mengekpos semua metode dari \textit{thread}\textit{ }dan menyediakan beberapa metode tambahan : \par
\begin{itemize}
	\item \textbf{t}\textbf{hreading.activeCount() } \par
	Mengembalikan jumlah objek \textit{thread} yang aktif \par
	\item \textbf{t}\textbf{hreading.currentThread() } \par
	Mengembalikan jumlah objek \textit{thread} dalam kontrol benang pemanggil \par
	\item \textbf{t}\textbf{hreading.enumerate() } \par
	Mengembalikan daftar semua benda \textit{thread}\textit{ }yang sedang aktif \par
	\vspace{12pt}
	Selain metode, modul \textit{thread}\textit{ing }memiliki \textit{thread}\textit{ }kelas yang mengimplementasikan \textit{thread}\textit{ing. }Metode yang disediakan oleh \textit{thread}\textit{ }kelas adalah sebagai berikut : \par
	\item \textbf{run()} \par
	Metode adalah titik masuk untuk \textit{thread} \par
	\item \textbf{start()} \par
	Metode dimulai\textbf{ }\textit{thread}\textit{ }dengan memanggil metode run \par
	\item \textbf{join(}\textbf{[time]}\textbf{)} \par
	Menunggu benang untuk mengakhiri \par
	\item \textbf{isAlive()} \par
	Metode memeriksa apakah\textbf{ }\textit{thread}\textit{ }masih mengeksekusi\textbf{ } \par
	\item \textbf{getName()} \par
	Metode mengambalikan nama\textbf{ }\textit{thread} \par
	\item \textbf{setName()} \par
	Metode menetapkan nama\textbf{ }\textit{thread} \par
	\end{itemize}
	\vspace{12pt}
	
\section{Membuat Thread Menggunakan Threading Modul} 
Untuk melaksanakan \textit{thread}\textit{ }baru menggunakan\textit{ threading} harus melakukan hal berikut : \par
	\begin{itemize}
		\item Mendefinisikan subclass dari \textit{thread} kelas \par
		\item Menimpa  $  \_  $init $  \_  $ (self [args]) metode untuk menambahkan argumen tambahan \par
		\item Menimpa run(self[args]) metode untuk menerapkan apa \textit{thread} harus dilakukan ketika mulai 
	\end{itemize}
\par
\noindent 


%%%%%%%%%%%%  Start New Page here %%%%%%%%%%%%%%


\newpage

\vspace{12pt}
Setelah membuat baru \textit{thread} subclass, dapat membuah sebuah instance dari itu dan kemudian memulai \textit{thread} baru dengan menerapkan \textit{start(),} yang ada gilirinnya panggilan \textit{run()} metode. \par
\vspace{12pt}
	Contoh :
\begin{verbatim}
#!/usr/bin/python

import threading
import time

exitFlag = 0

class myThread (threading.Thread):
    def __init__(self, threadID, name, counter):
        threading.Thread.__init__(self)
        self.threadID = threadID
        self.name = name
        self.counter = counter
    def run(self):
        print "Starting " + self.name
        print_time(self.name, self.counter, 5)
        print "Exiting " + self.name

def print_time(threadName, delay, counter):
    while counter:
        if exitFlag:
            threadName.exit()
        time.sleep(delay)
        print "%s: %s" % (threadName, time.ctime(time.time()))
        counter -= 1

# Create new threads
thread1 = myThread(1, "Thread-1", 1)
thread2 = myThread(2, "Thread-2", 2)

# Start new Threads
thread1.start()
thread2.start()

print "Exiting Main Thread"
\end{verbatim}
	Ketika kode diatas dijalankan, menghasilkan hasil sebagai berikut:
\par
\noindent 
{\fontsize{10pt}{10pt}\selectfont Starting Thread-1} \par
\noindent 
{\fontsize{10pt}{10pt}\selectfont Starting Thread-2} \par
\noindent 
{\fontsize{10pt}{10pt}\selectfont Exiting Main Thread} \par
\noindent 
{\fontsize{10pt}{10pt}\selectfont Thread-1 : Thu Mar 21 09:10:03 2013} \par
\noindent 
{\fontsize{10pt}{10pt}\selectfont Thread-1 : Thu Mar 21 09:10:04 2013} \par
\noindent 
{\fontsize{10pt}{10pt}\selectfont Thread-2 : Thu Mar 21 09:10:04 2013} \par
\noindent 
{\fontsize{10pt}{10pt}\selectfont Thread-1 : Thu Mar 21 09:10:05 2013} \par
\noindent 
{\fontsize{10pt}{10pt}\selectfont Thread-2 : Thu Mar 21 09:10:06 2013} \par
\noindent 
{\fontsize{10pt}{10pt}\selectfont Thread-1 : Thu Mar 21 09:10:07 2013} \par
\noindent 
{\fontsize{10pt}{10pt}\selectfont Exiting Thread-1} \par
\noindent 
{\fontsize{10pt}{10pt}\selectfont Thread-2 : Thu Mar 21 09:10:08 2013} \par
\noindent 
{\fontsize{10pt}{10pt}\selectfont Thread-2 : Thu Mar 21 09:10:10 2013} \par
\noindent 
{\fontsize{10pt}{10pt}\selectfont Thread-2 : Thu Mar 21 09:10:12 2013} \par
\noindent 
{\fontsize{10pt}{10pt}\selectfont Exiting Thread=2} \par
\vspace{10pt}

\section{Sinkronisasi Thread}
\textit{Threading} modul disediakan dengan Python termasuk sederhana untuk menerapkan mekanisme bahwa memungkinkan untuk menyinkronkan \textit{thread}\textit{ }penguncian. Sebuah kunci baru dibuat dengan memanggil \textit{lock() }metode yang mengembalikan kunci baru. \par
The \textit{acquire}\textit{ }\textit{(blocking)}\textit{ }metode objek kunci baru digunakan untuk memaksa \textit{thread}\textit{ }untuk menjalankan serempak. Opsional \textit{blocking} parameter memungkikan untuk mengontrol apakah\textit{ thread} menunggu untuk mendapatkan kunci. \par
Jika \textit{blocking} diatur ke 0, \textit{thread} segera kembali dengan nilai 0 jika kunci tidak dapat diperoleh dan dengan 1 jika kunci dikuisisi. Jika pemblokiran diatur ke 1, blok dan menunggu kunci yang akan dirilis. \par
The \textit{release()} metode objek kunci baru digunakan untuk melepaskan kunci ketika tidak lagi diperlukan.  \par
\noindent 
Contoh:
\begin{verbatim}
\noindent 
#!/usr/bin/python
import threading
import time
class myThread (threading.Thread):
  def_init_(self, threadID, name, counter):
	threading.Thread._init_ (self)
	self.threadID = threadID
	self.name = name
	self.counter = counter
  def run(self)
 	print  "Starting" + self.name

 	#Get lock to synchronize threads
	ThreadLock.acquire()
print_time(self.name, self.counter, 3)

#Free lock to realease next thread
	ThreadLock.release()
 
Def print_time(threadName, delay, counter):
	while counter:
	time.sleep(delay)
	print  " \% s:\% s"\% (threadName, time.ctime(time.time()))
	counter -= 1
	
threadLock = threading.Lock()
	threads = []

#  Create new threads
thread1 = myThread(1, "Thread-1", 1)
thread2 = myThread(2, "Thread-2", 2)

#Start new Threads
thread1.start()
thread2.start()

# Add threads to thread list
threads.append(thread1)
thread2.append(thread2)

#Wait for all threads to complete}
Fort t in threads:
	t.join()
	print "Exiting Main thread"
\end{verbatim}

Bila kode diatas dieksekusi, maka menghasilkan sebagai berikut : \par
\vspace{10pt}
\noindent 
{\fontsize{10pt}{10pt}\selectfont Starting Thread-1} \par
\noindent 
{\fontsize{10pt}{10pt}\selectfont Starting Thread-2} \par
\noindent 
{\fontsize{10pt}{10pt}\selectfont Thread-1: Thu Mar 21 09:11:28 2013} \par
\noindent 
{\fontsize{10pt}{10pt}\selectfont Thread-1: Thu Mar 21 09:11:29 2013} \par
\noindent 
{\fontsize{10pt}{10pt}\selectfont Thread-1: Thu Mar 21 09:11:30 2013} \par
\noindent 
{\fontsize{10pt}{10pt}\selectfont Thread-2: Thu Mar 21 09:11:32 2013} \par
\noindent 
{\fontsize{10pt}{10pt}\selectfont Thread-2: Thu Mar 21 09:11:34 2013} \par
\noindent 
{\fontsize{10pt}{10pt}\selectfont Thread-2: Thu Mar 21 09:11:36 2013} \par
\noindent 
{\fontsize{10pt}{10pt}\selectfont Exiting Main Thread} \par
\vspace{12pt}

\section{Multithreaded Antrian Prioritas} \par
The queue modul memungkinkan untuk membuat objek antrian baru yang dapat menampung jumlah tertentu item. Ada metode berikut untuk mengontrol antrian : \par
\begin{itemize}
	\item \textbf{get()} \par
		Menghapus dan mengembalikan item dari antrian
	\par
	\item \textbf{put()} \par
		Menambahkan item ke antrian
	\par
	\item \textbf{qsize()} \par
		Mengembalikan jumlah item yang saat ini dalam antrian
	\par
	\item \textbf{empty()} \par
		Mengembalikan benar jika antrian kosong jika tidak, salah
	\par
	\item \textbf{full()}\end{itemize}
\par
	Mengembalikan benar jika antrian penuh jika tidak, salah
\par
\vspace{12pt}
\vspace{12pt}
\vspace{12pt}
\noindent 
Contoh: 
\begin{verbatim}
#!/usr/bin/python
# -*- coding: UTF-8 -*-

import Queue
import threading
import time

exitFlag = 0

class myThread (threading.Thread):
    def __init__(self, threadID, name, q):
        threading.Thread.__init__(self)
        self.threadID = threadID
        self.name = name
        self.q = q
    def run(self):
        print "Starting " + self.name
        process_data(self.name, self.q)
        print "Exiting " + self.name

def process_data(threadName, q):
    while not exitFlag:
        queueLock.acquire()
        if not workQueue.empty():
            data = q.get()
            queueLock.release()
            print "%s processing %s" % (threadName, data)
        else:
            queueLock.release()
        time.sleep(1)

threadList = ["Thread-1", "Thread-2", "Thread-3"]
nameList = ["One", "Two", "Three", "Four", "Five"]
queueLock = threading.Lock()
workQueue = Queue.Queue(10)
threads = []
threadID = 1

# Create new threadsfor tName in threadList:
    thread = myThread(threadID, tName, workQueue)
    thread.start()
    threads.append(thread)
    threadID += 1
# Fill the queuequeueLock.acquire()
for word in nameList:
    workQueue.put(word)
queueLock.release()

# Wait for queue to empty} \par
while not workQueue.empty():
    pass
    
# Notify threads it?s time to exit} \par
exitFlag = 1

# Wait for all threads to complete} \par
for t in threads:
    t.join()
print "Exiting Main Thread"
\end{verbatim}

Bila kode diatas dieksekusi, maka menghasilkan hasil sebagai berikut: \par
\vspace{12pt}
\noindent 
{\fontsize{10pt}{10pt}\selectfont Starting Thread-1} \par
\noindent 
{\fontsize{10pt}{10pt}\selectfont Starting Thread-2} \par
\noindent 
{\fontsize{10pt}{10pt}\selectfont Starting Thread-3} \par
\noindent 
{\fontsize{10pt}{10pt}\selectfont Thread-1 processing One} \par
\noindent 
{\fontsize{10pt}{10pt}\selectfont Thread-2 processing Two} \par
\noindent 
{\fontsize{10pt}{10pt}\selectfont Thread-3 processing Three} \par
\noindent 
{\fontsize{10pt}{10pt}\selectfont Thread-1 processing Four} \par
\noindent 
{\fontsize{10pt}{10pt}\selectfont Thread-2 processing Five} \par
\noindent 
{\fontsize{10pt}{10pt}\selectfont Exiting Thread-3} \par
\noindent 
{\fontsize{10pt}{10pt}\selectfont Exiting Thread-1} \par
\noindent 
{\fontsize{10pt}{10pt}\selectfont Exiting Thread-2} \par
\noindent 
{\fontsize{10pt}{10pt}\selectfont Exiting Main Thread} \par
\vspace{12pt}

end{document}
