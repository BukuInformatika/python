%%%%%%%%%%%%  Generated using docx2latex.pythonanywhere.com  %%%%%%%%%%%%%%


\documentclass[a4paper,12pt]{report}

% Other options in place of 'report' are 1)article 2)book 3)letter
% Other options in place of 'a4paper' are 1)a5paper 2)b5paper 3)letterpaper 4)legalpaper 5)executivepaper


 %%%%%%%%%%%%  Include Packages  %%%%%%%%%%%%%%


\usepackage{amsmath}
\usepackage{latexsym}
\usepackage{amsfonts}
\usepackage{amssymb}
\usepackage{graphicx}
\usepackage{txfonts}
\usepackage{wasysym}
\usepackage{enumitem}
\usepackage{adjustbox}
\usepackage{ragged2e}
\usepackage{tabularx}
\usepackage{changepage}
\usepackage{setspace}
\usepackage{hhline}
\usepackage{multicol}
\usepackage{float}
\usepackage{multirow}
\usepackage{makecell}
\usepackage{fancyhdr}
\usepackage[toc,page]{appendix}
\usepackage[utf8]{inputenc}
\usepackage[T1]{fontenc}
\usepackage{hyperref}


 %%%%%%%%%%%%  Define Colors For Hyperlinks  %%%%%%%%%%%%%%


\hypersetup{
colorlinks=true,
linkcolor=blue,
filecolor=magenta,
urlcolor=cyan,
}
\urlstyle{same}


 %%%%%%%%%%%%  Set Depths for Sections  %%%%%%%%%%%%%%

% 1) Section
% 1.1) SubSection
% 1.1.1) SubSubSection
% 1.1.1.1) Paragraph
% 1.1.1.1.1) Subparagraph


\setcounter{tocdepth}{5}
\setcounter{secnumdepth}{5}


 %%%%%%%%%%%%  Set Page Margins  %%%%%%%%%%%%%%


\usepackage[a4paper,bindingoffset=0.2in,headsep=0.5cm,left=1.0in,right=1.0in,bottom=2cm,top=2cm,headheight=2cm]{geometry}
\everymath{\displaystyle}


 %%%%%%%%%%%%  Set Depths for Nested Lists created by \begin{enumerate}  %%%%%%%%%%%%%%


\setlistdepth{9}
\newlist{myEnumerate}{enumerate}{9}
	\setlist[myEnumerate,1]{label=\arabic*)}
	\setlist[myEnumerate,2]{label=\alph*)}
	\setlist[myEnumerate,3]{label=(\roman*)}
	\setlist[myEnumerate,4]{label=(\arabic*)}
	\setlist[myEnumerate,5]{label=(\Alph*)}
	\setlist[myEnumerate,6]{label=(\Roman*)}
	\setlist[myEnumerate,7]{label=\arabic*}
	\setlist[myEnumerate,8]{label=\alph*}
	\setlist[myEnumerate,9]{label=\roman*}

\renewlist{itemize}{itemize}{9}
	\setlist[itemize]{label=$\cdot$}
	\setlist[itemize,1]{label=\textbullet}
	\setlist[itemize,2]{label=$\circ$}
	\setlist[itemize,3]{label=$\ast$}
	\setlist[itemize,4]{label=$\dagger$}
	\setlist[itemize,5]{label=$\triangleright$}
	\setlist[itemize,6]{label=$\bigstar$}
	\setlist[itemize,7]{label=$\blacklozenge$}
	\setlist[itemize,8]{label=$\prime$}



 %%%%%%%%%%%%  Header here  %%%%%%%%%%%%%%


\pagestyle{fancy}
\fancyhf{}


 %%%%%%%%%%%%  Footer here  %%%%%%%%%%%%%%




 %%%%%%%%%%%%  Print Page Numbers  %%%%%%%%%%%%%%


\rfoot{\thepage}


 %%%%%%%%%%%%  This sets linespacing (verticle gap between Lines) Default=1 %%%%%%%%%%%%%%


\setstretch{1.08}


 %%%%%%%%%%%%  Document Code starts here %%%%%%%%%%%%%%


\begin{document}
\sloppy
\begin{center}{\fontsize{14pt}{14pt}\selectfont \textbf{TUPLES} \\}\end{center} \par
\noindent 
{\fontsize{14pt}{14pt}\selectfont Sebuah tupel adalah urutan objek Python yang tidak berubah. Tupel adalah urutan, seperti daftar. Perbedaan antara tupel dan daftar adalah, tupel tidak dapat diubah tidak seperti daftar dan tupel menggunakan tanda kurung, sedangkan daftar menggunakan tanda kurung siku. \\} \par
\noindent 
{\fontsize{14pt}{14pt}\selectfont Membuat tuple semudah memasukkan nilai-nilai yang dipisahkan koma. Opsional Anda dapat memasukkan nilai-nilai yang dipisahkan koma ini di antara tanda kurung juga. Misalnya - \\} \par
\noindent 
\vspace{14pt}
\noindent 
{\fontsize{14pt}{14pt}\selectfont Tup1 = ('fisika', 'kimia', 1997, 2000); \\} \par
\noindent 
{\fontsize{14pt}{14pt}\selectfont Tup2 = (1, 2, 3, 4, 5); \\} \par
\noindent 
{\fontsize{14pt}{14pt}\selectfont Tup3 = "a", "b", "c", "d"; \\} \par
\noindent 
{\fontsize{14pt}{14pt}\selectfont Tuple kosong ditulis sebagai dua tanda kurung yang tidak berisi apa - \\} \par
\noindent 
{\fontsize{14pt}{14pt}\selectfont tup1 = (); \\} \par
\noindent 
{\fontsize{14pt}{14pt}\selectfont Untuk menulis tupel yang berisi satu nilai, Anda harus menyertakan koma, meskipun hanya ada satu nilai - \\} \par
\noindent 
{\fontsize{14pt}{14pt}\selectfont Tup1 = (50,); \\} \par
\noindent 
{\fontsize{14pt}{14pt}\selectfont Seperti indeks string, indeks tuple mulai dari 0, dan mereka dapat diiris, digabungkan, dan seterusnya. \\} \par
\noindent 
{\fontsize{14pt}{14pt}\selectfont Mengakses Nilai pada Tuples: \\} \par
\noindent 
{\fontsize{14pt}{14pt}\selectfont Untuk mengakses nilai dalam tupel, gunakan tanda kurung siku untuk mengiris beserta indeks atau indeks untuk mendapatkan nilai yang tersedia pada indeks tersebut. Misalnya - \\} \par
\noindent 
{\fontsize{14pt}{14pt}\selectfont  $  \#  $! / Usr / bin / python \\} \par
\vspace{14pt}
\noindent 
{\fontsize{14pt}{14pt}\selectfont Tup1 = ('fisika', 'kimia', 1997, 2000); \\} \par
\noindent 
{\fontsize{14pt}{14pt}\selectfont Tup2 = (1, 2, 3, 4, 5, 6, 7); \\} \par
\vspace{14pt}
\noindent 
{\fontsize{14pt}{14pt}\selectfont Cetak "tup1 [0]:", tup1 [0] \\} \par
\noindent 
{\fontsize{14pt}{14pt}\selectfont Cetak "tup2 [1: 5]:", tup2 [1: 5] \\} \par
\noindent 
{\fontsize{14pt}{14pt}\selectfont Bila kode diatas dieksekusi, maka menghasilkan hasil sebagai berikut - \\} \par
\noindent 
{\fontsize{14pt}{14pt}\selectfont tup1 [0]: fisika \\} \par
\noindent 
{\fontsize{14pt}{14pt}\selectfont Tup2 [1: 5]: [2, 3, 4, 5] \\} \par
\noindent 
{\fontsize{14pt}{14pt}\selectfont Memperbarui Tupel \\} \par
\noindent 
{\fontsize{14pt}{14pt}\selectfont Tupel tidak berubah yang berarti Anda tidak dapat memperbarui atau mengubah nilai elemen tupel. Anda dapat mengambil bagian dari tupel yang ada untuk membuat tupel baru seperti ditunjukkan oleh contoh berikut - \\} \par
\noindent 
{\fontsize{14pt}{14pt}\selectfont  $  \#  $! / Usr / bin / python \\} \par
\vspace{14pt}
\noindent 
{\fontsize{14pt}{14pt}\selectfont Tup1 = (12, 34.56); \\} \par
\noindent 
{\fontsize{14pt}{14pt}\selectfont Tup2 = ('abc', 'xyz'); \\} \par
\vspace{14pt}
\noindent 
{\fontsize{14pt}{14pt}\selectfont  $  \#  $ Tindakan berikut tidak berlaku untuk tupel \\} \par
\noindent 
{\fontsize{14pt}{14pt}\selectfont  $  \#  $ Tup1 [0] = 100; \\} \par
\vspace{14pt}
\noindent 
{\fontsize{14pt}{14pt}\selectfont  $  \#  $ Jadi mari kita buat tupel baru sebagai berikut \\} \par
\noindent 
{\fontsize{14pt}{14pt}\selectfont Tup3 = tup1 + tup2; \\} \par
\noindent 
{\fontsize{14pt}{14pt}\selectfont Cetak tup3 \\} \par
\noindent 
{\fontsize{14pt}{14pt}\selectfont Bila kode diatas dieksekusi, maka menghasilkan hasil sebagai berikut - \\} \par
\noindent 
{\fontsize{14pt}{14pt}\selectfont (12, 34.56, 'abc', 'xyz') \\} \par
\noindent 
{\fontsize{14pt}{14pt}\selectfont Hapus Elemen Tuple \\} \par
\noindent 
{\fontsize{14pt}{14pt}\selectfont Menghapus elemen tuple individual tidak mungkin dilakukan. Tentu saja, tidak ada yang salah dengan menggabungkan tuple lain dengan unsur-unsur yang tidak diinginkan dibuang. \\} \par
\noindent 
{\fontsize{14pt}{14pt}\selectfont Untuk secara eksplisit menghapus keseluruhan tuple, cukup gunakan del statement. Sebagai contoh: \\} \par
\noindent 
{\fontsize{14pt}{14pt}\selectfont  $  \#  $! / Usr / bin / python \\} \par
\vspace{14pt}
\noindent 
{\fontsize{14pt}{14pt}\selectfont Tup = ('fisika', 'kimia', 1997, 2000); \\} \par
\vspace{14pt}
\noindent 
{\fontsize{14pt}{14pt}\selectfont Cetak tup \\} \par
\noindent 
{\fontsize{14pt}{14pt}\selectfont Del tup; \\} \par
\noindent 
{\fontsize{14pt}{14pt}\selectfont Cetak "Setelah menghapus tup:" \\} \par
\noindent 
{\fontsize{14pt}{14pt}\selectfont Cetak tup \\} \par
\noindent 
{\fontsize{14pt}{14pt}\selectfont Ini menghasilkan hasil berikut. Perhatikan pengecualian yang diangkat, ini karena setelah del tup tupel tidak ada lagi - \\} \par
\noindent 
{\fontsize{14pt}{14pt}\selectfont ('Fisika', 'kimia', 1997, 2000) \\} \par
\noindent 
{\fontsize{14pt}{14pt}\selectfont Setelah menghapus tup: \\} \par
\noindent 
{\fontsize{14pt}{14pt}\selectfont Traceback (panggilan terakhir): \\} \par
\noindent 
{\fontsize{14pt}{14pt}\selectfont  $  $ $  $File "test.py", baris 9, di <module> \\} \par
\noindent 
{\fontsize{14pt}{14pt}\selectfont  $  $ $  $ $  $ $  $Cetak tup; \\} \par
\noindent 
{\fontsize{14pt}{14pt}\selectfont NameError: nama 'tup' tidak didefinisikan \\} \par
\noindent 
{\fontsize{14pt}{14pt}\selectfont Operasi Tuple Dasar \\} \par
\noindent 
{\fontsize{14pt}{14pt}\selectfont Tupel merespons operator + dan * seperti string; Mereka berarti penggabungan dan pengulangan di sini juga, kecuali hasilnya adalah tupel baru, bukan string. \\} \par
\noindent 
{\fontsize{14pt}{14pt}\selectfont Sebenarnya, tupel menanggapi semua operasi urutan umum yang kami gunakan pada senar di bab sebelumnya - \\} \par


 %%%%%%%%%%%%  Table No:1 Here %%%%%%%%%%%%%%


\begin{table}[H]
\centering
\begin{adjustbox}{width=\textwidth}
\begin{tabular}{ p{1.97in}p{1.96in}p{1.97in} }
\hhline{---}
\multicolumn{1}{|p{1.97in}}{\Centering \textbf{Python Expression}} & \multicolumn{1}{|p{1.96in}}{\Centering \textbf{Results }} & \multicolumn{1}{|p{1.97in}|}{\Centering \textbf{Description}} & \hhline{---}
\multicolumn{1}{|p{1.97in}}{len((1, 2, 3))} & \multicolumn{1}{|p{1.96in}}{3} & \multicolumn{1}{|p{1.97in}|}{Length} & \hhline{---}
\multicolumn{1}{|p{1.97in}}{(1, 2, 3) + (4, 5, 6)} & \multicolumn{1}{|p{1.96in}}{(1, 2, 3, 4, 5, 6)} & \multicolumn{1}{|p{1.97in}|}{Concatenation} & \hhline{---}
\multicolumn{1}{|p{1.97in}}{('Hi!',) * 4} & \multicolumn{1}{|p{1.96in}}{('Hi!', 'Hi!', 'Hi!', 'Hi!')} & \multicolumn{1}{|p{1.97in}|}{Repetition} & \hhline{---}
\multicolumn{1}{|p{1.97in}}{3 in (1, 2, 3)} & \multicolumn{1}{|p{1.96in}}{True} & \multicolumn{1}{|p{1.97in}|}{Membership} & \hhline{---}
\multicolumn{1}{|p{1.97in}}{for x in (1, 2, 3): print x,} & \multicolumn{1}{|p{1.96in}}{1 2 3} & \multicolumn{1}{|p{1.97in}|}{Iteration} & \hline
\end{tabular}
\end{adjustbox}
\end{table}


 %%%%%%%%%%%%  Table No:1 Ends Here %%%%%%%%%%%%%%


\noindent 
Indexing, Slicing, dan Matrixes \par
\noindent 
Karena tupel adalah urutan, pengindeksan dan pengiris bekerja dengan cara yang sama untuk tupel seperti yang mereka lakukan untuk string. Dengan asumsi masukan berikut - \par
\noindent 
L = ('spam', 'Spam', 'SPAM!') \par
\noindent 
{\fontsize{14pt}{14pt}\selectfont  $  $ \\} \par


 %%%%%%%%%%%%  Table No:2 Here %%%%%%%%%%%%%%


\begin{table}[H]
\centering
\begin{adjustbox}{width=\textwidth}
\begin{tabular}{ p{2.09in}p{2.08in}p{2.09in} }
\hhline{---}
\multicolumn{1}{|p{2.09in}}{\Centering \textbf{Python Expression}} & \multicolumn{1}{|p{2.08in}}{\Centering \textbf{Results }} & \multicolumn{1}{|p{2.09in}|}{\Centering \textbf{Description}} & \hhline{---}
\multicolumn{1}{|p{2.09in}}{L[2]} & \multicolumn{1}{|p{2.08in}}{'SPAM!'} & \multicolumn{1}{|p{2.09in}|}{Offsets start at zero} & \hhline{---}
\multicolumn{1}{|p{2.09in}}{L[-2]} & \multicolumn{1}{|p{2.08in}}{'Spam'} & \multicolumn{1}{|p{2.09in}|}{Negative: count from the right} & \hhline{---}
\multicolumn{1}{|p{2.09in}}{L[1:]} & \multicolumn{1}{|p{2.08in}}{['Spam', 'SPAM!']} & \multicolumn{1}{|p{2.09in}|}{Slicing fetches sections} & \hline
\end{tabular}
\end{adjustbox}
\end{table}


 %%%%%%%%%%%%  Table No:2 Ends Here %%%%%%%%%%%%%%


\vspace{12pt}
\noindent 
{\fontsize{14pt}{14pt}\selectfont Tidak melampirkan delimiters \\} \par
\noindent 
{\fontsize{14pt}{14pt}\selectfont Setiap kumpulan beberapa objek, yang dipisahkan koma, ditulis tanpa mengidentifikasi simbol, yaitu tanda kurung untuk daftar, tanda kurung untuk tupel, dll., Default tupel, seperti yang ditunjukkan dalam contoh singkat ini - \\} \par
\noindent 
{\fontsize{14pt}{14pt}\selectfont  $  \#  $! / Usr / bin / python \\} \par
\vspace{14pt}
\noindent 
{\fontsize{14pt}{14pt}\selectfont cetak 'abc', -4.24e93, 18 + 6.6j, 'xyz' \\} \par
\noindent 
{\fontsize{14pt}{14pt}\selectfont x, y = 1, 2; \\} \par
\noindent 
{\fontsize{14pt}{14pt}\selectfont Cetak "Nilai x, y:", x, y \\} \par
\noindent 
{\fontsize{14pt}{14pt}\selectfont Bila kode diatas dieksekusi, maka menghasilkan hasil sebagai berikut - \\} \par
\noindent 
{\fontsize{14pt}{14pt}\selectfont abc -4.24e + 93 (18 + 6.6j) xyz \\} \par
\noindent 
{\fontsize{14pt}{14pt}\selectfont Nilai x, y: 1 2 \\} \par
\noindent 
{\fontsize{14pt}{14pt}\selectfont Built-in Fungsi Tuple \\} \par
\noindent 
{\fontsize{14pt}{14pt}\selectfont Python mencakup fungsi tupel berikut – \\} \par
\vspace{14pt}
\vspace{14pt}
\vspace{12pt}


 %%%%%%%%%%%%  Table No:3 Here %%%%%%%%%%%%%%


\begin{table}[H]
\centering
\begin{adjustbox}{width=\textwidth}
\begin{tabular}{ p{0.3in}p{5.15in} }
\hhline{--}
\multicolumn{1}{|p{0.3in}}{\Centering \textbf{SN}} & \multicolumn{1}{|p{5.15in}|}{\Centering \textbf{Function with Description}} & \hhline{--}
\multicolumn{1}{|p{0.3in}}{1} & \multicolumn{1}{|p{5.15in}|}{\href{https://www.tutorialspoint.com/python/tuple $  \_  $cmp.htm}{cmp(tuple1, tuple2)}
Compares elements of both tuples.} & \hhline{--}
\multicolumn{1}{|p{0.3in}}{2} & \multicolumn{1}{|p{5.15in}|}{\href{https://www.tutorialspoint.com/python/tuple $  \_  $len.htm}{len(tuple)}
Gives the total length of the tuple.} & \hhline{--}
\multicolumn{1}{|p{0.3in}}{3} & \multicolumn{1}{|p{5.15in}|}{\href{https://www.tutorialspoint.com/python/tuple $  \_  $max.htm}{max(tuple)}
Returns item from the tuple with max value.} & \hhline{--}
\multicolumn{1}{|p{0.3in}}{4} & \multicolumn{1}{|p{5.15in}|}{\href{https://www.tutorialspoint.com/python/tuple $  \_  $min.htm}{min(tuple)}
Returns item from the tuple with min value.} & \hhline{--}
\multicolumn{1}{|p{0.3in}}{5} & \multicolumn{1}{|p{5.15in}|}{\href{https://www.tutorialspoint.com/python/tuple $  \_  $tuple.htm}{tuple(seq)}
Converts a list into tuple.} & \hline
\end{tabular}
\end{adjustbox}
\end{table}


 %%%%%%%%%%%%  Table No:3 Ends Here %%%%%%%%%%%%%%


\vspace{14pt}
\vspace{14pt}
\vspace{14pt}
\vspace{14pt}
\vspace{14pt}
\noindent 
{\fontsize{14pt}{14pt}\selectfont Dalam pemrograman Python, tuple mirip dengan daftar. Perbedaan antara keduanya adalah kita tidak bisa mengubah unsur tuple begitu diberikan sedangkan dalam daftar, elemen bisa diubah. \\} \par
\noindent 
{\fontsize{14pt}{14pt}\selectfont Keuntungan Tuple over List \\} \par
\vspace{14pt}
\noindent 
{\fontsize{14pt}{14pt}\selectfont Karena, tupel sangat mirip dengan daftar, keduanya juga digunakan dalam situasi yang sama. \\} \par
\vspace{14pt}
\noindent 
{\fontsize{14pt}{14pt}\selectfont Namun, ada beberapa keuntungan dari penerapan tupel dari daftar. Di bawah ini tercantum beberapa keuntungan utama: \\} \par
\vspace{14pt}
\noindent 
{\fontsize{14pt}{14pt}\selectfont  $  $ $  $ $  $ $  $Kami umumnya menggunakan tuple untuk tipe data heterogen dan berbeda untuk tipe data homogen (sejenis). \\} \par
\noindent 
{\fontsize{14pt}{14pt}\selectfont  $  $ $  $ $  $ $  $Karena tupel tidak dapat diubah, iterasi melalui tupel lebih cepat daripada daftar. Jadi ada sedikit peningkatan kinerja. \\} \par
\noindent 
{\fontsize{14pt}{14pt}\selectfont  $  $ $  $ $  $ $  $Tupel yang mengandung unsur yang tidak berubah dapat digunakan sebagai kunci untuk kamus. Dengan daftar, ini tidak mungkin. \\} \par
\noindent 
{\fontsize{14pt}{14pt}\selectfont  $  $ $  $ $  $ $  $Jika Anda memiliki data yang tidak berubah, menerapkannya sebagai tupel akan menjamin bahwa itu tetap dilindungi penulisan. \\} \par
\vspace{20pt}
\noindent 
{\fontsize{14pt}{14pt}\selectfont Dalam pemrograman Python, tuple mirip dengan daftar. Perbedaan antara keduanya adalah kita tidak bisa mengubah unsur tuple begitu diberikan sedangkan dalam daftar, elemen bisa diubah. \\} \par
\noindent 
{\fontsize{14pt}{14pt}\selectfont Keuntungan Tuple over List \\} \par
\noindent 
\vspace{14pt}
\noindent 
{\fontsize{14pt}{14pt}\selectfont Karena, tupel sangat mirip dengan daftar, keduanya juga digunakan dalam situasi yang sama. \\} \par
\noindent 
\vspace{14pt}
\noindent 
{\fontsize{14pt}{14pt}\selectfont Namun, ada beberapa keuntungan dari penerapan tupel dari daftar. Di bawah ini tercantum beberapa keuntungan utama: \\} \par
\noindent 
\vspace{14pt}
\noindent 
{\fontsize{14pt}{14pt}\selectfont  $  $ $  $ $  $ $  $Kami umumnya menggunakan tuple untuk tipe data heterogen dan berbeda untuk tipe data homogen (sejenis). \\} \par
\noindent 
{\fontsize{14pt}{14pt}\selectfont  $  $ $  $ $  $ $  $Karena tupel tidak dapat diubah, iterasi melalui tupel lebih cepat daripada daftar. Jadi ada sedikit peningkatan kinerja. \\} \par
\noindent 
{\fontsize{14pt}{14pt}\selectfont  $  $ $  $ $  $ $  $Tupel yang mengandung unsur yang tidak berubah dapat digunakan sebagai kunci kamus. Dengan daftar, ini tidak mungkin. \\} \par
\noindent 
{\fontsize{14pt}{14pt}\selectfont  $  $ $  $ $  $ $  $Jika Anda memiliki data yang tidak berubah, menerapkannya sebagai tupel akan menjamin bahwa itu tetap dilindungi penulisan. \\} \par
\noindent 
\vspace{14pt}
\noindent 
{\fontsize{14pt}{14pt}\selectfont Membuat Tuple \\} \par
\noindent 
\vspace{14pt}
\noindent 
{\fontsize{14pt}{14pt}\selectfont Sebuah tuple dibuat dengan menempatkan semua item (elemen) di dalam tanda kurung (), dipisahkan dengan koma. Tanda kurung bersifat opsional namun merupakan praktik yang baik untuk menuliskannya. \\} \par
\noindent 
\vspace{14pt}
\noindent 
{\fontsize{14pt}{14pt}\selectfont Sebuah tuple dapat memiliki sejumlah item dan mereka mungkin memiliki tipe yang berbeda (integer, float, list, string etc.). \\} \par
\vspace{20pt}
\vspace{20pt}
\noindent 
{\fontsize{14pt}{14pt}\selectfont  $  \#  $ empty tuple \\} \par
\noindent 
{\fontsize{14pt}{14pt}\selectfont  $  \#  $ Output: () \\} \par
\noindent 
{\fontsize{14pt}{14pt}\selectfont my $  \_  $tuple = () \\} \par
\noindent 
{\fontsize{14pt}{14pt}\selectfont print(my $  \_  $tuple) \\} \par
\vspace{14pt}
\noindent 
{\fontsize{14pt}{14pt}\selectfont  $  \#  $ tuple having integers \\} \par
\noindent 
{\fontsize{14pt}{14pt}\selectfont  $  \#  $ Output: (1, 2, 3) \\} \par
\noindent 
{\fontsize{14pt}{14pt}\selectfont my $  \_  $tuple = (1, 2, 3) \\} \par
\noindent 
{\fontsize{14pt}{14pt}\selectfont print(my $  \_  $tuple) \\} \par
\vspace{14pt}
\noindent 
{\fontsize{14pt}{14pt}\selectfont  $  \#  $ tuple with mixed datatypes \\} \par
\noindent 
{\fontsize{14pt}{14pt}\selectfont  $  \#  $ Output: (1, "Hello", 3.4) \\} \par
\noindent 
{\fontsize{14pt}{14pt}\selectfont my $  \_  $tuple = (1, "Hello", 3.4) \\} \par
\noindent 
{\fontsize{14pt}{14pt}\selectfont print(my $  \_  $tuple) \\} \par
\vspace{14pt}
\noindent 
{\fontsize{14pt}{14pt}\selectfont  $  \#  $ nested tuple \\} \par
\noindent 
{\fontsize{14pt}{14pt}\selectfont  $  \#  $ Output: ("mouse", [8, 4, 6], (1, 2, 3)) \\} \par
\noindent 
{\fontsize{14pt}{14pt}\selectfont my $  \_  $tuple = ("mouse", [8, 4, 6], (1, 2, 3)) \\} \par
\noindent 
{\fontsize{14pt}{14pt}\selectfont print(my $  \_  $tuple) \\} \par
\vspace{14pt}
\noindent 
{\fontsize{14pt}{14pt}\selectfont  $  \#  $ tuple can be created without parentheses \\} \par
\noindent 
{\fontsize{14pt}{14pt}\selectfont  $  \#  $ also called tuple packing \\} \par
\noindent 
{\fontsize{14pt}{14pt}\selectfont  $  \#  $ Output: 3, 4.6, "dog" \\} \par
\vspace{14pt}
\noindent 
{\fontsize{14pt}{14pt}\selectfont my $  \_  $tuple = 3, 4.6, "dog" \\} \par
\noindent 
{\fontsize{14pt}{14pt}\selectfont print(my $  \_  $tuple) \\} \par
\vspace{14pt}
\noindent 
{\fontsize{14pt}{14pt}\selectfont  $  \#  $ tuple unpacking is also possible \\} \par
\noindent 
{\fontsize{14pt}{14pt}\selectfont  $  \#  $ Output: \\} \par
\noindent 
{\fontsize{14pt}{14pt}\selectfont  $  \#  $ 3 \\} \par
\noindent 
{\fontsize{14pt}{14pt}\selectfont  $  \#  $ 4.6 \\} \par
\noindent 
{\fontsize{14pt}{14pt}\selectfont  $  \#  $ dog \\} \par
\noindent 
{\fontsize{14pt}{14pt}\selectfont a, b, c = my $  \_  $tuple \\} \par
\noindent 
{\fontsize{14pt}{14pt}\selectfont print(a) \\} \par
\noindent 
{\fontsize{14pt}{14pt}\selectfont print(b) \\} \par
\noindent 
{\fontsize{14pt}{14pt}\selectfont print(c) \\} \par
\vspace{14pt}
\noindent 
{\fontsize{14pt}{14pt}\selectfont Membuat tuple dengan satu elemen agak rumit. \\} \par
\noindent 
\vspace{14pt}
\noindent 
{\fontsize{14pt}{14pt}\selectfont Memiliki satu elemen dalam kurung saja tidak cukup. Kita membutuhkan koma trailing untuk menunjukkan bahwa sebenarnya ada tupel. \\} \par
\vspace{20pt}
\noindent 
{\fontsize{20pt}{20pt}\selectfont  $  \#  $ only parentheses is not enough \\} \par
\noindent 
{\fontsize{20pt}{20pt}\selectfont  $  \#  $ Output: <class 'str'> \\} \par
\noindent 
{\fontsize{20pt}{20pt}\selectfont my $  \_  $tuple = ("hello") \\} \par
\noindent 
{\fontsize{20pt}{20pt}\selectfont print(type(my $  \_  $tuple)) \\} \par
\vspace{20pt}
\noindent 
{\fontsize{20pt}{20pt}\selectfont  $  \#  $ need a comma at the end \\} \par
\noindent 
{\fontsize{20pt}{20pt}\selectfont  $  \#  $ Output: <class 'tuple'> \\} \par
\noindent 
{\fontsize{20pt}{20pt}\selectfont my $  \_  $tuple~= ("hello",)   \\} \par
\noindent 
{\fontsize{20pt}{20pt}\selectfont print(type(my $  \_  $tuple)) \\} \par
\vspace{20pt}
\noindent 
{\fontsize{20pt}{20pt}\selectfont  $  \#  $ parentheses is optional \\} \par
\noindent 
{\fontsize{20pt}{20pt}\selectfont  $  \#  $ Output: <class 'tuple'> \\} \par
\noindent 
{\fontsize{20pt}{20pt}\selectfont my $  \_  $tuple = "hello", \\} \par
\noindent 
{\fontsize{20pt}{20pt}\selectfont print(type(my $  \_  $tuple)) \\} \par
\vspace{20pt}
\noindent 
{\fontsize{14pt}{14pt}\selectfont Mengakses Elemen dalam Tuple \\} \par
\noindent 
\vspace{14pt}
\noindent 
{\fontsize{14pt}{14pt}\selectfont Ada berbagai cara untuk mengakses elemen tuple. \\} \par
\noindent 
{\fontsize{14pt}{14pt}\selectfont 1. Pengindeksan \\} \par
\noindent 
\vspace{14pt}
\noindent 
{\fontsize{14pt}{14pt}\selectfont Kita bisa menggunakan operator indeks [] untuk mengakses item di tupel dimana indeks dimulai dari 0. \\} \par
\noindent 
\vspace{14pt}
\noindent 
{\fontsize{14pt}{14pt}\selectfont Jadi, tupel yang memiliki 6 elemen akan memiliki indeks dari 0 sampai 5. Mencoba mengakses elemen lain yang (6, 7, ...) akan menghasilkan IndexError. \\} \par
\noindent 
\vspace{14pt}
\noindent 
{\fontsize{14pt}{14pt}\selectfont Indeks harus berupa bilangan bulat, jadi kita tidak bisa menggunakan float atau jenis lainnya. Ini akan menghasilkan TypeError. \\} \par
\noindent 
\vspace{14pt}
\noindent 
{\fontsize{14pt}{14pt}\selectfont Demikian juga, tuple bersarang diakses menggunakan pengindeksan nested, seperti yang ditunjukkan pada contoh di bawah ini. \\} \par
\noindent 
{\fontsize{20pt}{20pt}\selectfont my $  \_  $tuple = ('p','e','r','m','i','t') \\} \par
\vspace{20pt}
\noindent 
{\fontsize{20pt}{20pt}\selectfont  $  \#  $ Output: 'p' \\} \par
\noindent 
{\fontsize{20pt}{20pt}\selectfont print(my $  \_  $tuple[0]) \\} \par
\vspace{20pt}
\noindent 
{\fontsize{20pt}{20pt}\selectfont  $  \#  $ Output: 't' \\} \par
\noindent 
{\fontsize{20pt}{20pt}\selectfont print(my $  \_  $tuple[5]) \\} \par
\vspace{20pt}
\noindent 
{\fontsize{20pt}{20pt}\selectfont  $  \#  $ index must be in range \\} \par
\noindent 
{\fontsize{20pt}{20pt}\selectfont  $  \#  $ If you uncomment line 14, \\} \par
\noindent 
{\fontsize{20pt}{20pt}\selectfont  $  \#  $ you will get an error. \\} \par
\noindent 
{\fontsize{20pt}{20pt}\selectfont  $  \#  $ IndexError: list index out of range \\} \par
\vspace{20pt}
\noindent 
{\fontsize{20pt}{20pt}\selectfont  $  \#  $print(my $  \_  $tuple[6]) \\} \par
\vspace{20pt}
\noindent 
{\fontsize{20pt}{20pt}\selectfont  $  \#  $ index must be an integer \\} \par
\noindent 
{\fontsize{20pt}{20pt}\selectfont  $  \#  $ If you uncomment line 21, \\} \par
\noindent 
{\fontsize{20pt}{20pt}\selectfont  $  \#  $ you will get an error. \\} \par
\noindent 
{\fontsize{20pt}{20pt}\selectfont  $  \#  $ TypeError: list indices must be integers, not float \\} \par
\vspace{20pt}
\noindent 
{\fontsize{20pt}{20pt}\selectfont  $  \#  $my $  \_  $tuple[2.0] \\} \par
\vspace{20pt}
\noindent 
{\fontsize{20pt}{20pt}\selectfont  $  \#  $ nested tuple \\} \par
\noindent 
{\fontsize{20pt}{20pt}\selectfont n $  \_  $tuple = ("mouse", [8, 4, 6], (1, 2, 3)) \\} \par
\vspace{20pt}
\noindent 
{\fontsize{20pt}{20pt}\selectfont  $  \#  $ nested index \\} \par
\noindent 
{\fontsize{20pt}{20pt}\selectfont  $  \#  $ Output: 's' \\} \par
\noindent 
{\fontsize{20pt}{20pt}\selectfont print(n $  \_  $tuple[0][3]) \\} \par
\vspace{20pt}
\noindent 
{\fontsize{20pt}{20pt}\selectfont  $  \#  $ nested index \\} \par
\noindent 
{\fontsize{20pt}{20pt}\selectfont  $  \#  $ Output: 4 \\} \par
\noindent 
{\fontsize{20pt}{20pt}\selectfont print(n $  \_  $tuple[1][1]) \\} \par
\vspace{26pt}
\noindent 
{\fontsize{14pt}{14pt}\selectfont Slicing \\} \par
\vspace{14pt}
\noindent 
{\fontsize{14pt}{14pt}\selectfont Kita bisa mengakses berbagai item dalam tupel dengan menggunakan operator pengiris - titik dua ":". \\} \par
\vspace{20pt}
\noindent 
{\fontsize{20pt}{20pt}\selectfont my $  \_  $tuple = ('p','r','o','g','r','a','m','i','z') \\} \par
\vspace{20pt}
\noindent 
{\fontsize{20pt}{20pt}\selectfont  $  \#  $ elements 2nd to 4th \\} \par
\noindent 
{\fontsize{20pt}{20pt}\selectfont  $  \#  $ Output: ('r', 'o', 'g') \\} \par
\noindent 
{\fontsize{20pt}{20pt}\selectfont print(my $  \_  $tuple[1:4]) \\} \par
\vspace{20pt}
\noindent 
{\fontsize{20pt}{20pt}\selectfont  $  \#  $ elements beginning to 2nd \\} \par
\noindent 
{\fontsize{20pt}{20pt}\selectfont  $  \#  $ Output: ('p', 'r') \\} \par
\noindent 
{\fontsize{20pt}{20pt}\selectfont print(my $  \_  $tuple[:-7]) \\} \par
\vspace{20pt}
\noindent 
{\fontsize{20pt}{20pt}\selectfont  $  \#  $ elements 8th to end \\} \par
\noindent 
{\fontsize{20pt}{20pt}\selectfont  $  \#  $ Output: ('i', 'z') \\} \par
\noindent 
{\fontsize{20pt}{20pt}\selectfont print(my $  \_  $tuple[7:]) \\} \par
\vspace{20pt}
\noindent 
{\fontsize{20pt}{20pt}\selectfont  $  \#  $ elements beginning to end \\} \par
\noindent 
{\fontsize{20pt}{20pt}\selectfont  $  \#  $ Output: ('p', 'r', 'o', 'g', 'r', 'a', 'm', 'i', 'z') \\} \par
\noindent 
{\fontsize{20pt}{20pt}\selectfont print(my $  \_  $tuple[:]) \\} \par
\vspace{20pt}
\noindent 
{\fontsize{14pt}{14pt}\selectfont Mengubah Tuple \\} \par
\vspace{14pt}
\noindent 
{\fontsize{14pt}{14pt}\selectfont Tidak seperti daftar, tupel tidak dapat diubah. \\} \par
\vspace{14pt}
\noindent 
{\fontsize{14pt}{14pt}\selectfont Ini berarti elemen tupel tidak dapat diubah begitu telah ditetapkan. Tapi, jika elemen itu sendiri adalah datatype yang bisa berubah seperti daftar, item nested-nya bisa diubah. \\} \par
\vspace{14pt}
\noindent 
{\fontsize{14pt}{14pt}\selectfont Kita juga bisa menugaskan tuple ke nilai yang berbeda (reassignment). \\} \par
\vspace{16pt}
\noindent 
{\fontsize{16pt}{16pt}\selectfont my $  \_  $tuple = (4, 2, 3, [6, 5]) \\} \par
\vspace{16pt}
\noindent 
{\fontsize{16pt}{16pt}\selectfont  $  \#  $ we cannot change an element \\} \par
\noindent 
{\fontsize{16pt}{16pt}\selectfont  $  \#  $ If you uncomment line 8 \\} \par
\noindent 
{\fontsize{16pt}{16pt}\selectfont  $  \#  $ you will get an error: \\} \par
\noindent 
{\fontsize{16pt}{16pt}\selectfont  $  \#  $ TypeError: 'tuple' object does not support item assignment \\} \par
\vspace{16pt}
\noindent 
{\fontsize{16pt}{16pt}\selectfont  $  \#  $my $  \_  $tuple[1] = 9 \\} \par
\vspace{16pt}
\noindent 
{\fontsize{16pt}{16pt}\selectfont  $  \#  $ but item of mutable element can be changed \\} \par
\noindent 
{\fontsize{16pt}{16pt}\selectfont  $  \#  $ Output: (4, 2, 3, [9, 5]) \\} \par
\noindent 
{\fontsize{16pt}{16pt}\selectfont my $  \_  $tuple[3][0] = 9 \\} \par
\noindent 
{\fontsize{16pt}{16pt}\selectfont print(my $  \_  $tuple) \\} \par
\vspace{16pt}
\noindent 
{\fontsize{16pt}{16pt}\selectfont  $  \#  $ tuples can be reassigned \\} \par
\noindent 
{\fontsize{16pt}{16pt}\selectfont  $  \#  $ Output: ('p', 'r', 'o', 'g', 'r', 'a', 'm', 'i', 'z') \\} \par
\noindent 
{\fontsize{16pt}{16pt}\selectfont my $  \_  $tuple = ('p','r','o','g','r','a','m','i','z') \\} \par
\noindent 
{\fontsize{16pt}{16pt}\selectfont print(my $  \_  $tuple) \\} \par
\end{document}
