
\section*{Phyton}

Berkenalan dengan Python

Python adalah bahasa pemograman tingkat tinggi yang dapat digunakan secara luas di berbagai bidang. Python diciptakan pertama kali oleh Guido van Rossum pada tahun 1991. Sintaksnya dan fungsi pada Python dipengaruhi oleh beberapa bahasa seperti C, C++, Lisp, Perl dan Java. Oleh karena itu, kita dapat menemui konsep pemograman procedural, functional dan object-oriented di Python. Python relatif mudah dipelajari bila dibandingkan dengan C++, Java dan PHP karena sintaks Python lebih singkat, lebih jelas dan mudah dipahami oleh programmer pemula.
Meskipun di Indonesia Python tidak sepopuler PHP dan Java, Python sangat patut untuk dipelajari karena banyak pilihan librari Python yang bisa kita gunakan yang biasanya tidak dapat kita temui di bahasa lain. Misalnya numpy dan scipy, alternatif untuk komputasi scientific, scikit-learn untuk implementasi machine learning (klasifikasi, regresi, klustering, ekstraksi fitur dll) dan masih banyak lagi. Pada artikel ini saya akan menjelaskan dasar-dasar dari pemograman Python


