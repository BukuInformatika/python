%Kelompok 1 D4 TI 3D
%Wahyu Maruti Adjie_1154034
%Muhammad Nur Ikhsan_1154087
%Emy Safitri_1154102
%Andi Ikram Maulana_1154065
%Ilman Mubarik Sidiq_1154114

\section
Modul memungkinkan Anda mengatur kode Python secara logis. Mengelompokkan kode terkait ke dalam suatu modul membuat kode lebih cepat dipahami dan digunakan. Modul ini merupakan objek Python dengan atribut yang diberi nama sewenang-wenang sehingga Anda bisa mengikat dan memberi referensi.
Cukup, modul merupakan file yang terdiri dari kode Python. Modul bisa mendefinisikan fungsi, kelas dan variabel. Modul juga dapat menyertakan kode runnable.
Example
Kode Python untuk sebuah modul yang di beri nama aname biasanya berada pada sebuah file bernama aname.py. 

\subsection{import}
Ketika penafsir menemukan sebuah pernyataan impor, ia mengimpor modul apabila modul tersebut berada di jalur pencarian. Jalur pencarian merupakan daftar direktori yang ditafsirkan juru bahasa sebelum mengimpor modul. Misalnya, untuk mengimpor modul support.py, Anda perlu meletakkan perintah berikut di bagian atas skrip - 
 
\subsection{buat modul}Modul hanya dimuat satu kali, berapa pun jumlahnya diimpor. Hal ini mencegah eksekusi modul terjadi berulang-ulang jika terjadi beberapa impor.
 

\subsection{PYTHONPATH} 
The PYTHONPATH merupakan variabel lingkungan, yang terdiri dari daftar direktori. Sintaksis PYTHONPATH sama dengan variabel shell PATH. 
Berikut adalah PYTHONPATH khas dari sistem Windows:
 
\subsection{variabel}
Pernyataan global VarName memberitahu Python bahwa VarName merupakan variabel global. Python berhenti mencari namespace lokal untuk variabel tersebut. 
Sebagai contoh, kita mendefinisikan sebuah variabel Money in the global namespace. Dalam fungsi Money, kita menetapkan Money a value, oleh karena itu Python mengasumsikan variabel lokal Moneyas. Namun, kami mengakses nilai variabel lokal Moneybefore yang mengaturnya, jadi UnboundLocalError merupakan hasilnya. Uncommenting pernyataan global memperbaiki masalahnya.
 
\subsection{the dir( ) Function}
Fungsi dir () built-in mengembalikan daftar string yang diurutkan yang berisi nama yang ditentukan oleh sebuah modul.
Daftar berisi nama semua modul, variabel dan fungsi yang didefinisikan dalam modul. Berikut merupakan contoh sederhana - 



\subsection{paket}
Paket merupakan struktur direktori file hirarkis yang mendefinisikan satu lingkungan aplikasi Python yang terdiri dari modul dan subpackages dan sub-subpackages, dan seterusnya.
Pertimbangkan sebuah file Pots.py yang tersedia di direktori Phone. File ini memiliki baris kode sumber berikut -

Jika Anda berhenti dari juru bahasa Python dan memasukkannya lagi, definisi yang telah Anda buat (fungsi dan variabel) hilang. Oleh karena itu, jika Anda ingin menulis program yang agak lama, sebaiknya Anda menggunakan editor teks untuk menyiapkan masukan bagi penerjemah dan menjalankannya dengan file itu sebagai masukan. Ini dikenal dengan membuat skrip. Seiring program Anda semakin lama, Anda mungkin ingin membaginya menjadi beberapa file untuk memudahkan perawatan. Anda mungkin juga ingin menggunakan fungsi praktis yang telah Anda tulis di beberapa program tanpa menyalin definisinya ke dalam setiap program.
Untuk mendukung ini, Python memiliki cara untuk menempatkan definisi dalam file dan menggunakannya dalam naskah atau dalam contoh juru bahasa interaktif. File seperti itu disebut modul; definisi dari modul dapat diimpor ke modul lain atau masuk ke modul utama (kumpulan variabel yang Anda akses ke dalam naskah yang dieksekusi di tingkat atas dan dalam mode kalkulator).
Modul merupakan file yang berisi definisi dan pernyataan Python. Nama file merupakan nama modul dengan akhiran .py ditambahkan. Dalam sebuah modul, nama modul (sebagai string) tersedia sebagai nilai dari variabel global. Misalnya, gunakan editor teks favorit Anda untuk membuat file bernama fibo.py di direktori saat ini dengan konten berikut. 

Modul dapat berisi pernyataan eksekusi serta definisi fungsi. Pernyataan ini dimaksudkan untuk menginisialisasi modul. Mereka dieksekusi hanya untuk pertama kalinya nama modul ditemukan dalam sebuah pernyataan impor. [1] (Mereka juga dijalankan jika file dijalankan sebagai skrip.
Setiap modul memiliki tabel simbol pribadinya, yang digunakan sebagai tabel simbol global oleh semua fungsi yang didefinisikan dalam modul. Dengan demikian, penulis modul dapat menggunakan variabel global dalam modul tanpa mengkhawatirkan bentrokan kecelakaan dengan variabel global pengguna. Di sisi lain, jika Anda tahu apa yang Anda lakukan, Anda dapat menyentuh variabel global modul dengan notasi yang sama yang digunakan untuk merujuk pada fungsinya, nama modname.itemname.
Modul bisa mengimpor modul lainnya. Sudah menjadi kebiasaan tapi tidak diharuskan untuk menempatkan semua pernyataan impor di awal modul (atau naskah, dalam hal ini). Nama modul yang diimpor ditempatkan di tabel simbol global modul pengimpor.

Python memeriksa tanggal modifikasi sumber dari versi yang dikompilasi untuk melihat apakah sudah ketinggalan zaman dan perlu dikompilasi ulang. Ini merupakan proses yang benar-benar otomatis. Selain itu, modul yang dikompilasi bersifat platform-independen, sehingga perpustakaan yang sama dapat dibagi antar sistem dengan arsitektur yang berbeda. Python tidak memeriksa cache dalam dua situasi. Pertama, selalu mengkompilasi ulang dan tidak menyimpan hasilnya untuk modul yang dimuat langsung dari baris perintah. Kedua, tidak memeriksa cache jika tidak ada modul sumber. Untuk mendukung distribusi non-source (dikompilasi saja), modul yang dikompilasi harus berada dalam direktori sumber, dan tidak boleh ada modul sumber.
Beberapa tip untuk para ahli: 
