Perkenalan Phyton
1. Sejarah
Bahasa pemrograman Python adalah bahasa yang dibuat oleh seorang keturunan Belanda yaitu Guido van Rossum. Awalnya, pembuatan bahasa pemrograman ini adalah untuk membuat skrip bahasa tingkat tinggi pada sebuah sistem operasi yang terdistribusi Amoeba. Python telah digunakan oleh beberapa pengembang dan bahkan digunakan oleh beberapa perusahaan untuk pembuatan perangkat lunak komersial.
Pemrograman bahasa python ini adalah pemrogram gratis atau freeware, sehingga dapat dikembangkan, dan tidak ada batasan dalam penyalinannya dan mendistribusikan. Terdapat beberapa pelayanan yang disediakan lengkap dengan source codenya, debugger dan profiler, interface, fungsi sistem, GUI, dan basisdatanya. Python tersedia untuk berbagai Sistem Operasi, seperti Unix (linux), PCs (DOS, Windows, OS/2), Machintosh dan sebagainya.

2. Dukungan Komunitas yang Aktif

Python adalah salah satu pemrograman yang terus berkembang dan bertahan dikarenakan dukungan komunitas yang aktif diseluruh dunia. Banyak forum-forum ataupun blogger-blogger yang sering membagi pengalaman dalam menggunakan python. Hal ini memudahkan bagi pengguna pemula maupun pengembang untuk bertanya dan sharing tentang ilmu pemrograman ini.
