% Kelas D4 TI 3B Kelompok 3
% Diana Satima Gistivani 1154018
% INDAH RAHMAWATI 1154070

\section{Python XML Processing}
  XML adalah bahasa open source portable yang mungkinkan pemrogram mengemangkan aplikasi yang dapat dibaca oleh aplikasi lain, 
terlepas dari sistem operasi dan bahasa pengembangnya.
\subsubsection{Apa itu XML}
  Extensible Markup Languange (XML) adalah bahasa markup seperti HTML atau SGML. 
Ini direkomendasikan oleh World Wide Web Consortium dan tersedia sebagai standar terbuka.
XML sangat berguna untuk mencatat data berukuran kecil dan menengah tanpa memerlukan tulang punggung berbasis SQL. 

\subsection{Keunggulan dan Kelemahan Python XML Processing}
\subsubsection{Keunggulan Python XML Processing}
\begin{enumerate}
Keunggulan dari python dapat dijabarkan dari faktor-faktor berikut ini :
\item Quality : Python merupakan piranti lunak yang memakai meodologi reusability sehingga komponen-komponen pembangun piranti      lunak mudah digunakan dan diatur.
\item Productivity : penulisan program menggunakan python lebih mudah, karena interpreter menangani source code secara terpisah pada lower-level language.Interpreter menangani tipe deklarasi variabel,manajemen memory dan source code.
\item Portability : program python dapat dieksekusi di berbagai jenis komputer.Dengan demikian proses eksekusi program dapat dilakukan tanpa mengubah source code program.
\item Integration : phyton dirancang untuk dapat berinteraksi dengan aplikasi lain.Dengan demikian,program yang dibangun dengan bahasa pemrograman lain dapat dieksekusi dengan mudah padascript phyton dengan menggunakan function bahasa pemrograman lainnya.
\end{enumerate}




