%%%%%%%%%%%%%%
%% Run LaTeX on this file several times to get Table of Contents,
%% cross-references, and citations.

%% If you have font problems, you may edit the w-bookps.sty file
%% to customize the font names to match those on your system.

%% w-bksamp.tex. Current Version: Feb 16, 2012
%%%%%%%%%%%%%%%%%%%%%%%%%%%%%%%%%%%%%%%%%%%%%%%%%%%%%%%%%%%%%%%%
%
%  Sample file for
%  Wiley Book Style, Design No.: SD 001B, 7x10
%  Wiley Book Style, Design No.: SD 004B, 6x9
%
%
%  Prepared by Amy Hendrickson, TeXnology Inc.
%  http://www.texnology.com
%%%%%%%%%%%%%%%%%%%%%%%%%%%%%%%%%%%%%%%%%%%%%%%%%%%%%%%%%%%%%%%%

%%%%%%%%%%%%%
% 7x10
%\documentclass{wileySev}

% 6x9
\documentclass{wileySix}

\usepackage{graphicx}

%%%%%%%
%% for times math: However, this package disables bold math (!)
%% \mathbf{x} will still work, but you will not have bold math
%% in section heads or chapter titles. If you don't use math
%% in those environments, mathptmx might be a good choice.

% \usepackage{mathptmx}

% For PostScript text
\usepackage{w-bookps}

%%%%%%%%%%%%%%%%%%%%%%%%%%%%%%%%%%%%%%%%%%%%%%%%%%%%%%%%%%%%%%%%
%% Other packages you might want to use:

% for chapter bibliography made with BibTeX
% \usepackage{chapterbib}

% for multiple indices
% \usepackage{multind}

% for answers to problems
% \usepackage{answers}

%%%%%%%%%%%%%%%%%%%%%%%%%%%%%%
%% Change options here if you want:
%%
%% How many levels of section head would you like numbered?
%% 0= no section numbers, 1= section, 2= subsection, 3= subsubsection
%%==>>
\setcounter{secnumdepth}{3}

%% How many levels of section head would you like to appear in the
%% Table of Contents?
%% 0= chapter titles, 1= section titles, 2= subsection titles, 
%% 3= subsubsection titles.
%%==>>
\setcounter{tocdepth}{2}

%% Cropmarks? good for final page makeup
%% \docropmarks

%%%%%%%%%%%%%%%%%%%%%%%%%%%%%%
%
% DRAFT
%
% Uncomment to get double spacing between lines, current date and time
% printed at bottom of page.
% \draft
% (If you want to keep tables from becoming double spaced also uncomment
% this):
% \renewcommand{\arraystretch}{0.6}
%%%%%%%%%%%%%%%%%%%%%%%%%%%%%%

%%%%%%% Demo of section head containing sample macro:
%% To get a macro to expand correctly in a section head, with upper and
%% lower case math, put the definition and set the box 
%% before \begin{document}, so that when it appears in the 
%% table of contents it will also work:

\newcommand{\VT}[1]{\ensuremath{{V_{T#1}}}}

%% use a box to expand the macro before we put it into the section head:

\newbox\sectsavebox
\setbox\sectsavebox=\hbox{\boldmath\VT{xyz}}

%%%%%%%%%%%%%%%%% End Demo


\begin{document}


\booktitle{Survey Methodology}
\subtitle{This is the Subtitle}

\authors{Robert M. Groves\\
\affil{Universitat de les Illes Balears}
Floyd J. Fowler, Jr.\\
\affil{University of New Mexico}
}

\offprintinfo{Survey Methodology, Second Edition}{Robert M. Groves}

%% Can use \\ if title, and edition are too wide, ie,
%% \offprintinfo{Survey Methodology,\\ Second Edition}{Robert M. Groves}

%%%%%%%%%%%%%%%%%%%%%%%%%%%%%%
%% 
\halftitlepage

\titlepage


\begin{copyrightpage}{2007}
Survey Methodology / Robert M. Groves . . . [et al.].
\       p. cm.---(Wiley series in survey methodology)
\    ``Wiley-Interscience."
\    Includes bibliographical references and index.
\    ISBN 0-471-48348-6 (pbk.)
\    1. Surveys---Methodology.  2. Social 
\  sciences---Research---Statistical methods.  I. Groves, Robert M.  II. %
Series.\\

HA31.2.S873 2007
001.4'33---dc22                                             2004044064
\end{copyrightpage}



\dedication{To my parents}

\begin{contributors}
\name{Masayki Abe,} Fujitsu Laboratories Ltd., Fujitsu Limited, Atsugi,
Japan

\name{L. A. Akers,} Center for Solid State Electronics Research, Arizona
State University, Tempe, Arizona

\name{G. H. Bernstein,} Department of Electrical and
Computer Engineering, University of Notre Dame, Notre Dame, South Bend, 
Indiana; formerly of
Center for Solid State Electronics Research, Arizona
State University, Tempe, Arizona 
\end{contributors}

\contentsinbrief
\tableofcontents
\listoffigures
\listoftables


\begin{foreword}
This is the foreword to the book.
\end{foreword}

\begin{preface}
This is an example preface.
This is an example preface.
This is an example preface.
This is an example preface.

\prefaceauthor{R. K. Watts}
\where{Durham, North Carolina\\
September, 2007}

\end{preface}


\begin{acknowledgments}
From Dr.~Jay Young, consultant from Silver Spring, Maryland, I received
the initial push to even consider writing this book. Jay was a constant
``peer reader'' and very welcome advisor durying this year-long process.


To all these wonderful people I owe a deep sense of gratitude especially now
that this project has been completed.
\authorinitials{G. T. S.}
\end{acknowledgments}

\begin{acronyms}
\acro{ACGIH}{American Conference of Governmental Industrial Hygienists}
\acro{AEC}{Atomic Energy Commission}
\acro{OSHA}{Occupational Health and Safety Commission}
\acro{SAMA}{Scientific Apparatus Makers Association}
\end{acronyms}

\begin{glossary}
\term{NormGibbs}Draw a sample from a posterior distribution
of data with an unknown mean and variance using Gibbs sampling.

\term{pNull}Test a one sided hypothesis from a numberically
specified posterior CDF or from a sample from the posterior

\term{sintegral}A numerical integration using Simpson's rule
\end{glossary}

\begin{symbols}
\term{A}Amplitude

\term{\hbox{\&}}Propositional logic symbol 

\term{a}Filter Coefficient

\bigskip

\term{\mathcal{B}}Number of Beats
\end{symbols}

\begin{introduction}

%% optional, but if you want to list author:

\introauthor{Catherine Clark, PhD.}
{Harvard School of Public Health\\
Boston, MA, USA}

The era of modern \index{microelectronics}\index{microelectronics!modern} 
began in 1958 with the invention of the
integrated circuit by J.~S.~Kilby
 of Texas Instruments \cite{kilby}.
His first chip is shown in Fig.~I. For comparison,
Fig.~I.2 shows a modern microprocessor chip, \cite{beren}.


This is the introduction.
This is the introduction.
This is the introduction.
This is the introduction.
This is the introduction.
This is the introduction.

\begin{equation}
ABC {\cal DEF} \alpha\beta\Gamma\Delta\sum^{abc}_{def}
\end{equation}


\begin{chapreferences}{3.}
\bibitem{zkilby}J. S. Kilby,
``Invention of the Integrated Circuit,'' {\it IEEE Trans. Electron Devices,}
{\bf ED-23,} 648 (1976).

\bibitem{zhamming}R. W. Hamming,
                 {\it Numerical Methods for Scientists and 
                 Engineers}, Chapter N-1, McGraw-Hill, 
                 New York, 1962.

\bibitem{zHu}J. Lee, K. Mayaram, and C. Hu, ``A Theoretical
               Study of Gate/Drain Offset in LDD MOSFETs''
                     {\it IEEE Electron Device Lett.,} {\bf EDL-7}(3). 152 
                     (1986).
\end{chapreferences}
\end{introduction}


\part[Submicron Semiconductor Manufacture]
{Submicron Semiconductor\\ Manufacture}


\chapter[The Submicrometer Silicon MOSFET]
{The Submicrometer\\ Silicon MOSFET}


\prologue{The sheer volumne of answers can often stifle insight...The purpose
of computing\index{computing!the purpose} is insight, not numbers.}
{Hamming \cite{hamming}}


\section{Here is a normal section}
Here is some text.

\subsection{This is the subsection}
Here is some normal text.
Here is some normal text.
Here is some normal text.
Here is some normal text.
Here is some normal text.
Here is some normal text.
Here is some normal text.
Here is some normal text.
Here is some normal text.
Here is some normal text.
Here is some normal text.


\subsubsection{This is the subsubsection}
Here is some text after the subsubsection.
Here is some text after the subsubsection.
Here is some text after the subsubsection.
Here is some text after the subsubsection.

\paragraph{This is the paragraph}
Here is some normal text.
Here is some normal text.
Here is some normal text.
Here is some normal text.

\section{Tips On Special Section Heads}
Here are some things you can do for a special
section head.

\section[This Version of Section Head will be sent Contents]
{Break Long Section heads\\ with double backslash}
Here is some normal text.
Here is some normal text.
Here is some normal text.

 \section[This show how to explicitly break lines
\string\hfill\string\break\space in Table of Contents]
{Here is a Section Title}
See this section head for information on how to explicitly break lines in
table of contents.

\section{How to get \lowercase{lower case} in section head: \lowercase{$p$}$H$}
Here is some normal text.
Here is some normal text.
Here is some normal text.

\section{How to use a macro that has both upper and lower case parts: 
\copy\sectsavebox}
See the top of this file where the definition and box were set.

%% Sending different version of section to running head, 
%% so that the size of math is correct in running head:
\markright{Sample macro \VT{\lowercase{xyz}} sent to running head}

\section{Equation}

For optimal vertical spacing, no blank lines before or after
equations
\begin{equation}
\alpha\beta\Gamma\Delta
\end{equation}
as you see here.


\chapter{First Edited Book Sample Chapter Title}
\chapterauthors{G. Alvarez and R. K. Watts
\chapteraffil{Carnegie Mellon University, Pittsburgh, Pennsylvania}
}

\section{Here is a normal section}
Here is some text.


\chapter{Second Edited Book Sample Chapter Title}
\chapterauthors{George Smeal, Ph.D.\affilmark{1}, Sally Smith,
M.D.\affilmark{2} and Stanley Kubrick\affilmark{1}
\chapteraffil{\affilmark{1}AT\&T Bell Laboratories
Murray Hill, New Jersey\\
\affilmark{2}Harvard Medical School,
Boston, Massachusetts}
}

\section{Sample Section}
Here is some sample text.

\newpage

\section{Example, Figure and Tables}
\vskip6pt
\begin{example}[Optional Example Name]
Use Black's law [Equation (6.3)] to estimate the reduction in useful product
life if a metal line is initially run at 55$^\circ$C at a maximum line
current density.
\end{example}




\begin{figure}[ht]
illustration here
%\centerline{\includegraphics[width=.5\textwidth]{filename}}
\caption{Short figure caption.}
\end{figure}

\begin{figure}[ht]
\vskip2pt
\caption{Oscillograph for  memory address access operations,
showing 500 ps
address access time and superimposed signals
of address access in 1 kbit
memory plane.}
\end{figure}

\begin{table}[ht]
\caption{Small Table}
\centering
\begin{tabular}{cccc}
\hline
one&two&three&four\\
\hline
C&D&E&F\\
\hline
\end{tabular}
\end{table}



\begin{table}[ht]
\caption{Effects of the two types of $\alpha\beta\sum^A_B$ scaling proposed by Dennard \newline
and
co-workers$^{a,b}$}
\begin{tabular*}{\textwidth}{@{\extracolsep{\fill}}lcc}
\hline
Parameter& $\kappa$ Scaling & $\kappa$, $\lambda$ Scaling\cr
\hline
Dimension&$\kappa^{-1}$&$\lambda^{-1}$\cr
Voltage&$\kappa^{-1}$&$\kappa^{-1}$\cr
Currant&$\kappa^{-1}$&$\lambda/\kappa^{2}$\cr
Dopant Concentration&$\kappa$&$\lambda^2/\kappa$\cr
\hline
\end{tabular*}
\begin{tablenotes}
$^a$Refs.~19 and 20.

$^b\kappa, \lambda>1$.
\end{tablenotes}
\end{table}

\subsection{Side by Side Tables and Figures}

\begin{figure}[ht]
\sidebyside{
Space for figure...
\caption{This caption will go on the left side of
the page. It is the initial caption of two side-by-side captions.}
}
{
Space for second figure...
\caption{This caption will go on the right side of
the page. It is the second of two side-by-side captions.}
}
\end{figure}


The command \verb+\sidebyside{}{}+ works similarly for tables:

 \begin{table}[ht]
 \sidebyside{
\caption{Table Caption} 
\begin{tabular}{cccc}
one&two&three&four\\
a &little&sample&table
\end{tabular}
}
 {
\caption{Table Caption}
\begin{tabular}{cccc}
A&B&C&D\\
a &second little& sample&table
\end{tabular}
}
 \end{table}


When using \verb+\sidebyside+, one must
use the cross referencing command \verb+\label{}+ after and  {\it outside} 
 of \verb+\caption{}+:

\begin{verbatim}
 \begin{table} 
 \sidebyside{\caption{Table Caption}\label{tab1}
 first table}
 {\caption{Table Caption}\label{tab2} second table}
 \end{table}
\end{verbatim}
 or,
\begin{verbatim}
 \begin{figure} 
 \sidebyside{\vskip<dimen>\caption{fig caption}\label{fig1}}
 {\vskip<dimen>\caption{fig caption}\label{fig2}}
 \end{figure}
\end{verbatim}





\section{Algorithm}
This is a sample algorithm.

\begin{algorithm}
{\bf state\_transition algorithm} $\{$
\        for each neuron $j\in\{0,1,\ldots,M-1\}$
\        $\{$   
\            calculate the weighted sum $S_j$ using Eq. (6);
\            if ($S_j>t_j$)
\                    $\{$turn ON neuron; $Y_1=+1\}$   
\            else if ($S_j<t_j$)
\                    $\{$turn OFF neuron; $Y_1=-1\}$   
\            else
\                    $\{$no change in neuron state; $y_j$ remains %
unchanged;$\}$ 
\        $\}$   
$\}$   
\end{algorithm}

Here is some normal text.
Here is some normal text.
Here is some normal text.
Here is some normal text.
Here is some normal text.
Here is some normal text.
Here is some normal text.
Here is some normal text.
Here is some normal text.
Here is some normal text.
Here is some normal text.
Here is some normal text.
Here is some normal text.
Here is some normal text.


\begin{quote}
This is a sample of extract or quotation.
This is a sample of extract or quotation.
This is a sample of extract or quotation.
\end{quote}

\begin{enumerate}
\item
This is the first item in the numbered list.

\item
This is the second item in the numbered list.
This is the second item in the numbered list.
This is the second item in the numbered list.
\end{enumerate}

\begin{itemize}
\item
This is the first item in the itemized list.

\item
This is the first item in the itemized list.
This is the first item in the itemized list.
This is the first item in the itemized list.
\end{itemize}

\begin{itemize}
\item[]
This is the first item in the itemized list.

\item[]
This is the first item in the itemized list.
This is the first item in the itemized list.
This is the first item in the itemized list.
\end{itemize}

\begin{problems}
\prob
For Hooker's data, Problem 1.2, use the Box and Cox and Atkinson procedures to determine a appropriate transformation of PRES
in the regression of PRES on TEMP. find $\hat\lambda$, $\tilde\lambda$,
the score test, and the added variable plot for the score. 
Summarize the results.

\prob
The following data were collected in a study of the effect of dissolved sulfur
on the surface tension of liquid copper (Baes and Killogg, 1953).

{\centering
\vskip6pt
\begin{tabular}{rlcc}
\hline
&&\multicolumn2c{$Y$= Decrease in Surface Tension}\\
\multicolumn2c{$x$ = Weight \% sulfur}
&\multicolumn2c{(dynes/cm), two Replicates}\\
\hline
0.&034&301&316\\
0.&093&430&422\\
0.&30&593&586\\
\hline
\end{tabular}
\vskip6pt
}


\subprob
Find the transformations of $X$ and $Y$ sot that in the transformed scale 
the regression is linear.

\subprob
Assuming that $X$ is transformed to $\ln(X)$, which choice of $Y$ gives 
better results,
$Y$ or $\ln(Y)$? (Sclove, 1972).

\sidebysidesubprob{In the case of $\alpha_1$?}{In the case of $\alpha_2$?}

\prob
Examine the Longley data, Problem 3.3, for applicability of assumptions of the
linear model.

\sidebysideprob{In the case of $\Gamma_1$?}{In the case of $\Gamma_2$?}

\end{problems}


\begin{exercises}
\exer
For Hooker's data, Exercise 1.2, use the Box and Cox and Atkinson procedures to determine a appropriate transformation of PRES
in the regression of PRES on TEMP. find $\hat\lambda$, $\tilde\lambda$,
the score test, and the added variable plot for the score. 
Summarize the results.

\exer
The following data were collected in a study of the effect of dissolved sulfur
on the surface tension of liquid copper (Baes and Killogg, 1953).

{\centering
\vskip6pt
\begin{tabular}{rlcc}
\hline
&&\multicolumn2c{$Y$= Decrease in Surface Tension}\\
\multicolumn2c{$x$ = Weight \% sulfur}
&\multicolumn2c{(dynes/cm), two Replicates}\\
\hline
0.&034&301&316\\
0.&093&430&422\\
0.&30&593&586\\
\hline
\end{tabular}
\vskip6pt
}


\subexer
Find the transformations of $X$ and $Y$ sot that in the transformed scale 
the regression is linear.

\subexer
Assuming that $X$ is transformed to $\ln(X)$, which choice of $Y$ gives 
better results,
$Y$ or $\ln(Y)$? (Sclove, 1972).

\sidebysidesubexer{In the case of $\Delta_1$?}{In the case of $\Delta_2$?}

\exer
Examine the Longley data, Problem 3.3, for applicability of assumptions of the
linear model.

\sidebysideexer{In the case of $\Gamma_1$?}{In the case of $\Gamma_2$?}

\end{exercises}


\section{Summary}
This is a summary of this chapter.
Here are some references: \cite{xkilby}, \cite{xberen}.

\chapter{Home}

\section{Sample Section}
Here is some sample text.

\section{Example, Figure and Tables}
\vskip6pt
\begin{example}[Optional Example Name]
	Use Black's law [Equation (6.3)] to estimate the reduction in useful product
	life if a metal line is initially run at 55$^\circ$C at a maximum line
	current density.
\end{example}

\section{Algorithm}
This is a sample algorithm.

\section{Summary}
This is a summary of this chapter.
Here are some references: \cite{xkilby}, \cite{xberen}.

\chapter{Overview}

\section{Sample Section}
Here is some sample text.

\section{Example, Figure and Tables}
\vskip6pt
\begin{example}[Optional Example Name]
	Use Black's law [Equation (6.3)] to estimate the reduction in useful product
	life if a metal line is initially run at 55$^\circ$C at a maximum line
	current density.
\end{example}

\section{Algorithm}
This is a sample algorithm.

\section{Summary}
This is a summary of this chapter.
Here are some references: \cite{xkilby}, \cite{xberen}.

\chapter{Environtment Setup}

\section{Sample Section}
Here is some sample text.

\section{Example, Figure and Tables}
\vskip6pt
\begin{example}[Optional Example Name]
	Use Black's law [Equation (6.3)] to estimate the reduction in useful product
	life if a metal line is initially run at 55$^\circ$C at a maximum line
	current density.
\end{example}

\section{Algorithm}
This is a sample algorithm.

\section{Summary}
This is a summary of this chapter.
Here are some references: \cite{xkilby}, \cite{xberen}.

\chapter{Basic Syntax}

\section{Sample Section}
Here is some sample text.

\section{Example, Figure and Tables}
\vskip6pt
\begin{example}[Optional Example Name]
	Use Black's law [Equation (6.3)] to estimate the reduction in useful product
	life if a metal line is initially run at 55$^\circ$C at a maximum line
	current density.
\end{example}

\section{Algorithm}
This is a sample algorithm.

\section{Summary}
This is a summary of this chapter.
Here are some references: \cite{xkilby}, \cite{xberen}.

\chapter{Variabel Type}

\section{Sample Section}
Here is some sample text.

\section{Example, Figure and Tables}
\vskip6pt
\begin{example}[Optional Example Name]
	Use Black's law [Equation (6.3)] to estimate the reduction in useful product
	life if a metal line is initially run at 55$^\circ$C at a maximum line
	current density.
\end{example}

\section{Algorithm}
This is a sample algorithm.

\section{Summary}
This is a summary of this chapter.
Here are some references: \cite{xkilby}, \cite{xberen}.

\chapter{Basic Operator}

\section{Sample Section}
Here is some sample text.

\section{Example, Figure and Tables}
\vskip6pt
\begin{example}[Optional Example Name]
	Use Black's law [Equation (6.3)] to estimate the reduction in useful product
	life if a metal line is initially run at 55$^\circ$C at a maximum line
	current density.
\end{example}

\section{Algorithm}
This is a sample algorithm.

\section{Summary}
This is a summary of this chapter.
Here are some references: \cite{xkilby}, \cite{xberen}.

\chapter{Desicion Making}

\section{Sample Section}
Here is some sample text.

\section{Example, Figure and Tables}
\vskip6pt
\begin{example}[Optional Example Name]
	Use Black's law [Equation (6.3)] to estimate the reduction in useful product
	life if a metal line is initially run at 55$^\circ$C at a maximum line
	current density.
\end{example}

\section{Algorithm}
This is a sample algorithm.

\section{Summary}
This is a summary of this chapter.
Here are some references: \cite{xkilby}, \cite{xberen}.

\chapter{Loop}

\section{Sample Section}
Here is some sample text.

\section{Example, Figure and Tables}
\vskip6pt
\begin{example}[Optional Example Name]
	Use Black's law [Equation (6.3)] to estimate the reduction in useful product
	life if a metal line is initially run at 55$^\circ$C at a maximum line
	current density.
\end{example}

\section{Algorithm}
This is a sample algorithm.

\section{Summary}
This is a summary of this chapter.
Here are some references: \cite{xkilby}, \cite{xberen}.

\chapter{Numbers}

\section{Sample Section}
Here is some sample text.

\section{Example, Figure and Tables}
\vskip6pt
\begin{example}[Optional Example Name]
	Use Black's law [Equation (6.3)] to estimate the reduction in useful product
	life if a metal line is initially run at 55$^\circ$C at a maximum line
	current density.
\end{example}

\section{Algorithm}
This is a sample algorithm.

\section{Summary}
This is a summary of this chapter.
Here are some references: \cite{xkilby}, \cite{xberen}.

\chapter{Strings}

\section{Sample Section}
Here is some sample text.

\section{Example, Figure and Tables}
\vskip6pt
\begin{example}[Optional Example Name]
	Use Black's law [Equation (6.3)] to estimate the reduction in useful product
	life if a metal line is initially run at 55$^\circ$C at a maximum line
	current density.
\end{example}

\section{Algorithm}
This is a sample algorithm.

\section{Summary}
This is a summary of this chapter.
Here are some references: \cite{xkilby}, \cite{xberen}.

\chapter{Lists}

\section{Sample Section}
Here is some sample text.

\section{Example, Figure and Tables}
\vskip6pt
\begin{example}[Optional Example Name]
	Use Black's law [Equation (6.3)] to estimate the reduction in useful product
	life if a metal line is initially run at 55$^\circ$C at a maximum line
	current density.
\end{example}

\section{Algorithm}
This is a sample algorithm.

\section{Summary}
This is a summary of this chapter.
Here are some references: \cite{xkilby}, \cite{xberen}.

\chapter{Tuples}

\section{Sample Section}
Here is some sample text.

\section{Example, Figure and Tables}
\vskip6pt
\begin{example}[Optional Example Name]
	Use Black's law [Equation (6.3)] to estimate the reduction in useful product
	life if a metal line is initially run at 55$^\circ$C at a maximum line
	current density.
\end{example}

\section{Algorithm}
This is a sample algorithm.

\section{Summary}
This is a summary of this chapter.
Here are some references: \cite{xkilby}, \cite{xberen}.

\chapter{Dictionary}

\section{Sample Section}
Here is some sample text.

\section{Example, Figure and Tables}
\vskip6pt
\begin{example}[Optional Example Name]
	Use Black's law [Equation (6.3)] to estimate the reduction in useful product
	life if a metal line is initially run at 55$^\circ$C at a maximum line
	current density.
\end{example}

\section{Algorithm}
This is a sample algorithm.

\section{Summary}
This is a summary of this chapter.
Here are some references: \cite{xkilby}, \cite{xberen}.

\chapter{Date & Time}

\section{Sample Section}
Here is some sample text.

\section{Example, Figure and Tables}
\vskip6pt
\begin{example}[Optional Example Name]
	Use Black's law [Equation (6.3)] to estimate the reduction in useful product
	life if a metal line is initially run at 55$^\circ$C at a maximum line
	current density.
\end{example}

\section{Algorithm}
This is a sample algorithm.

\section{Summary}
This is a summary of this chapter.
Here are some references: \cite{xkilby}, \cite{xberen}.

\chapter{Functions}

\section{Sample Section}
Here is some sample text.

\section{Example, Figure and Tables}
\vskip6pt
\begin{example}[Optional Example Name]
	Use Black's law [Equation (6.3)] to estimate the reduction in useful product
	life if a metal line is initially run at 55$^\circ$C at a maximum line
	current density.
\end{example}

\section{Algorithm}
This is a sample algorithm.

\section{Summary}
This is a summary of this chapter.
Here are some references: \cite{xkilby}, \cite{xberen}.

\chapter{Modules}

\section{Sample Section}
Here is some sample text.

\section{Example, Figure and Tables}
\vskip6pt
\begin{example}[Optional Example Name]
	Use Black's law [Equation (6.3)] to estimate the reduction in useful product
	life if a metal line is initially run at 55$^\circ$C at a maximum line
	current density.
\end{example}

\section{Algorithm}
This is a sample algorithm.

\section{Summary}
This is a summary of this chapter.
Here are some references: \cite{xkilby}, \cite{xberen}.

\chapter{Files I/O}

\section{Sample Section}
Here is some sample text.

\section{Example, Figure and Tables}
\vskip6pt
\begin{example}[Optional Example Name]
	Use Black's law [Equation (6.3)] to estimate the reduction in useful product
	life if a metal line is initially run at 55$^\circ$C at a maximum line
	current density.
\end{example}

\section{Algorithm}
This is a sample algorithm.

\section{Summary}
This is a summary of this chapter.
Here are some references: \cite{xkilby}, \cite{xberen}.

\chapter{Exceptions}

\section{Sample Section}
Here is some sample text.

\section{Example, Figure and Tables}
\vskip6pt
\begin{example}[Optional Example Name]
	Use Black's law [Equation (6.3)] to estimate the reduction in useful product
	life if a metal line is initially run at 55$^\circ$C at a maximum line
	current density.
\end{example}

\section{Algorithm}
This is a sample algorithm.

\section{Summary}
This is a summary of this chapter.
Here are some references: \cite{xkilby}, \cite{xberen}.

\chapter{Clasess/Object}

\section{Sample Section}
Here is some sample text.

\section{Example, Figure and Tables}
\vskip6pt
\begin{example}[Optional Example Name]
	Use Black's law [Equation (6.3)] to estimate the reduction in useful product
	life if a metal line is initially run at 55$^\circ$C at a maximum line
	current density.
\end{example}

\section{Algorithm}
This is a sample algorithm.

\section{Summary}
This is a summary of this chapter.
Here are some references: \cite{xkilby}, \cite{xberen}.

\chapter{Reg Expression}

\section{Sample Section}
Here is some sample text.

\section{Example, Figure and Tables}
\vskip6pt
\begin{example}[Optional Example Name]
	Use Black's law [Equation (6.3)] to estimate the reduction in useful product
	life if a metal line is initially run at 55$^\circ$C at a maximum line
	current density.
\end{example}

\section{Algorithm}
This is a sample algorithm.

\section{Summary}
This is a summary of this chapter.
Here are some references: \cite{xkilby}, \cite{xberen}.

\chapter{CGI Programming}

\section{Sample Section}
Here is some sample text.

\section{Example, Figure and Tables}
\vskip6pt
\begin{example}[Optional Example Name]
	Use Black's law [Equation (6.3)] to estimate the reduction in useful product
	life if a metal line is initially run at 55$^\circ$C at a maximum line
	current density.
\end{example}

\section{Algorithm}
This is a sample algorithm.

\section{Summary}
This is a summary of this chapter.
Here are some references: \cite{xkilby}, \cite{xberen}.

\chapter{Databases Access}

\section{Sample Section}
Here is some sample text.

\section{Example, Figure and Tables}
\vskip6pt
\begin{example}[Optional Example Name]
	Use Black's law [Equation (6.3)] to estimate the reduction in useful product
	life if a metal line is initially run at 55$^\circ$C at a maximum line
	current density.
\end{example}

\section{Algorithm}
This is a sample algorithm.

\section{Summary}
This is a summary of this chapter.
Here are some references: \cite{xkilby}, \cite{xberen}.

\chapter{Networking}

\section{Sample Section}
Here is some sample text.

\section{Example, Figure and Tables}
\vskip6pt
\begin{example}[Optional Example Name]
	Use Black's law [Equation (6.3)] to estimate the reduction in useful product
	life if a metal line is initially run at 55$^\circ$C at a maximum line
	current density.
\end{example}

\section{Algorithm}
This is a sample algorithm.

\section{Summary}
This is a summary of this chapter.
Here are some references: \cite{xkilby}, \cite{xberen}.

\chapter{Sending Email}

\section{Sample Section}
Here is some sample text.

\section{Example, Figure and Tables}
\vskip6pt
\begin{example}[Optional Example Name]
	Use Black's law [Equation (6.3)] to estimate the reduction in useful product
	life if a metal line is initially run at 55$^\circ$C at a maximum line
	current density.
\end{example}

\section{Algorithm}
This is a sample algorithm.

\section{Summary}
This is a summary of this chapter.
Here are some references: \cite{xkilby}, \cite{xberen}.

\chapter[Python Multithread Programming]
{Python Multithread Programming}
\\}\end{center} \par
\vspace{14pt}
\vspace{14pt}
Menjalankan beberapa\textit{ thread} mirip dengan menjalankan beberapa program yang berbeda secara bersamaan, namun dengan manfaat berikut : \par
\begin{itemize}
\item Beberapa \textit{thread} dalam proses berbagi ruang data yang sama dengan benang induk dan karena dapat saling berbagi informasi atau berkomunikasi satu sama lain dengan lebih muda daripada jika prosesnya terpisah \par
\item \textit{thread} terkadang disebut proses ringan dan tidak membutuhkan banyak memori atas, mereka lebih murah daripada proses.\end{itemize}
\par
Sebuah \textit{thread} memiliki permulaan, urutan eksekusi dan sebuah kesimpulan. Ini memiliki pointer perintah yang melacak dari mana dalam konteksnya saat ini berjalan.  \par
\noindent 
\begin{itemize}
\item Hal ini dapat dilakukan sebelum pre-\textit{empted} (\textit{inturrepted}) \par
\noindent 
\item Untuk sementara dapat ditunda sementara \textit{thread} lainnya yang sedang berjalan ini disebut unggul. \end{itemize}
\par
\noindent 

\section{Memulai Thread Baru}
\\}\end{center \par
\noindent 
\hspace*{0.5in} Untuk melakukan \textit{thread} lain, perlu memanggil metode berikut yang tersedia dimodul \textit{thread} : \par
\noindent 
\begin{center}{\fontsize{9pt}{9pt}\selectfont Thread.start $  \_  $new $  \_  $thread (function, args [, kwargs] )}\end{center} \par
\vspace{12pt}
Pemanggilan metode ini memungkinkan cara cepat dan tepat untuk membuat \textit{thread} baru di linux dan window. \par
Pemanggilan metode segera kembali dan anak  \textit{thread} dimulai dan fungsi pemanggilan dengan daftar \textit{args} telah berlalu. Saat fungsi kembali ujung \textit{thread} akan berakhir.   \par
Disini, \textit{args }adalah tupel argumen. Gunakan tupel kosong untuk memanggil fungsi tanpa melewati argumen. \textit{Kwargs} adalah kamus opsional argumen kata kunci.  \par
\noindent 
\par
\noindent 
Contoh : \par
\noindent 
{\fontsize{10pt}{10pt}\selectfont  $  \#  $!/usr/bin/python} \par
\vspace{10pt}
\noindent 
{\fontsize{10pt}{10pt}\selectfont Import thread} \par
\noindent 
{\fontsize{10pt}{10pt}\selectfont Import time} \par
\vspace{10pt}
\noindent 
{\fontsize{10pt}{10pt}\selectfont  $  \#  $ Define a function for the thread} \par
\noindent 
{\fontsize{10pt}{10pt}\selectfont Def print $  \_  $time (threadNamw, delay):} \par
\noindent 
{\fontsize{10pt}{10pt}\selectfont  \hspace*{0.5in} Count = 0} \par
\noindent 
{\fontsize{10pt}{10pt}\selectfont  \hspace*{0.5in} While count <5:} \par
\noindent 
{\fontsize{10pt}{10pt}\selectfont  \hspace*{0.5in} Time.sleep(delay)} \par
\noindent 
{\fontsize{10pt}{10pt}\selectfont  \hspace*{0.5in} Count +=1} \par
\noindent 
{\fontsize{10pt}{10pt}\selectfont  \hspace*{0.5in} Print  $ " $ $  \%  $s :  $  \%  $s $ " $  $  \%  $ (threadName, time.ctime(time.time()))} \par
\vspace{10pt}
\noindent 
{\fontsize{10pt}{10pt}\selectfont  $  \#  $ Create two thread as follows} \par
\noindent 
{\fontsize{10pt}{10pt}\selectfont try:} \par
\noindent 
{\fontsize{10pt}{10pt}\selectfont  thread.start $  \_  $new $  \_  $thread(print $  \_  $time, ( $ " $Thread-1 $ " $, 2, ))} \par
\noindent 
{\fontsize{10pt}{10pt}\selectfont  thread.start $  \_  $new $  \_  $thread(print $  \_  $time, ( $ " $Thread-2 $ " $, 4,))} \par
\noindent 
{\fontsize{10pt}{10pt}\selectfont except:} \par
\noindent 
{\fontsize{10pt}{10pt}\selectfont ~~ print  $ " $Error: unable to start thread $ " $} \par
\vspace{10pt}
\noindent 
{\fontsize{10pt}{10pt}\selectfont while 1:} \par
\noindent 
{\fontsize{10pt}{10pt}\selectfont pass} \par
\noindent 
~~~~~~ Bila kode diatas dieksekusi, maka menghasilkan hasil sebagai berikut : \par
\noindent 
\begin{center}{\fontsize{10pt}{10pt}\selectfont Thread-1 : Thu Jan 22 15:42:17 2009}\end{center} \par
\noindent 
\begin{center}{\fontsize{10pt}{10pt}\selectfont Thread-1 : Thu Jan 22 15:42:19 2009}\end{center} \par
\noindent 
\begin{center}{\fontsize{10pt}{10pt}\selectfont Thread-2 : Thu Jan 22 15:42:19 2009}\end{center} \par
\noindent 
\begin{center}{\fontsize{10pt}{10pt}\selectfont Thread-1 : Thu Jan 22 15:42:21 2009}\end{center} \par
\noindent 
\begin{center}{\fontsize{10pt}{10pt}\selectfont Thread-2 : Thu Jan 22 15:42:23 2009}\end{center} \par
\noindent 
\begin{center}{\fontsize{10pt}{10pt}\selectfont Thread-1 : Thu Jan 22 15:42:23 2009}\end{center} \par
\noindent 
\begin{center}{\fontsize{10pt}{10pt}\selectfont Thread-1~:  Thu Jan 22 15:42:23 2009}\end{center} \par
\noindent 
\begin{center}{\fontsize{10pt}{10pt}\selectfont Thread-1 : Thu Jan 22 15:42:25 2009}\end{center} \par
\noindent 
\begin{center}{\fontsize{10pt}{10pt}\selectfont Thread-2 : Thu Jan 22 15:42:27 2009}\end{center} \par
\noindent 
\begin{center}{\fontsize{10pt}{10pt}\selectfont Thread-2 : Thu Jan 22 15:42:31 2009}\end{center} \par
\noindent 
\begin{center}{\fontsize{10pt}{10pt}\selectfont Thread-2 : Thu Jan 22 15:42:35 2009}\end{center} \par
\vspace{12pt}
Meskipun sangat efektif untuk benang tingkat rendah, namun modul \textit{thread} sangat terbatas dibandingkan dengan modul yang baru. \par
\vspace{12pt}
\section{Modul Threading }
\\}\end{center} \par
Modul threading yang lebih baru disertakan dengan Python 2.4 memberikan jauh lebih kuat, dukungan tingkat tinggi untuk \textit{thread}\textit{ }dari modul\textit{ }\textit{thread}\textit{ }dibahas pada bagian sebelumnya. \par
The \textit{thread}\textit{ing }modul mengekpos semua metode dari \textit{thread}\textit{ }dan menyediakan beberapa metode tambahan : \par
\begin{itemize}
	\item \textbf{t}\textbf{hreading.activeCount() } \par
	Mengembalikan jumlah objek \textit{thread} yang aktif \par
	\item \textbf{t}\textbf{hreading.currentThread() } \par
	Mengembalikan jumlah objek \textit{thread} dalam kontrol benang pemanggil \par
	\item \textbf{t}\textbf{hreading.enumerate() } \par
	Mengembalikan daftar semua benda \textit{thread}\textit{ }yang sedang aktif \par
	\vspace{12pt}
	Selain metode, modul \textit{thread}\textit{ing }memiliki \textit{thread}\textit{ }kelas yang mengimplementasikan \textit{thread}\textit{ing. }Metode yang disediakan oleh \textit{thread}\textit{ }kelas adalah sebagai berikut : \par
	\item \textbf{run()} \par
	Metode adalah titik masuk untuk \textit{thread} \par
	\item \textbf{start()} \par
	Metode dimulai\textbf{ }\textit{thread}\textit{ }dengan memanggil metode run \par
	\item \textbf{join(}\textbf{[time]}\textbf{)} \par
	Menunggu benang untuk mengakhiri \par
	\item \textbf{isAlive()} \par
	Metode memeriksa apakah\textbf{ }\textit{thread}\textit{ }masih mengeksekusi\textbf{ } \par
	\item \textbf{getName()} \par
	Metode mengambalikan nama\textbf{ }\textit{thread} \par
	\item \textbf{setName()} \par
	Metode menetapkan nama\textbf{ }\textit{thread} \par
	\vspace{12pt}
\section{Membuat Thread Menggunakan Threading Modul}
\\}\end{center} \par
Untuk melaksanakan \textit{thread}\textit{ }baru menggunakan\textit{ threading} harus melakukan hal berikut : \par
\item \textbf Mendefinisikan subclass dari \textit{thread} kelas \par
\item Menimpa  $  \_  $init $  \_  $ (self [args]) metode untuk menambahkan argumen tambahan \par
\item Menimpa run(self[args]) metode untuk menerapkan apa \textit{thread} harus dilakukan ketika mulai \end{itemize}
\par
\begin{adjustwidth}
Setelah membuat baru \textit{thread} subclass, dapat membuah seuah instance dari itu dan kemudian memulai \textit{thread} baru dengan menerapkan \textit{start(),} yang ada gilirinnya panggilan \textit{run()} metode.\end{adjustwidth}
\par
\vspace{12pt}
\begin{adjustwidth}
Contoh :\end{adjustwidth}
\par
\noindent 
{\fontsize{10pt}{10pt}\selectfont  $  \#  $!/usr/bin/python} \par
\vspace{10pt}
\noindent 
{\fontsize{10pt}{10pt}\selectfont import threading} \par
\noindent 
{\fontsize{10pt}{10pt}\selectfont import time} \par
\vspace{10pt}
\noindent 
{\fontsize{10pt}{10pt}\selectfont exitFlag = 0} \par
\vspace{10pt}
\noindent 
{\fontsize{10pt}{10pt}\selectfont class myThread (threading.Thread):} \par
\noindent 
{\fontsize{10pt}{10pt}\selectfont  \hspace*{0.5in} def $  \_  $init $  \_  $(self, threadID, name, counter) :} \par
\noindent 
{\fontsize{10pt}{10pt}\selectfont ~~~~~~ threading.Thread. $  \_  $init $  \_  $(self)} \par
\noindent 
{\fontsize{10pt}{10pt}\selectfont  \hspace*{0.5in} self.threadID = threadID} \par
\noindent 
{\fontsize{10pt}{10pt}\selectfont  \hspace*{0.5in} self.name = name} \par
\vspace{10pt}
\vspace{10pt}
\vspace{10pt}
\noindent 


%%%%%%%%%%%%  Start New Page here %%%%%%%%%%%%%%


\newpage

{\fontsize{9pt}{9pt}\selectfont self.counter = counter} \par
\noindent 
{\fontsize{9pt}{9pt}\selectfont def run (self) :} \par
\noindent 
{\fontsize{9pt}{9pt}\selectfont  \hspace*{0.5in} print  $ " $Starting  $ " $ + self.name} \par
\noindent 
{\fontsize{9pt}{9pt}\selectfont  \hspace*{0.5in} print $  \_  $time(self.name, self.counter, 5)} \par
\noindent 
{\fontsize{9pt}{9pt}\selectfont  \hspace*{0.5in} print  $ " $Exiting  $ " $+ self.name} \par
\vspace{9pt}
\noindent 
{\fontsize{9pt}{9pt}\selectfont def print $  \_  $time(threadName, delay, counter):} \par
\noindent 
{\fontsize{9pt}{9pt}\selectfont while counter:} \par
\noindent 
{\fontsize{9pt}{9pt}\selectfont  \hspace*{0.5in} if exitFlag:} \par
\noindent 
{\fontsize{9pt}{9pt}\selectfont  \hspace*{0.5in}  \hspace*{0.5in} threadName.exit()} \par
\noindent 
{\fontsize{9pt}{9pt}\selectfont  \hspace*{0.5in} time.sleep(delay)} \par
\noindent 
{\fontsize{9pt}{9pt}\selectfont  \hspace*{0.5in} print  $ " $ $  \%  $s:  $  \%  $s $ " $  $  \%  $ (threadName, time.ctime(time.time()))} \par
\noindent 
{\fontsize{9pt}{9pt}\selectfont counter -= 1} \par
\vspace{9pt}
\noindent 
{\fontsize{9pt}{9pt}\selectfont  $  \#  $ Create new threads} \par
\noindent 
{\fontsize{9pt}{9pt}\selectfont thread1 = myThread(1,  $ " $Thread-1 $ " $, 1)} \par
\noindent 
{\fontsize{9pt}{9pt}\selectfont thread2 = myThread(2,  $ " $Thread-2 $ " $, 2)} \par
\vspace{9pt}
\noindent 
{\fontsize{9pt}{9pt}\selectfont  $  \#  $ Start new threads} \par
\noindent 
{\fontsize{9pt}{9pt}\selectfont thread1.start()} \par
\noindent 
{\fontsize{9pt}{9pt}\selectfont thread2.start()} \par
\noindent 
{\fontsize{9pt}{9pt}\selectfont print  $ " $Exiting Main Thread $ " $} \par
\vspace{12pt}
\begin{adjustwidth}
Ketika kode diatas dijalankan, menghasilkan hasil sebagai berikut:\end{adjustwidth}
\par
\noindent 
{\fontsize{10pt}{10pt}\selectfont Starting Thread-1} \par
\noindent 
{\fontsize{10pt}{10pt}\selectfont Starting Thread-2} \par
\noindent 
{\fontsize{10pt}{10pt}\selectfont Exiting Main Thread} \par
\noindent 
{\fontsize{10pt}{10pt}\selectfont Thread-1 : Thu Mar 21 09:10:03 2013} \par
\noindent 
{\fontsize{10pt}{10pt}\selectfont Thread-1 : Thu Mar 21 09:10:04 2013} \par
\noindent 
{\fontsize{10pt}{10pt}\selectfont Thread-2 : Thu Mar 21 09:10:04 2013} \par
\noindent 
{\fontsize{10pt}{10pt}\selectfont Thread-1 : Thu Mar 21 09:10:05 2013} \par
\noindent 
{\fontsize{10pt}{10pt}\selectfont Thread-2 : Thu Mar 21 09:10:06 2013} \par
\noindent 
{\fontsize{10pt}{10pt}\selectfont Thread-1 : Thu Mar 21 09:10:07 2013} \par
\noindent 
{\fontsize{10pt}{10pt}\selectfont Exiting Thread-1} \par
\noindent 
{\fontsize{10pt}{10pt}\selectfont Thread-2 : Thu Mar 21 09:10:08 2013} \par
\noindent 
{\fontsize{10pt}{10pt}\selectfont Thread-2 : Thu Mar 21 09:10:10 2013} \par
\noindent 
{\fontsize{10pt}{10pt}\selectfont Thread-2 : Thu Mar 21 09:10:12 2013} \par
\noindent 
{\fontsize{10pt}{10pt}\selectfont Exiting Thread=2} \par
\vspace{12pt}
\section{Sinkronisasi Thread}
\\}\end{center} \par
{T}\textit{hread}\textit{ing }modul disediakan dengan Python termasuk sederhana untuk menerapkan mekanisme bahwa memungkinkan untuk menyinkronkan \textit{thread}\textit{ }penguncian. Sebuah kunci baru dibuat dengan memanggil \textit{lock() }metode yang mengembalikan kunci baru. \par
The \textit{acquire}\textit{ }\textit{(blocking)}\textit{ }metode objek kunci baru digunakan untuk memaksa \textit{thread}\textit{ }untuk menjalankan serempak. Opsional \textit{blocking} parameter memungkikan untuk mengontrol apakah\textit{ thread} menunggu untuk mendapatkan kunci. \par
Jika \textit{blocking} diatur ke 0, \textit{thread} segera kembali dengan nilai 0 jika kunci tidak dapat diperoleh dan dengan 1 jika kunci dikuisisi. Jika pemblokiran diatur ke 1, blok dan menunggu kunci yang akan dirilis. \par
The \textit{release()} metode objek kunci baru digunakan untuk melepaskan kunci ketika tidak lagi diperlukan.  \par
Contoh: \par
\noindent 
{\fontsize{10pt}{10pt}\selectfont  $  \#  $!/usr/bin/python} \par
\vspace{10pt}
\noindent 
{\fontsize{10pt}{10pt}\selectfont import threading} \par
\noindent 
{\fontsize{10pt}{10pt}\selectfont import time} \par
\vspace{10pt}
\noindent 
{\fontsize{10pt}{10pt}\selectfont class myThread (threading.Thread):} \par
\noindent 
{\fontsize{10pt}{10pt}\selectfont ~ def $  \_  $init $  \_  $(self, threadID, name, counter):} \par
\noindent 
{\fontsize{10pt}{10pt}\selectfont ~~~~ threading.Thread. $  \_  $init $  \_  $(self)} \par
\noindent 
{\fontsize{10pt}{10pt}\selectfont ~~~~ self.threadID = threadID} \par
\noindent 
{\fontsize{10pt}{10pt}\selectfont ~~~~ self.name = name} \par
\noindent 
{\fontsize{10pt}{10pt}\selectfont ~~~~ self.counter = counter} \par
\noindent 
{\fontsize{10pt}{10pt}\selectfont ~ def run(self)} \par
\noindent 
{\fontsize{10pt}{10pt}\selectfont ~~~~ print  $ " $Starting  $ " $+ self.name} \par
\noindent 
{\fontsize{10pt}{10pt}\selectfont ~~~~  $  \#  $ Get lock to synchronize threads} \par
\noindent 
{\fontsize{10pt}{10pt}\selectfont ~~~~ ThreadLock.acquire()} \par
\noindent 
{\fontsize{10pt}{10pt}\selectfont ~~~~ print $  \_  $time(self.name, self.counter, 3)} \par
\noindent 
{\fontsize{10pt}{10pt}\selectfont ~~~~  $  \#  $ Free lock to realease next thread} \par
\noindent 
{\fontsize{10pt}{10pt}\selectfont ~~~~ ThreadLock.release()} \par
\noindent 
{\fontsize{10pt}{10pt}\selectfont ~ } \par
\noindent 
{\fontsize{10pt}{10pt}\selectfont ~ Def print $  \_  $time(threadName, delay, counter):} \par
\noindent 
{\fontsize{10pt}{10pt}\selectfont ~~ while counter:} \par
\noindent 
{\fontsize{10pt}{10pt}\selectfont ~~~ time.sleep(delay)} \par
\noindent 
{\fontsize{10pt}{10pt}\selectfont ~~~ print  $ " $ $  \%  $s:  $  \%  $s $ " $  $  \%  $ (threadName, time.ctime(time.time()))} \par
\noindent 
{\fontsize{10pt}{10pt}\selectfont ~~~ counter -= 1} \par
\noindent 
{\fontsize{10pt}{10pt}\selectfont ~ threadLock = threading.Lock()} \par
\noindent 
{\fontsize{10pt}{10pt}\selectfont ~ threads = []} \par
\vspace{10pt}
\noindent 
{\fontsize{10pt}{10pt}\selectfont  $  \#  $ Create new threads} \par
\noindent 
{\fontsize{10pt}{10pt}\selectfont thread1 = myThread(1,  $ " $Thread-1,1 )} \par
\noindent 
{\fontsize{10pt}{10pt}\selectfont thread2 = myThread(2,  $ " $Thread-2,2 )} \par
\vspace{10pt}
\noindent 
{\fontsize{10pt}{10pt}\selectfont  $  \#  $ Start new Threads} \par
\noindent 
{\fontsize{10pt}{10pt}\selectfont thread1.start()} \par
\noindent 
{\fontsize{10pt}{10pt}\selectfont thread2.start()} \par
\vspace{10pt}
\noindent 
{\fontsize{10pt}{10pt}\selectfont  $  \#  $ Add threads to thread list} \par
\noindent 
{\fontsize{10pt}{10pt}\selectfont threads.append(thread1)} \par
\noindent 
{\fontsize{10pt}{10pt}\selectfont thread2.append(thread2)} \par
\vspace{10pt}
\noindent 
{\fontsize{10pt}{10pt}\selectfont  $  \#  $ Wait for all threads to complete} \par
\noindent 
{\fontsize{10pt}{10pt}\selectfont Fort t in threads:} \par
\noindent 
{\fontsize{10pt}{10pt}\selectfont ~~~~ t.join()} \par
\noindent 
{\fontsize{10pt}{10pt}\selectfont print  $ " $Exiting Main thread $ " $} \par
\vspace{10pt}
\noindent 
Bila kode diatas dieksekusi, maka menghasilkan sebagai berikut : \par
\vspace{10pt}
\noindent 
{\fontsize{10pt}{10pt}\selectfont Starting Thread-1} \par
\noindent 
{\fontsize{10pt}{10pt}\selectfont Starting Thread-2} \par
\noindent 
{\fontsize{10pt}{10pt}\selectfont Thread-1: Thu Mar 21 09:11:28 2013} \par
\noindent 
{\fontsize{10pt}{10pt}\selectfont Thread-1: Thu Mar 21 09:11:29 2013} \par
\noindent 
{\fontsize{10pt}{10pt}\selectfont Thread-1: Thu Mar 21 09:11:30 2013} \par
\noindent 
{\fontsize{10pt}{10pt}\selectfont Thread-2: Thu Mar 21 09:11:32 2013} \par
\noindent 
{\fontsize{10pt}{10pt}\selectfont Thread-2: Thu Mar 21 09:11:34 2013} \par
\noindent 
{\fontsize{10pt}{10pt}\selectfont Thread-2: Thu Mar 21 09:11:36 2013} \par
\noindent 
{\fontsize{10pt}{10pt}\selectfont Exiting Main Thread} \par
\vspace{12pt}
\section{Multithreaded Antrian Prioritas}
\\}\end{center} \par
The queue modul memungkinkan untuk membuat objek antrian baru yang dapat menampung jumlah tertentu item. Ada metode berikut untuk mengontrol antrian : \par
\begin{itemize}
	\item \textbf{get()} \par
	\begin{adjustwidth}
		Menghapus dan mengembalikan item dari antrian\end{adjustwidth}
	\par
	\item \textbf{put()} \par
	\begin{adjustwidth}
		Menambahkan item ke antrian\end{adjustwidth}
	\par
	\item \textbf{qsize()} \par
	\begin{adjustwidth}
		Mengembalikan jumlah item yang saat ini dalam antrian\end{adjustwidth}
	\par
	\item \textbf{empty()} \par
	\begin{adjustwidth}
		Mengembalikan benar jika antrian kosong jika tidak, salah\end{adjustwidth}
	\par
	\item \textbf{full()}\end{itemize}
\par
\begin{adjustwidth}
	Mengembalikan benar jika antrian penuh jika tidak, salah\end{adjustwidth}
\par
Contoh: \par
\noindent 
{\fontsize{10pt}{10pt}\selectfont  $  \#  $!/usr/bin/python} \par
\vspace{10pt}
\noindent 
{\fontsize{10pt}{10pt}\selectfont import Queue} \par
\noindent 
{\fontsize{10pt}{10pt}\selectfont import threading} \par
\noindent 
{\fontsize{10pt}{10pt}\selectfont import time} \par
\vspace{10pt}
\noindent 
{\fontsize{10pt}{10pt}\selectfont exitFlag = 0} \par
\vspace{10pt}
\noindent 
{\fontsize{10pt}{10pt}\selectfont class myThread (threading.Thread):} \par
\noindent 
{\fontsize{10pt}{10pt}\selectfont ~~def   $  \_  $init $  \_  $(self, threadID, name, q):} \par
\noindent 
{\fontsize{10pt}{10pt}\selectfont ~~~~ threading.Thread. $  \_  $init $  \_  $(self)} \par
\noindent 
{\fontsize{10pt}{10pt}\selectfont ~~~ self.name = name} \par
\noindent 
{\fontsize{10pt}{10pt}\selectfont ~~~ self.q = q} \par
\noindent 
{\fontsize{10pt}{10pt}\selectfont  def run(self):} \par
\noindent 
{\fontsize{10pt}{10pt}\selectfont ~~~~~ print  $ " $Starting  $ " $+ self.name} \par
\noindent 
{\fontsize{10pt}{10pt}\selectfont ~~~~~ process $  \_  $data(self.name, self.q)} \par
\noindent 
{\fontsize{10pt}{10pt}\selectfont ~~~~~ print  $ " $Exiting  $ " $+ self.name} \par
\vspace{10pt}
\noindent 
{\fontsize{10pt}{10pt}\selectfont def process $  \_  $data(threadName, q):} \par
\noindent 
{\fontsize{10pt}{10pt}\selectfont ~~~ while not exitFlag:} \par
\noindent 
{\fontsize{10pt}{10pt}\selectfont ~~~ queuLock.acquire()} \par
\noindent 
{\fontsize{10pt}{10pt}\selectfont ~~~ if not workQueu.empty():} \par
\noindent 
{\fontsize{10pt}{10pt}\selectfont ~~~~~~~ data = q.get()} \par
\noindent 
{\fontsize{10pt}{10pt}\selectfont ~~~~~~~ queueLock.release()} \par
\noindent 
{\fontsize{10pt}{10pt}\selectfont ~~~~~~~ print  $ " $ $  \%  $s processing  $  \%  $s $ " $  $  \%  $ (threadName, data)} \par
\noindent 
{\fontsize{10pt}{10pt}\selectfont ~~~~ else:} \par
\noindent 
{\fontsize{10pt}{10pt}\selectfont ~~~~~~~ queueLock.release()} \par
\noindent 
{\fontsize{10pt}{10pt}\selectfont ~~~~~~~ time.sleep(1)} \par
\vspace{12pt}
\noindent 
{\fontsize{10pt}{10pt}\selectfont threadList = [ $ " $Thread-1 $ " $,  $ " $Thread-2 $ " $,  $ " $Thread-3 $ " $]} \par
\noindent 
{\fontsize{10pt}{10pt}\selectfont nameList = [ $ " $One $ " $,  $ " $Two $ " $,  $ " $Three $ " $,  $ " $Four $ " $,  $ " $Five $ " $]} \par
\noindent 
{\fontsize{10pt}{10pt}\selectfont queueLock = threading.Lock()} \par
\noindent 
{\fontsize{10pt}{10pt}\selectfont workLock = Queue.Queue(10)} \par
\noindent 
{\fontsize{10pt}{10pt}\selectfont threads = []} \par
\noindent 
{\fontsize{10pt}{10pt}\selectfont threadID = 1} \par
\vspace{10pt}
\noindent 
{\fontsize{10pt}{10pt}\selectfont  $  \#  $ Create new threads} \par
\noindent 
{\fontsize{10pt}{10pt}\selectfont For tName in threadList:} \par
\noindent 
{\fontsize{10pt}{10pt}\selectfont ~~~ thread = myThread(threadID, tName, workQueue)} \par
\noindent 
{\fontsize{10pt}{10pt}\selectfont ~~~ thread.start()} \par
\noindent 
{\fontsize{10pt}{10pt}\selectfont ~~~ thread.append(thread)} \par
\noindent 
{\fontsize{10pt}{10pt}\selectfont ~~~ threadID +=1} \par
\vspace{10pt}
\noindent 
{\fontsize{10pt}{10pt}\selectfont  $  \#  $ Fill the queue} \par
\noindent 
{\fontsize{10pt}{10pt}\selectfont queueLock.acquire()} \par
\noindent 
{\fontsize{10pt}{10pt}\selectfont for word in nameList:} \par
\noindent 
{\fontsize{10pt}{10pt}\selectfont ~~~ workQueue.put(word)} \par
\noindent 
{\fontsize{10pt}{10pt}\selectfont queueLock.release()} \par
\vspace{10pt}
\noindent 
{\fontsize{10pt}{10pt}\selectfont  $  \#  $ Wait for queue to empty} \par
\noindent 
{\fontsize{10pt}{10pt}\selectfont while not workQueue.empty():} \par
\noindent 
{\fontsize{10pt}{10pt}\selectfont pass} \par
\vspace{10pt}
\noindent 
{\fontsize{10pt}{10pt}\selectfont  $  \#  $ Notify threads it’s time to exit} \par
\noindent 
{\fontsize{10pt}{10pt}\selectfont exitFlag = 1} \par
\vspace{10pt}
\noindent 
{\fontsize{10pt}{10pt}\selectfont  $  \#  $ Wait for all threads to complete} \par
\noindent 
{\fontsize{10pt}{10pt}\selectfont For t in threads:} \par
\noindent 
{\fontsize{10pt}{10pt}\selectfont ~~~ t.join()} \par
\noindent 
{\fontsize{10pt}{10pt}\selectfont print  $ " $Exiting Main Thread $ " $} \par
\vspace{10pt}
\noindent 
Bila kode diatas dieksekusi, maka menghasilkan hasil sebagai berikut: \par
\vspace{12pt}
\noindent 
{\fontsize{10pt}{10pt}\selectfont Starting Thread-1} \par
\noindent 
{\fontsize{10pt}{10pt}\selectfont Starting Thread-2} \par
\noindent 
{\fontsize{10pt}{10pt}\selectfont Starting Thread-3} \par
\noindent 
{\fontsize{10pt}{10pt}\selectfont Thread-1 processing One} \par
\noindent 
{\fontsize{10pt}{10pt}\selectfont Thread-2 processing Two} \par
\noindent 
{\fontsize{10pt}{10pt}\selectfont Thread-3 processing Three} \par
\noindent 
{\fontsize{10pt}{10pt}\selectfont Thread-1 processing Four} \par
\noindent 
{\fontsize{10pt}{10pt}\selectfont Thread-2 processing Five} \par
\noindent 
{\fontsize{10pt}{10pt}\selectfont Exiting Thread-3} \par
\noindent 
{\fontsize{10pt}{10pt}\selectfont Exiting Thread-1} \par
\noindent 
{\fontsize{10pt}{10pt}\selectfont Exiting Thread-2} \par
\noindent 
{\fontsize{10pt}{10pt}\selectfont Exiting Main Thread} \par
\noindent 
\vspace{40pt}

\chapter{XML Processing}
XML adalah bahasa open source portable yang memungkinkan pemrogram mengemangkan aplikasi yang dapat dibaca oleh aplikasi lain, terlepas dari sistem operasi dan bahasa pengembangnya. \par
\vspace{12pt}
\noindent 
Apa itu XML? \par
Extensible Markup Languange (XML) adalah bahasa markup seperti HTML atau SGML. Ini direkomendasikan oleh World Wide Web Consortium dan tersedia sebagai standar terbuka.  \par
XML sangat berguna untuk mencatat data berukuran kecil dan menengah tanpa memerlukan tulang punggung berbasis SQL. \par
\vspace{12pt}
\noindent 

\section{Arsitektur Parsing XML dan API}
 \par
\noindent 
\hspace*{0.5in} Perpustakaan standar Python menyediakan seperangkat antarmuka minimal tapi berguna untuk bekerja dengan XML.  \par
\noindent 
\hspace*{0.5in} Dua API yang paling dasar dan umum digunakan untuk data XML adalah antarmuka SAX dan DOM. \par
\noindent 
\hspace*{0.5in} API sederhana untuk XML (SAX): mendaftarkan panggilan kemali untuk acara yang diminati dan kemudian membiarkan parser berjalan melalui dokumen. Ini berguna bila dokumen berukuran besar atau memiliki keterbatasan memori, ini memparsing file tidak pernah tersimpan dalam memori. \par
\noindent 
\hspace*{0.5in} API Document Objek Model (DOM): ini adalah rekomendasi World Wide Web Consortium dimana keseluruhan file dibaca ke memori dan disimpan dalam bentuk hierarkies (tree-based) untuk mewakili semua fitur dokumen XML.  \par
\noindent 
\hspace*{0.5in} SAX jelas tidak bisa memproses informasi secepat DOM saat bisa bekerjadengan file besar. Di sisi lain, menggunakan DOM secara eklusifenar-benar dapat membunuh sumber daya, terutama jika digunakan pada banyak file kecil. \par
\noindent 
~~~~~~~~~~~ SAX hanya bisa dibaca sementara DOM mengizinkan perubahan pada file XML. Kedua API yang berbeda ini saling melengkapi satu sama lain, tidak ada alasan mengapa tidak dapat menggunakannya untuk proyek besar. \par
\noindent 


\vspace{12pt}
\vspace{12pt}
\noindent 
Contoh: \par
\noindent 
<collection shelf="New Arrivals"> \par
\noindent 
<movie title="Enemy Behind"> \par
\noindent 
~~ <type>War, Thriller</type> \par
\noindent 
~~ <format>DVD</format> \par
\noindent 
~~ <year>2003</year> \par
\noindent 
~~ <rating>PG</rating> \par
\noindent 
~~ <stars>10</stars> \par
\noindent 
~~ <description>Talk about a US-Japan war</description> \par
\noindent 
</movie> \par
\noindent 
<movie title="Transformers"> \par
\noindent 
~~ <type>Anime, Science Fiction</type> \par
\noindent 
~~ <format>DVD</format> \par
\noindent 
~~ <year>1989</year> \par
\noindent 
~~ <rating>R</rating> \par
\noindent 
~~ <stars>8</stars> \par
\noindent 
~~ <description>A schientific fiction</description> \par
\noindent 
</movie> \par
\noindent 
~~ <movie title="Trigun"> \par
\noindent 
~~ <type>Anime, Action</type> \par
\noindent 
~~ <format>DVD</format> \par
\noindent 
~~ <episodes>4</episodes> \par
\noindent 
~~ <rating>PG</rating> \par
\noindent 
~~ <stars>10</stars> \par
\noindent 
~~ <description>Vash the Stampede!</description> \par
\noindent 
</movie> \par
\noindent 
<movie title="Ishtar"> \par
\noindent 
~~ <type>Comedy</type> \par
\noindent 
~~ <format>VHS</format> \par
\noindent 
~~ <rating>PG</rating> \par
\noindent 
~~ <stars>2</stars> \par
\noindent 
~~ <description>Viewable boredom</description> \par
\noindent 
</movie> \par
\noindent 
</collection> \par
\vspace{10pt}
\noindent

\section{Parsing XML dan API SAX}
\par
\noindent 
\hspace*{0.5in} SAX adalah antarmuka standar untuk parsing XML berbasis event. Parsing XML dengan SAX umumnya mengharuskan untuk membuat C\textit{ontrolHandler }dengan subclassing xml.sax \textit{controlhandler}. \par
\noindent 
\hspace*{0.5in} \textit{ControlHandler }menangani tag dan atribut tertentu dari XML. Objek \textit{ControlHandler }menyediakan metode untuk menangani berbagai aktivitas parsing. Parsing memanggil metode \textit{ControlHandler }saat memparsing file XML. \par
\noindent 
\hspace*{0.5in} Metode \textit{startDocument} dan \textit{endDocument} disebut awal dan akhir setiap elemen. Jika parsing tidak dalam mode namespace, metode \textit{startElement} (tag attribute) dan \textit{endElement} (tag) dipanggil. Jika tidak, metode yang sesuai \textit{startElemenNS} dan \textit{endElemenNS} dipanggil. Disini, tah adalah tag elemen dan atriut adalah atribut.  \par
\noindent 
\hspace*{0.5in} Berikut ini metode penting untuk memahami sebelum melanjutkan ke materi berikutnya : \par
\noindent 
\begin{myEnumerate}
	\item Metode \textit{make $  \_  $parser} \par
	Metode berikut membuat objek parsing baru dan mengembalikannya. Objek parsing diuat akan menjadi tipe parsing pertama yang ditemukan sistem.  \par
	\vspace{10pt}
	{\fontsize{10pt}{10pt}\selectfont xml.sax.make $  \_  $parser([parser $  \_  $list])} \par
	\vspace{12pt}
	Berikut adalah detail parameternya : \par
	Parser $  \_  $list : pilihan argumen yang terdiri dari daftar parsing untuk digunakan yang semuanya harus menerapkan metode \textit{make $  \_  $parse} \par
	\noindent 
	\item Metode \textit{parser} \par
	Metode berikut membuat parsing SAX dan menggunakannya untuk mengurai dokumen \par
	\vspace{10pt}
	{\fontsize{10pt}{10pt}\selectfont xml.sax.parser(xmlfile, contenthandler[, errorhandler])} \par
	\vspace{10pt}
	Berikut adalah detail dari parameternya: \par
	\noindent 
	\begin{itemize}
		\item \textit{Xmlfile } \par
		Ini adalah nama file XML yang bisa dibaca. \par
		\noindent 
		\item \textit{ContentHandler } \par
		Ini harus menjadi objek \textit{ContenHandler} \par
		\noindent 
		\item \textit{ErrorHandler} \par
		Jika ditentukan, e\textit{rrorhandler} harus menjadi objek \textit{ErrorHandler} SAX \par
		\noindent 
		\item Metode\textit{ parseString}\end{itemize}
	\par
	Membuat parsing SAX dan mengurai string XML yang ditentukan : \par
	\vspace{12pt}
	{\fontsize{10pt}{10pt}\selectfont xml.sax.parsertring(xmlstring,contenthandler[, errorhandler])} \par
	\vspace{12pt}
	Brikut ini adalah detail nama dar parameter : \par
	\noindent 
	\item \textit{XMLstring} \par
	Nama dari string yang bisa dibaca \par
	\noindent 
	\item \textit{ContentHandler} \par
	Menjadi objek ContenHandler \par
	\noindent 
	\item \textit{ErrorHandler} \par
	Menjadi objek ErorHandler SAX \par
	\vspace{12pt}
	\noindent 
	Contoh : \par
	\noindent 
	$  \#  $!/usr/bin/python \par
	\vspace{12pt}
	\noindent 
	import xml.sax \par
	\vspace{12pt}
	\noindent 
	class MovieHandler( xml.sax.ContentHandler ): \par
	\noindent 
	~~ def  $  \_  $ $  \_  $init $  \_  $ $  \_  $(self): \par
	\noindent 
	~~~~~ self.CurrentData = "" \par
	\noindent 
	~~~~~ self.type = "" \par
	\noindent 
	~~~~~ self.format = "" \par
	\noindent 
	~~~~~ self.year = "" \par
	\noindent 
	~~~~~ self.rating = "" \par
	\noindent 
	~~~~~ self.stars = "" \par
	\noindent 
	~~~~~ self.description = "" \par
	\vspace{12pt}
	\noindent 
	~~  $  \#  $ Call when an element starts \par
	\noindent 
	~~ def startElement(self, tag, attributes): \par
	\noindent 
	~~~~~ self.CurrentData = tag \par
	\noindent 
	~~~~~ if tag == "movie": \par
	\noindent 
	~~~~~~~~ print "*****Movie*****" \par
	\noindent 
	~~~~~~~~ title = attributes["title"] \par
	\noindent 
	~~~~~~~~ print "Title:", title \par
	\vspace{12pt}
	\noindent 
	~~  $  \#  $ Call when an elements ends \par
	\noindent 
	~~ def endElement(self, tag): \par
	\noindent 
	~~~~~ if self.CurrentData == "type": \par
	\noindent 
	~~~~~~~~ print "Type:", self.type \par
	\noindent 
	~~~~~ elif self.CurrentData == "format": \par
	\noindent 
	~~~~~~~~ print "Format:", self.format \par
	\noindent 
	~~~~~ elif self.CurrentData == "year": \par
	\noindent 
	~~~~~~~~ print "Year:", self.year \par
	\noindent 
	~~~~~ elif self.CurrentData == "rating": \par
	\noindent 
	~~~~~~~~ print "Rating:", self.rating \par
	\noindent 
	~~~~~ elif self.CurrentData == "stars": \par
	\noindent 
	~~~~~~~~ print "Stars:", self.stars \par
	\noindent 
	~~~~~ elif self.CurrentData == "description": \par
	\noindent 
	~~~~~~~~ print "Description:", self.description \par
	\noindent 
	~~~~~ self.CurrentData = "" \par
	\vspace{12pt}
	\noindent 
	~~  $  \#  $ Call when a character is read \par
	\noindent 
	~~ def characters(self, content): \par
	\noindent 
	~~~~~ if self.CurrentData == "type": \par
	\noindent 
	~~~~~~~~ self.type = content \par
	\noindent 
	~~~~~ elif self.CurrentData == "format": \par
	\noindent 
	~~~~~~~~ self.format = content \par
	\noindent 
	~~~~~ elif self.CurrentData == "year": \par
	\noindent 
	~~~~~~~~ self.year = content \par
	\noindent 
	~~~~~ elif self.CurrentData == "rating": \par
	\noindent 
	~~~~~~~~ self.rating = content \par
	\noindent 
	~~~~~ elif self.CurrentData == "stars": \par
	\noindent 
	~~~~~~~~ self.stars = content \par
	\noindent 
	~~~~~ elif self.CurrentData == "description": \par
	\noindent 
	~~~~~~~~ self.description = content \par
	\noindent 
	~  \par
	\noindent 
	if (  $  \_  $ $  \_  $name $  \_  $ $  \_  $ == " $  \_  $ $  \_  $main $  \_  $ $  \_  $"): \par
	\noindent 
	~~  \par
	\noindent 
	~~  $  \#  $ create an XMLReader \par
	\noindent 
	~~ parser = xml.sax.make $  \_  $parser() \par
	\noindent 
	~~  $  \#  $ turn off namepsaces \par
	\noindent 
	~~ parser.setFeature(xml.sax.handler.feature $  \_  $namespaces, 0) \par
	\vspace{12pt}
	\noindent 
	~~  $  \#  $ override the default ContextHandler \par
	\noindent 
	~~ Handler = MovieHandler() \par
	\noindent 
	~~ parser.setContentHandler( Handler ) \par
	\noindent 
	~~  \par
	\noindent 
	~~ parser.parse("movies.xml") \par
	\vspace{12pt}
	\vspace{12pt}
	\noindent 
	Ini akan menghasilkan hasil sebagai berikut: \par
	\noindent 
	{\fontsize{10pt}{10pt}\selectfont *****Movie******} \par
	\noindent 
	{\fontsize{10pt}{10pt}\selectfont *****Movie*****} \par
	\noindent 
	{\fontsize{10pt}{10pt}\selectfont Title: Enemy Behind} \par
	\noindent 
	{\fontsize{10pt}{10pt}\selectfont Type: War, Thriller} \par
	\noindent 
	{\fontsize{10pt}{10pt}\selectfont Format: DVD} \par
	\noindent 
	{\fontsize{10pt}{10pt}\selectfont Year: 2003} \par
	\noindent 
	{\fontsize{10pt}{10pt}\selectfont Rating: PG} \par
	\noindent 
	{\fontsize{10pt}{10pt}\selectfont Stars: 10} \par
	\noindent 
	{\fontsize{10pt}{10pt}\selectfont Description: Talk about a US-Japan war} \par
	\noindent 
	{\fontsize{10pt}{10pt}\selectfont *****Movie*****} \par
	\noindent 
	{\fontsize{10pt}{10pt}\selectfont Title: Transformers} \par
	\noindent 
	{\fontsize{10pt}{10pt}\selectfont Type: Anime, Science Fiction} \par
	\noindent 
	{\fontsize{10pt}{10pt}\selectfont Format: DVD} \par
	\noindent 
	{\fontsize{10pt}{10pt}\selectfont Year: 1989} \par
	\noindent 
	{\fontsize{10pt}{10pt}\selectfont Rating: R} \par
	\noindent 
	{\fontsize{10pt}{10pt}\selectfont Stars: 8} \par
	\noindent 
	{\fontsize{10pt}{10pt}\selectfont Description: A schientific fiction} \par
	\noindent 
	{\fontsize{10pt}{10pt}\selectfont *****Movie*****} \par
	\noindent 
	{\fontsize{10pt}{10pt}\selectfont Title: Trigun} \par
	\noindent 
	{\fontsize{10pt}{10pt}\selectfont Type: Anime, Action} \par
	\noindent 
	{\fontsize{10pt}{10pt}\selectfont Format: DVD} \par
	\noindent 
	{\fontsize{10pt}{10pt}\selectfont Rating: PG} \par
	\noindent 
	{\fontsize{10pt}{10pt}\selectfont Stars: 10} \par
	\noindent 
	{\fontsize{10pt}{10pt}\selectfont Description: Vash the Stampede!} \par
	\noindent 
	{\fontsize{10pt}{10pt}\selectfont *****Movie*****} \par
	\noindent 
	{\fontsize{10pt}{10pt}\selectfont Title: Ishtar} \par
	\noindent 
	{\fontsize{10pt}{10pt}\selectfont Type: Comedy} \par
	\noindent 
	{\fontsize{10pt}{10pt}\selectfont Format: VHS} \par
	\noindent 
	{\fontsize{10pt}{10pt}\selectfont Rating: PG} \par
	\noindent 
	{\fontsize{10pt}{10pt}\selectfont Stars: 2} \par
	\noindent 
	{\fontsize{10pt}{10pt}\selectfont Description: Viewable boredom} \par
	\noindent 
	{\fontsize{10pt}{10pt}\selectfont 
			
\section{Parsing XML dan API DOM}
\par
\noindent 
\hspace*{0.5in} Document Ovject Model (DOM) adalah API lintas bahasa dari World Wide Web Consortium (W3C) untuk mengakses dan memodifikasi dokumen XML. \par
\noindent 
\hspace*{0.5in} DOM sangat berguna untuk aplikasi akses acak. SAX hanya memungkinkan melihat satu bit dokumen sekaligus. Jika melihat satu elemen SAX, tidak memiliki akses ke yang lain. \par
\noindent 
\hspace*{0.5in} Berikut adalah cara termudah untuk memuat dokumen XML dengan cepat dan membuat objek minidom menggunakan modul xml.dom. Objek minidom menyediakan metode parsing sederhana yang dengan cepat memuat pohon DOM dari file XML. \par
\noindent 
\hspace*{0.5in} Contoh~frase memanggil fungsi  parsing (file [,parsing]) dari objek minidokumen untuk mengurai file XML yang ditunjuk oleh file ke objek pohon DOM. \par
\noindent 
$  \#  $!/usr/bin/python \par
\vspace{12pt}
\noindent 
from xml.dom.minidom import parse \par
\noindent 
import xml.dom.minidom \par
\vspace{12pt}
\noindent 
$  \#  $ Open XML document using minidom parser \par
\noindent 
DOMTree = xml.dom.minidom.parse("movies.xml") \par
\noindent 
collection = DOMTree.documentElement \par
\noindent 
if collection.hasAttribute("shelf"): \par
\noindent 
~~ print "Root element :  $  \%  $s"  $  \%  $ collection.getAttribute("shelf") \par
\vspace{12pt}
\noindent 
$  \#  $ Get all the movies in the collection \par
\noindent 
movies = collection.getElementsByTagName("movie") \par
\vspace{12pt}
\noindent 
$  \#  $ Print detail of each movie. \par
\noindent 
for movie in movies: \par
\noindent 
~~ print "*****Movie*****" \par
\noindent 
~~ if movie.hasAttribute("title"): \par
\noindent 
~~~~~ print "Title:  $  \%  $s"  $  \%  $ movie.getAttribute("title") \par
\vspace{12pt}
\noindent 
~~ type = movie.getElementsByTagName('type')[0] \par
\noindent 
~~ print "Type:  $  \%  $s"  $  \%  $ type.childNodes[0].data \par
\noindent 
~~ format = movie.getElementsByTagName('format')[0] \par
\noindent 
~~ print "Format:  $  \%  $s"  $  \%  $ format.childNodes[0].data \par
\noindent 
~~ rating = movie.getElementsByTagName('rating')[0] \par
\noindent 
~~ print "Rating:  $  \%  $s"  $  \%  $ rating.childNodes[0].data \par
\noindent 
~~ description = movie.getElementsByTagName('description')[0] \par
\noindent 
~~ print "Description:  $  \%  $s"  $  \%  $ description.childNodes[0].data \par
\vspace{12pt}
\noindent 
Ini akan menghasilkan hasil sebagai berikut : \par
\noindent 
Root element : New Arrivals \par
\noindent 
*****Movie***** \par
\noindent 
Title: Enemy Behind \par
\noindent 
Type: War, Thriller \par
\noindent 
Format: DVD \par
\noindent 
Rating: PG \par
\noindent 
Description: Talk about a US-Japan war \par
\noindent 
*****Movie***** \par
\noindent 
Title: Transformers \par
\noindent 
Type: Anime, Science Fiction \par
\noindent 
Format: DVD \par
\noindent 
Rating: R \par
\noindent 
Description: A schientific fiction \par
\noindent 
*****Movie***** \par
\noindent 
Title: Trigun \par
\noindent 
Type: Anime, Action \par
\noindent 
Format: DVD \par
\noindent 
Rating: PG \par
\noindent 
Description: Vash the Stampede! \par
\noindent 
*****Movie***** \par
\noindent 
Title: Ishtar \par
\noindent 
Type: Comedy \par
\noindent 
Format: VHS \par
\noindent 
Rating: PG \par
\noindent 
Description: Viewable boredom \par
\vspace{12pt}
\noindent

\section{Membangun Parsing Document XML menggunakan Python}
\par
\noindent 
\hspace*{0.5in} Python mendukung untuk bekerja dengan berbagai bentuk markup data terstruktur. Selain mengurai xml.etree. \textit{ElementTree} mendukung pembuatan dokumen XML yang terbentuk dengan baik dari objek elemen yang dibangun dalam aplikasi. Kelas elemen digunakakan saat sebuah dokumen diurai untuk mengetahui bagaimana menghasilkan bentuk serial dari isinya kemudian dapat ditulis ke sebuah file.  \par
\vspace{12pt}
\noindent 
\hspace*{0.5in} Untuk membuat instance elemeb gunakan fungsi elemen contructor dan \textit{SubElemen()} pabrik. \par
\noindent 
Import xml.etree.ElementTree as xml \par
\vspace{12pt}
\noindent 
{\fontsize{10pt}{10pt}\selectfont filename =  $ " $/home/abc/Desktop/test $  \_  $xml.xml $ " $} \par
\noindent 
{\fontsize{10pt}{10pt}\selectfont toot = xml.Element( $ " $Users $ " $)} \par
\noindent 
{\fontsize{10pt}{10pt}\selectfont userelement = xml.Element( $ " $user $ " $)} \par
\noindent 
{\fontsize{10pt}{10pt}\selectfont root.append(userelement)} \par
\noindent 
\vspace{10pt}
\noindent 
Bila menjalankan ini, akan menghasilkan sebagai berikut : \par
\noindent 
{\fontsize{10pt}{10pt}\selectfont <Users>} \par
\noindent 
{\fontsize{10pt}{10pt}\selectfont  \hspace*{0.5in} <user>} \par
\noindent 
{\fontsize{10pt}{10pt}\selectfont  \hspace*{0.5in} <user>} \par
\noindent 
{\fontsize{10pt}{10pt}\selectfont </Users>} \par
\vspace{10pt}
\vspace{10pt}
\vspace{10pt}
\noindent 
Tambahkan anak-anak pegguna \par
\vspace{10pt}
\noindent 
{\fontsize{10pt}{10pt}\selectfont Uid = xml.SubElement(userelement,  $ " $uid $ " $)} \par
\noindent 
{\fontsize{10pt}{10pt}\selectfont Uid.text =  $ " $1 $ " $} \par
\vspace{10pt}
\noindent 
{\fontsize{10pt}{10pt}\selectfont FirstName = xml.SubElement(userelement,  $ " $FirstName $ " $)} \par
\noindent 
{\fontsize{10pt}{10pt}\selectfont FirstName.text =  $ " $testuser $ " $} \par
\vspace{10pt}
\noindent 
{\fontsize{10pt}{10pt}\selectfont LastName = xml.SubElement(userelement,  $ " $LastName $ " $} \par
\noindent 
{\fontsize{10pt}{10pt}\selectfont LastName.text =  $ " $testuser $ " $} \par
\vspace{10pt}
\noindent 
{\fontsize{10pt}{10pt}\selectfont Email = xml.SubElement(userelement,  $ " $Email $ " $)} \par
\noindent 
{\fontsize{10pt}{10pt}\selectfont Email.text = \href{mailto:testuser@test.com}{testuser@test.com}
} \par
\vspace{10pt}
\noindent 
{\fontsize{10pt}{10pt}\selectfont state = xml.SubElement(userelemet,  $ " $state $ " $)} \par
\noindent 
{\fontsize{10pt}{10pt}\selectfont state.text =  $ " $xyz $ " $} \par
\vspace{10pt}
\noindent 
{\fontsize{10pt}{10pt}\selectfont location = xml.SubElement(userelement,  $ " $location)} \par
\noindent 
{\fontsize{10pt}{10pt}\selectfont location.text = abc} \par
\vspace{10pt}
\noindent 
{\fontsize{10pt}{10pt}\selectfont tree = xml.ElementTree(root)} \par
\noindent 
{\fontsize{10pt}{10pt}\selectfont with open(filename,  $ " $w $ " $) as fh:} \par
\noindent 
{\fontsize{10pt}{10pt}\selectfont tree.write(fh)} \par
\vspace{10pt}
\noindent 
\hspace*{0.5in} Pertama buat elemen root dengan mengunakan fungsi \textit{ElementTree}. Kemudian membuat elemen pegguna dan menambahkannya ke root. Selanjutnya membuat \textit{SubElement }dengan melewatkan elemen pengguna (userelement) ke \textit{SubElemen} beserta namanya seperto  $ " $FirstName $ " $. Kemudian untuk setiap \textit{SubElement} tetapkan properti teks untuk memberi nilai. Di akhir, membuat \textit{ElementTree} dan menggunakannya untuk menulis XML ke file. \par
\noindent 
\hspace*{0.5in} Jika menjalankan ini akan menjadi sebagai berikut : \par
\noindent 
{\fontsize{10pt}{10pt}\selectfont <users>} \par
\noindent 
{\fontsize{10pt}{10pt}\selectfont  \hspace*{0.5in} <user>} \par
\noindent 
{\fontsize{10pt}{10pt}\selectfont  \hspace*{0.5in}  \hspace*{0.5in} <uid>1</uid>} \par
\noindent 
{\fontsize{10pt}{10pt}\selectfont  \hspace*{0.5in}  \hspace*{0.5in} <FirstName>testuser</FirstName>} \par
\noindent 
{\fontsize{10pt}{10pt}\selectfont  \hspace*{0.5in}  \hspace*{0.5in} <LastName>testuser</LastName>} \par
\noindent 
{\fontsize{10pt}{10pt}\selectfont  \hspace*{0.5in}  \hspace*{0.5in} <Email>\href{mailto:testuser@test.com $  \%  $3c/Email}{testuser@test.com</Email}
	>} \par
\noindent 
{\fontsize{10pt}{10pt}\selectfont  \hspace*{0.5in}  \hspace*{0.5in} <state>xyz</state>} \par
\noindent 
{\fontsize{10pt}{10pt}\selectfont  \hspace*{0.5in}  \hspace*{0.5in} <location>abc</location>} \par
\noindent 
{\fontsize{10pt}{10pt}\selectfont  \hspace*{0.5in} </user>} \par
\noindent 
{\fontsize{10pt}{10pt}\selectfont </Users>} \par
\vspace{10pt}
\noindent 
Parsing XML Documen : \par
\vspace{12pt}
\noindent 
{\fontsize{10pt}{10pt}\selectfont import xml.etree.ElementTree as ET} \par
\noindent 
{\fontsize{10pt}{10pt}\selectfont tree = ET.parse(‘Your $  \_  $XML $  \_  $file $  \_  $path’)} \par
\noindent 
{\fontsize{10pt}{10pt}\selectfont root = tree.getroot()} \par
\noindent 
{\fontsize{10pt}{10pt}\selectfont 	
}\vspace{10pt}

Disini \textit{getroot()} akan mengembalikan elemen dari dokumen XML \par
\vspace{10pt}
\noindent 
{\fontsize{10pt}{10pt}\selectfont <Users version= $ " $1.0 $ " $ languange= $ " $SPA $ " $>} \par
\noindent 
{\fontsize{10pt}{10pt}\selectfont  \hspace*{0.5in} <user>} \par
\noindent 
{\fontsize{10pt}{10pt}\selectfont  \hspace*{0.5in}  \hspace*{0.5in} <uid>1</uid>} \par
\noindent 
{\fontsize{10pt}{10pt}\selectfont  \hspace*{0.5in}  \hspace*{0.5in} <FirstName>testuser</FirstName>} \par
\noindent 
{\fontsize{10pt}{10pt}\selectfont  \hspace*{0.5in}  \hspace*{0.5in} <LastName>testuser</LastName>} \par
\noindent 
{\fontsize{10pt}{10pt}\selectfont  \hspace*{0.5in}  \hspace*{0.5in} <Email>testuser@tes.com/Email>} \par
\noindent 
{\fontsize{10pt}{10pt}\selectfont  \hspace*{0.5in}  \hspace*{0.5in} <state>xyz</state>} \par
\noindent 
{\fontsize{10pt}{10pt}\selectfont  \hspace*{0.5in}  \hspace*{0.5in} <location>abc</location>} \par
\noindent 
{\fontsize{10pt}{10pt}\selectfont  \hspace*{0.5in} </user>} \par
\noindent 
{\fontsize{10pt}{10pt}\selectfont </Users>} \par
\vspace{10pt}

\chapter{GUI Programming}
\par
\vspace{12pt}
Python menyediakan berbagai pilihan untuk mengembangkan antarmuka pengguna grafis (GUIs). Yang paling tercantum dibawah ini : \par
\noindent 
\begin{itemize}
	\item Tkinter \par
	Antarmuka Python ke toolkit Tk GUI dikirimkan dengan Python.  \par
	\noindent 
	\item wxPython \par
	antarmuka Python open-source untuk wxWindows \par
	\noindent 
	\item Jpython \par
	Port Python untuk java yang memberikan Python script akses tanpa batas ke perpustakaan kelas java pada mesin lokal \par
	\vspace{12pt}
	\noindent 
	
\section{Tkinter Pemrograman}
\par
Tkinter adalah perpustakaan GUI standar untuk Python. Python bila dikombinasikan dengan Tkinter menyediakan cara yang mudah dan cepat untuk membuat aplikasi GUI. Tkinter menyediakan antarmuka berorientasi ojek yang kuat untuk toolkit Tk GUI. \par
\noindent 
\hspace*{0.5in} Membuat aplikasi GUI menggunakan Tkinter adalah tugas yang mudah. Yang diperlukan adalah melakukan langkah-langkah sebagai berikut : \par
\noindent 
\item Mengimpor Tkinter modul \par
\noindent 
\item Buat jendela utama aplikasi GUI \par
\noindent 
\item Tambahkan satu atau lebih dari widget tersebut diatas ke aplikasi GUI \par
\noindent 
\item Masukkan acara loop utama untuk mengambil tindakan terhadap setiap peristiwa dipicu oleh pengguna\end{itemize}
\par
\noindent 
\vspace{12pt}
\vspace{12pt}
\noindent 
Contoh : \par
\noindent 
{\fontsize{10pt}{10pt}\selectfont  $  \#  $!/usr/bin/python} \par
\vspace{10pt}
\noindent 
{\fontsize{10pt}{10pt}\selectfont import Tkinter} \par
\noindent 
{\fontsize{10pt}{10pt}\selectfont top = Tkinter.Tk()} \par
\noindent 
{\fontsize{10pt}{10pt}\selectfont   $  \#  $ Code to add widgets will go here...} \par
\noindent 
{\fontsize{10pt}{10pt}\selectfont top.mainloop()} \par
\vspace{10pt}
\noindent 

\section{Tkinter Widget}
\par
\noindent 
\hspace*{0.5in} Tkinter menyediakan berbagai kontrol seperti tombol, label dan kotak teks yang digunakan dalam aplikasi GUI. Kontrol ini biasanya disebut widget.  \par
\noindent 
\hspace*{0.5in} Saat ini ada 15 jenis widget di Tkinter. Menyajikan widget serta penjelasan singkat pada tabel berikut ini : \par


%%%%%%%%%%%%  Table No:1 Here %%%%%%%%%%%%%%


\begin{table}[H]
	\centering
	\begin{adjustbox}{width=\textwidth}
		\begin{tabular}{ p{1.34in}p{2.48in} }
			\hhline{--}
			\multicolumn{1}{|p{1.34in}}{\Centering \textbf{Operator}} & \multicolumn{1}{|p{4.69in}|}{\Centering \textbf{Penjelasan}} & \hhline{--}
			\multicolumn{1}{|p{1.34in}}{Button} & \multicolumn{1}{|p{4.69in}|}{Menampilkan tombol dalam aplikasi} & \hhline{--}
			\multicolumn{1}{|p{1.34in}}{Canvas} & \multicolumn{1}{|p{4.69in}|}{Menggambar bentuk seperti garis, oval, poligon dan persegi panjang dalam aplikasi} & \hhline{--}
			\multicolumn{1}{|p{1.34in}}{Checkbutton} & \multicolumn{1}{|p{4.69in}|}{Menampilkan sejumlah pilihan sebagai kotak centang. Pengguna dapat memilih beberapa pilihan pada suatu waktu} & \hhline{--}
			\multicolumn{1}{|p{1.34in}}{Entry} & \multicolumn{1}{|p{4.69in}|}{Menampilka bidang garis teks tunggal untuk menerima nilai-nilai dari pengguna} & \hhline{--}
			\multicolumn{1}{|p{1.34in}}{Frame} & \multicolumn{1}{|p{4.69in}|}{Wadah untuk mengatur widget lainnya} & \hhline{--}
			\multicolumn{1}{|p{1.34in}}{Label } & \multicolumn{1}{|p{4.69in}|}{Memberikan keterangan garis single untuk widget lainnya. Hal ini berisi gambar} & \hhline{--}
			\multicolumn{1}{|p{1.34in}}{Listbox } & \multicolumn{1}{|p{4.69in}|}{Menyediakan daftar pilihan kepada pengguna} & \hhline{--}
			\multicolumn{1}{|p{1.34in}}{Menubutton} & \multicolumn{1}{|p{4.69in}|}{Menampilkan menu dalam aplikasi} & \hhline{--}
			\multicolumn{1}{|p{1.34in}}{Menu} & \multicolumn{1}{|p{4.69in}|}{Memberikan berbagai perintah untuk pengguna. Perintah-perintah ini terkandung di dalam MenuButton} & \hhline{--}
			\multicolumn{1}{|p{1.34in}}{Message} & \multicolumn{1}{|p{4.69in}|}{Menampilkan bidang teks multiline untuk menerima nilai-nilai dari pengguna} & \hhline{--}
			\multicolumn{1}{|p{1.34in}}{RadioButton} & \multicolumn{1}{|p{4.69in}|}{Menampilkan sejumah pilihan sebagai tombol radio. Pengguna dapat memilih hanya satu pilihan pada suatu waktu} & \hhline{--}
			\multicolumn{1}{|p{1.34in}}{Scale} & \multicolumn{1}{|p{4.69in}|}{Menyediakan widget slide} & \hhline{--}
			\multicolumn{1}{|p{1.34in}}{Scrollbar } & \multicolumn{1}{|p{4.69in}|}{Menambah kemampuan bergulir ke berbagai widget seperti kotak daftar} & \hhline{--}
			\multicolumn{1}{|p{1.34in}}{Text} & \multicolumn{1}{|p{4.69in}|}{Menampilka teks dalam beberapa garis} & \hhline{--}
			\multicolumn{1}{|p{1.34in}}{Toplevel} & \multicolumn{1}{|p{4.69in}|}{Menyediakan wajah jendela terpisah} & \hhline{--}
			\multicolumn{1}{|p{1.34in}}{PanedWindow} & \multicolumn{1}{|p{4.69in}|}{Wadah yang mengandung sejumlah panel disusun horizontal atau vertikal} & \hhline{--}
			\multicolumn{1}{|p{1.34in}}{LabelFrame} & \multicolumn{1}{|p{4.69in}|}{Wadah widget sederhana. Bertindak sebagai spacer atau wajah untuk layout jendela kompleks} & \hhline{--}
			\multicolumn{1}{|p{1.34in}}{TkMessageBox} & \multicolumn{1}{|p{4.69in}|}{Menampilkan kotak pesan dalam aplikasi} & \hhline{--}
			\multicolumn{1}{|p{1.34in}}{Spinbox } & \multicolumn{1}{|p{4.69in}|}{Memilih sejumlah tetap nilai-nilai} & \hline
		\end{tabular}
	\end{adjustbox}
\end{table}


%%%%%%%%%%%%  Table No:1 Ends Here %%%%%%%%%%%%%%


\vspace{12pt}
\noindent 
\hspace*{0.5in} Beberapa atribut umu sebagai ukuran, warna dan font ditentukan. Berikut adalah beberapa atriut standarm : \par
\noindent 
\begin{myEnumerate}
	\item Ukuran \par
	\noindent 
	Berbagai panjang, lebar, dan dimensi lain dari widget digambarkan dalam banyak unit yang berbeda seperti : \par
	\noindent 
	\begin{itemize}
		\item Jika menetapkan dimensi ke integer diasumsikan dalam piksel \par
		\noindent 
		\item Menentukan unit dengan menentukan dimensi untuk string yang berisi sejumlah diikuti oleh :\end{itemize}
	\par
	
	
	%%%%%%%%%%%%  Table No:2 Here %%%%%%%%%%%%%%
	
	
	\begin{table}[H]
		\centering
		\begin{adjustbox}{width=\textwidth}
			\begin{tabular}{ p{0.89in}p{1.93in} }
				\hhline{--}
				\multicolumn{1}{|p{1.06in}}{\Centering \textbf{Karakter}} & \multicolumn{1}{|p{4.2in}|}{\Centering \textbf{Penjelasan}} & \hhline{--}
				\multicolumn{1}{|p{1.06in}}{c} & \multicolumn{1}{|p{4.2in}|}{Sentimeter} & \hhline{--}
				\multicolumn{1}{|p{1.06in}}{i} & \multicolumn{1}{|p{4.2in}|}{Inci} & \hhline{--}
				\multicolumn{1}{|p{1.06in}}{m} & \multicolumn{1}{|p{4.2in}|}{Milimeter} & \hhline{--}
				\multicolumn{1}{|p{1.06in}}{p} & \multicolumn{1}{|p{4.2in}|}{Poin printer (1/72 $ " $)} & \hline
			\end{tabular}
		\end{adjustbox}
	\end{table}
	
	
	%%%%%%%%%%%%  Table No:2 Ends Here %%%%%%%%%%%%%%
	
	
	\hspace*{0.5in} \vspace{12pt}
	\hspace*{0.5in} Tkinter mengungkapkan panjang sebagai integer jumlah piksel. Berikut ini adalah daftar pilihan panjang umum: \par
	\noindent 
	\begin{itemize}
		\item borderwidth \par
		Lebar batas yang memberikan tampilan tiga dimensi untuk widget \par
		\noindent 
		\item highlightthickness \par
		Lebar puncak persegi panjang ketika widget memiliki fokus \par
		\noindent 
		\item padX padY \par
		Ruang tambahan widget dari manajer tata letak luar minimum widget perlu menampilkan isinya di x dan y arah \par
		\noindent 
		\item selectborderwidth \par
		Lebar perbatasan tiga dimensi disekitar dipilih item widget \par
		\noindent 
		\item wraplength \par
		Panjang garis maksimum untuk widget yang melakukan kata membungkus \par
		\noindent 
		\item height \par
		Tinggi diinginkan widget \par
		\noindent 
		\item underline \par
		Indeks karakter untuk menggarisawahi dalam teks widget  \par
		\noindent 
		\item width \par
		\noindent 
		\item Lebar diinginkan widget\end{itemize}
	\par
	\noindent 
	\item Warna \par
	\noindent 
	Tkinter memiliki warna dengan string. Ada dua cara umum untuk menentukan warna di Tkiter, yaitu : \par
	\noindent 
	\begin{itemize}
		\item Menggunakan string menentukan proporsi merah, hijau dan biru didigit heksadesimal. Misalnya  $ " $ $  \#  $ffff $ " $ putih,  $ " $ $  \#  $000000 $ " $ hitam dan  $ " $ $  \#  $000fff000 $ " $ hijau. \par
		\noindent 
		\item Menggunakan lokal standar nama warna . warna-warna  $ " $white $ " $, $ " $black $ " $,  $ " $green $ " $ dan  $ " $magenta $ " $ akan selalu tersedia.\end{itemize}
	\par
	\vspace{12pt}
	Pilihan warna umum : \par
	\noindent 
	\begin{itemize}
		\item activebackground \par
		Warna latar berlakang untuk widget ketika widget aktif \par
		\noindent 
		\item activeforeground \par
		Warna depan untuk widget ketika widget aktif \par
		\noindent 
		\item background \par
		Merepresentasikan sebagai \textit{bg} \par
		\noindent 
		\item disableforeground \par
		Warna depan untuk widget ketika widget dinonaktifkan \par
		\noindent 
		\item foreground \par
		Merepresentasikan fg \par
		\noindent 
		\item highlightbackground \par
		Warna latar belakang dari daerah puncak ketika widget memiliki fokus \par
		\noindent 
		\item hightlightcolor \par
		Warna depan dari wilayah puncak ketika widget memiliki fokus \par
		\noindent 
		\item selectbackground \par
		Warna latar belakang untuk item yang dipilih dari widget \par
		\noindent 
		\item selectforeground \par
		Warna depan untuk item yang dipilih dari widget \par
		\noindent 
		\item Font \par
		\noindent 
		Sebagai tupel yang elemen pertama adalah keluarga font diikuti dengan string yang berisi satu atau lebih gaya pengubah tebal,miring, garis bawah dan overstrike. \par
		\noindent 
		Contoh : \par
		\noindent 
		\item ( $ " $Helvetica $ " $, $ " $16 $ " $-point Helvetica biasa \par
		\noindent 
		\item ( $ " $Times $ " $, $ " $24 $ " $, $ " $beranimiring $ " $) untuk 24-point kali miring tebal\end{itemize}
	\par
	\vspace{12pt}
	Dapat membuat  $ " $font object $ " $ dengan mengimpor modul tkFont dan menggunakan kelas konstruktor font nya : \par
	Import tkFont \par
	Font = tkFont.Font (option, ....) \par
	\vspace{12pt}
	Berikut adalah daftar pilihan : \par
	\noindent 
	\begin{itemize}
		\item Family \par
		Font nama keluarga sebagai string \par
		\noindent 
		\item Size \par
		Font tinggi sebagai integer dalam poin \par
		\noindent 
		\item Weight \par
		Bold untuk teal, normal untuk berat badan secara teratur \par
		\noindent 
		\item Slant \par
		Italic untuk miring, roman untuk unstlanted \par
		\noindent 
		\item Underline \par
		1 untuk teks yang digarisbawahi, 0 untuk normal \par
		\noindent 
		\item Overstrike \par
		1 untuk teks telak, 0 untuk normal \par
		Jika berjalan di bawah X window system, dapat menggunakan salah satu nama font X. Sebagai contoh, font bernama  $ " $-*lucidatypewriter-medium-r-*-*-*-140-*-*-* $ " $ adalah favorit fixed-width font penulis untuk digunakan pada layar. \par
		\noindent 
		\item Jangkar \par
		\noindent 
		Jangkar digunakan untuk mendefinisikan mana teks diposisikan relatif terhadap titik acuan. Berikut adalah daftar kemungkinan konstanta yang dapat digunakan : \par
		\noindent 
		\item NW \par
		\noindent 
		\item N \par
		\noindent 
		\item NE \par
		\noindent 
		\item W \par
		\noindent 
		\item TENGAH \par
		\noindent 
		\item E \par
		\noindent 
		\item SW \par
		\noindent 
		\item S \par
		\noindent 
		\item SE\end{itemize}
	\par
	\vspace{12pt}
	Jika menggunakan tengah sebagai jangkar tek, tek akan ditengahkan horizontal dan vertikal disekitar titik referensi. \par
	Jangkar NW akan posisi teks sehingga titik referensi bertepatan dengan laut sudut kotak berisi teks \par
	Jangakr W akan pusat teks secara vertikal disekitar titik referensi dengan tepi kiri kotak teks yang melewati titik itu dan sebagainya. \par
	Jika membuat widget kecil didalam bingkai besar dan menggunakan jangkar = SE pilihan, widget akan ditempatkan disudut kanan bawah gambar. Jika menggunakan anchor = N sebaliknya widget akan dipusatkan disepanjang tepi atas. \par
	\noindent 
	\item Gaya relief \par
	\noindent 
	Widget mengacu pada efek 3-D simulasi terbaru disekitar bagian luar widget. Berikut adalah daftar konstanta yang mungkin dapat digunakan untuk atribut: \par
	\noindent 
	\begin{itemize}
		\item Datar \par
		\noindent 
		\item Dibesarkan \par
		\noindent 
		\item Cekung \par
		\noindent 
		\item Alur \par
		\noindent 
		\item Punggung bukit\end{itemize}
	\par
	\vspace{12pt}
	Contoh : \par
	{\fontsize{10pt}{10pt}\selectfont From Tkinter import *} \par
	{\fontsize{10pt}{10pt}\selectfont Import Tkinter} \par
	\vspace{10pt}
	{\fontsize{10pt}{10pt}\selectfont top = Tkinter.Tk()} \par
	{\fontsize{10pt}{10pt}\selectfont B1 = Tkinter.Button(top, text= $ " $FLAT $ " $, relief=FLAT)} \par
	{\fontsize{10pt}{10pt}\selectfont B2 = Tkinter.Button(top, text= $ " $RAISED $ " $, relief=RAISED)} \par
	{\fontsize{10pt}{10pt}\selectfont B3 =Tkinter.Button(top, text= $ " $SUNKEN $ " $, relief=SUNKEN)} \par
	{\fontsize{10pt}{10pt}\selectfont B4=Tkinter.Button(top, text= $ " $GROOVE $ " $, relief=GROOVE)} \par
	{\fontsize{10pt}{10pt}\selectfont B5=Tkinter.Button(top, text= $ " $RIDGE $ " $, relief=RIDGE)} \par
	\vspace{10pt}
	{\fontsize{10pt}{10pt}\selectfont B1.pack()} \par
	{\fontsize{10pt}{10pt}\selectfont B2.pack()} \par
	{\fontsize{10pt}{10pt}\selectfont B3.pack()} \par
	{\fontsize{10pt}{10pt}\selectfont B4.pack()} \par
	{\fontsize{10pt}{10pt}\selectfont B5.pack()} \par
	{\fontsize{10pt}{10pt}\selectfont top.mainloop()} \par
	\noindent 
	\item Britmaps \par
	\noindent 
	Ada beberapa jenis bitmap yang tersedia, diantaranya: \par
	\noindent 
	\begin{itemize}
		\item Kesalahan \par
		\noindent 
		\item Gray75 \par
		\noindent 
		\item Gray50 \par
		\noindent 
		\item Gray12 \par
		\noindent 
		\item Jam Pasir \par
		\noindent 
		\item Info \par
		\noindent 
		\item Questhead \par
		\noindent 
		\item Perantanyaan  \par
		\noindent 
		\item Peringatan\end{itemize}
	\par
	\vspace{12pt}
	Contoh: \par
	{\fontsize{10pt}{10pt}\selectfont From Tkinter import *} \par
	{\fontsize{10pt}{10pt}\selectfont Import Tkinter} \par
	\vspace{10pt}
	{\fontsize{10pt}{10pt}\selectfont Top = Tkinter.Tk()} \par
	\vspace{10pt}
	{\fontsize{10pt}{10pt}\selectfont B1 = Tkinter.Button(top, text = $ " $error $ " $, relief=RAISED,  $  \textbackslash  $ bitmap= $ " $error $ " $)} \par
	{\fontsize{10pt}{10pt}\selectfont B2 = Tkinter.Button(top, text = $ " $hourglass $ " $, relief=RAISED,  $  \textbackslash  $ bitmap= $ " $hourglass $ " $)} \par
	{\fontsize{10pt}{10pt}\selectfont B3 = Tkinter.Button(top, text = $ " $info $ " $, relief=RAISED,  $  \textbackslash  $ bitmap= $ " $info $ " $)} \par
	{\fontsize{10pt}{10pt}\selectfont B4 = Tkinter.Button(top, text = $ " $question $ " $, relief=RAISED,  $  \textbackslash  $ bitmap= $ " $question $ " $)} \par
	{\fontsize{10pt}{10pt}\selectfont B5 = Tkinter.Button(top, text = $ " $warning $ " $, relief=RAISED,  $  \textbackslash  $ bitmap= $ " $warning $ " $)} \par
	\vspace{10pt}
	{\fontsize{10pt}{10pt}\selectfont B1.pack()} \par
	{\fontsize{10pt}{10pt}\selectfont B2.pack()} \par
	{\fontsize{10pt}{10pt}\selectfont B3.pack()} \par
	{\fontsize{10pt}{10pt}\selectfont B4.pack()} \par
	{\fontsize{10pt}{10pt}\selectfont B5.pack()} \par
	{\fontsize{10pt}{10pt}\selectfont top.mainloop()} \par
	\noindent 
	\item Kursor \par
	\noindent 
	Berikut daftar menarik : \par
	\noindent 
	\begin{itemize}
		\item Panah \par
		\noindent 
		\item Lingkaran \par
		\noindent 
		\item Jam \par
		\noindent 
		\item Menyebrang \par
		\noindent 
		\item Dotbox \par
		\noindent 
		\item Bertukar \par
		\noindent 
		\item Fluer \par
		\noindent 
		\item Jantung \par
		\noindent 
		\item Manusia \par
		\noindent 
		\item Tikus \par
		\noindent 
		\item Bajak laut \par
		\noindent 
		\item Tamah \par
		\noindent 
		\item Antar jemput \par
		\noindent 
		\item Perekat \par
		\noindent 
		\item Laba-laba \par
		\noindent 
		\item Kaleng semprot \par
		\noindent 
		\item Bintang \par
		\noindent 
		\item Target \par
		\noindent 
		\item Tcross \par
		\noindent 
		\item Melakukan perjalanan \par
		\noindent 
		\item Menonton\end{itemize}
	\par
	\vspace{12pt}
	Contoh : \par
	{\fontsize{10pt}{10pt}\selectfont From Tkinter import *} \par
	{\fontsize{10pt}{10pt}\selectfont Import Tkinter} \par
	\vspace{10pt}
	{\fontsize{10pt}{10pt}\selectfont Top = Tkinter.Tk()} \par
	\vspace{10pt}
	{\fontsize{10pt}{10pt}\selectfont B1 = Tkinter.Button(top, text = $ " $circle $ " $, relief=RAISED,  $  \textbackslash  $ bitmap= $ " $circle $ " $)} \par
	{\fontsize{10pt}{10pt}\selectfont B2 = Tkinter.Button(top, text = $ " $plus $ " $, relief=RAISED,  $  \textbackslash  $ bitmap= $ " $plus $ " $)} \par
	\vspace{10pt}
	{\fontsize{10pt}{10pt}\selectfont B1.pack()} \par
	{\fontsize{10pt}{10pt}\selectfont B2.pack()} \par
	{\fontsize{10pt}{10pt}\selectfont top.mainloop()} \par
	\vspace{10pt}
	\noindent 

\section{Manajemen Geometri}
 \par
\noindent 
\hspace*{0.5in} Semua widget tkinter memiliki akses ke metode manajemen geometri tertentu, yang memiliki tujuan menggorganisir widget diseluruh wilayah widget induk. Tkinter mengekspos kelas manager geometri berikut : \par
\noindent 
\begin{itemize}
	\item Metode the \textit{pack()} \par
	\noindent 
	Manajer geometri ini mengatur widget diblok sebelum menempatkan mereka di widget induk \par
	\noindent 
	\item Metode the \textit{grid()} \par
	\noindent 
	Manajer geometri ini mengatur widget dalam struktur tabel seperti di widget induk \par
	\noindent 
	\item Metode~the  \textit{place()}\end{itemize}
\par
\noindent 
Manajer geometri ini mengatur widget dengan menempatkan dalam posisi tertentu dalam widget induk \par
\vspace{12pt}

\section{Manfaat Tkinter}
\par
Tkinter sangat sederhana. Erikut manfaat Tkinter dibandingkan GUI toolkit : \par
\noindent 
\begin{itemize}
	\item Tkinter mudah diakses oleh siapa saja. (Accessibilty)\vspace{\baselineskip}
	Tkinter merupakan toolkit yang ringan dan satu-satunya solusi GUI yang paling sederhana untuk Python sampai saat ini. Cukup menuliskan beberapa baris kode Python untuk membuat aplikasi GUI sederhana dengan Tkinter. Untuk menambahkan komponen baru pada Tkinter, dapat membuatnya dalam kode Python atau menambahkan paket ekstensi seperti Pmw, Tix, atau ttk. \par
	\noindent 
	\item Tkinter mudah digunakan di semua platform (Portability)\vspace{\baselineskip}
	Sebuah program Python yang dibangun menggunakan Tkinter dapat berjalan dengan baik di semua platform sistem operasi seperti Microsoft Windows, Linux, dan Macintosh. Dan dari segi tampilan window, akan terlihat sama dengan standar platform yang digunakan. \par
	\noindent 
	\item Tkinter selalu tersedia di Python (Availability)\vspace{\baselineskip}
	Tkinter merupakan modul standar pada pustaka Python. Sebagian besar paket instalasi Python sudah langsung berisi Tkinter. Khusus untuk beberapa distro Linux, perlu menambahkan paket Tkinter secara terpisah. Pada Windows, bisa langsung menggunakan Tkinter sesaat setelah menginstal paket instalasi Python. \par
	\noindent 
	\item Dokumentasi Tkinter sangat LUAR BIASA (Documentation)\vspace{\baselineskip}
	Python (plus Tkinter) ini bersifat open-source, maka banyak sekali komunitas-komunitas yng membahas Python dan Tkinter dan bisa belajar dan bertanya langsung dengan para ahli.\end{itemize}
\par
\vspace{12pt}

\chapter{Futher Expression}
\par
\vspace{14pt}
\noindent 
\hspace*{0.5in} Stiap kode yang dituliskan menggunakan bahasa yang dikompilasi seperti C, C++ atau Java dapat diintegrasikan ke skrip Python lainnya. Kode ini diagnggap sebagai ektensi. \par
\noindent 
\hspace*{0.5in} Modul ekstensi Python tidak lebih dari sekedar perpustakaan C biasa. Pada mesin Unix, perpustakaan ini biasanya diakhiri dengan .so (untuk objek bersama). Pada mesin windows, biasanya melihat .dll (untuk perpusatkaan yang terhubung secara dinamis).  \par
\vspace{12pt}
\noindent 

\section{Pra-Persyaratan untuk Menulis Ekstensi}
 \par
\noindent 
\hspace*{0.5in} Untuk memulai ektensi, memerlukan file header Python. Pada mesin Unix, biasanya memerlukan instalasi paket khusus pengembang seperti python 2-5. \par
\noindent 
\hspace*{0.5in} Pengguna window mendapatkan header ini sebagai bagian dari paket saat menggunakan pemasang Python biner. \par
\noindent 
\hspace*{0.5in} Harus memiliki pengetahuan yang baik tentang C atau C++ untuk menulis ekstensi Python menggunakan pemrograman C. \par
\noindent 
\hspace*{0.5in} Untuk melihat modul ekstensi Python, perlu mengelompokkan kode menjadi empat bagian : \par
\noindent 
\begin{itemize}
	\item File header Python h \par
	\noindent 
	\item Fungsi C yang ingin ditampilkan sebagai antarmuka dari modul \par
	\noindent 
	\item Sebuah tabel memetakan nama-nama fungsi saat pengembang Python melihat ke fungsi C didalam modul ekstensi \par
	\noindent 
	\item Fungsi inilisasi\end{itemize}
\par
\vspace{12pt}
Perlu menyertakan file header Python.h di file sumber C memberi akses ke API Python internal digunakan untuk menghitung modul ke penerjamah. \par
Menyertakan header Python.h sebelum header lain yang mungkin dibutuhkan. Mengikuti termasuk dengan fungsi yang ingin dipanggil dari Python. \par
Tanda tangan penerapan C fungsi selalu mengambil salah satu dari tiga bentuk berikut : \par
\noindent 
{\fontsize{10pt}{10pt}\selectfont static PyObject *MyFunction( PyObject *self, PyObject *args );} \par
\noindent 
\vspace{10pt}
\noindent 
{\fontsize{10pt}{10pt}\selectfont static PyObject *MyFunctionWithKeywords(PyObject *self,} \par
\noindent 
{\fontsize{10pt}{10pt}\selectfont ~~~~~~~~~~~~~~~~~~~~~~~~~~~~ PyObject *args,} \par
\noindent 
{\fontsize{10pt}{10pt}\selectfont ~~~~~~~~~~~~~~~~~~~~~~~~~~~~ PyObject *kw);} \par
\noindent 
\vspace{10pt}
\noindent 
{\fontsize{10pt}{10pt}\selectfont static PyObject *MyFunctionWithNoArgs( PyObject *self );} \par
\vspace{12pt}
\noindent 
\hspace*{0.5in} Masing-masing deklarasi seelumnya mengembalikan objek Python. Tidak ada yang namanya fungsi void dengan Python seperti ada di C. Jika ingin fungsi mengembalikan nilai, Python. Header Python mendefinisikan makro. Py $  \_  $Return $  \_  $None yang melakukan ini. \par
\noindent 
\hspace*{0.5in} Nama-nama fungsi C bisa menjadi apapun yang disuka karena tidak pernah diluar modul ekstensi mendefinisikan sebagai statis. \par
\noindent 
\hspace*{0.5in} Fungi Cbiasanya diberi nama dengan menggabnungkan modul dan fungsi Python bersama-sama yang ditunjukan disini : \par
\noindent 
static PyObject *\textit{module $  \_  $func}(PyObject *self, PyObject *args)  $  \{  $ \par
\noindent 
~~ /* Do your stuff here. */ \par
\noindent 
~~ Py $  \_  $RETURN $  \_  $NONE; \par
\noindent 
$  \}  $ \par
\vspace{12pt}
\vspace{12pt}
\noindent 
\hspace*{0.5in} Ini adalah fungsi Python yang disebut func didalam modul-modul. Memasukkan petunjuk ke fungsi C ke dalam tabel metode untuk modul yang biasanya muncul selanjutnya dikode sumber tael pemetaan metode. \par
\noindent 
\hspace*{0.5in} Tabel metode ini adalah susunan sederhana dari struktur PyMethodSef. Struktur itu terlihat seperti ini : \par
\noindent 
struct PyMethodDef  $  \{  $ \par
\noindent 
~~ char *ml $  \_  $name; \par
\noindent 
~~ PyCFunction ml $  \_  $meth; \par
\noindent 
~~ int ml $  \_  $flags; \par
\noindent 
~~ char *ml $  \_  $doc; \par
\noindent 
$  \}  $; \par
\vspace{12pt}
\vspace{12pt}
\vspace{12pt}
\noindent 
Inilai uraian anggota struktur ini : \par
\noindent 
\begin{itemize}
	\item Ml $  \_  $name \par
	Nama fungsi yang digunakan penafsir Python saat digunakan dalam program Python \par
	\noindent 
	\item Ml $  \_  $meth \par
	Menjadi alamaat ke fungsi yang memiliki salah satu tanda tangan yang dijelaskan dalam penelusuran sebelumnya \par
	\noindent 
	\item Ml $  \_  $flags \par
	Memberitahu penafsir yang mana dari tiga tanda tangan yang digunakan ml $  \_  $meth. Bendera ini biasanya mmiliki nilai meth $  \_  $varargs. Bendera ini dapat digandakan dengan or’ed dengan meth $  \_  $keywords jika ingin memiarkan argumen kata kunci masuk ke fungsi. Ini juga bisa memiliki nilai meth $  \_  $noargs yang menunjukan bahwa tidak ingin menerima argumen apa pun. \par
	\noindent 
	\item Ml $  \_  $doc \par
	Ini adalah docstring untuk fungsi yang bisa jadi NULL jika tidak ingin menulisnya. \par
	\vspace{12pt}
	\vspace{12pt}
	\noindent 
	\hspace*{0.5in} Tabel ini perlu diakhiri dengan sentinel yang terdiri dari NULL dan 0 untuk anggota yang sesuai. \par
	\vspace{12pt}
	\noindent 
	Contoh: \par
	\noindent 
	static PyMethodDef \textit{module} $  \_  $methods[] =  $  \{  $ \par
	\noindent 
	~~  $  \{  $ "\textit{func}", (PyCFunction)\textit{module $  \_  $func}, METH $  \_  $NOARGS, NULL  $  \}  $, \par
	\noindent 
	~~  $  \{  $ NULL, NULL, 0, NULL  $  \}  $ \par
	\noindent 
	$  \}  $; \par
	\vspace{12pt}
	\noindent 
	\hspace*{0.5in} Bagian terakhir dari modul ekstensi adalah fungsi inialisasi. Fungsi ini dipanggil oleh juru bahasa Python saat modul diisikan. Hal ini diperlukan agar fungsi diberi nama intiModule dimana modul adalah nama modul. \par
	\noindent 
	\hspace*{0.5in} Fungsi inialisasi perlu diekspor dari perpustakaan yang akan dibangun. Header Python mendefinisikan PyMODINIT $  \_  $Func untuk memasukkan mantra yang sesuai agar terjadi pada lingkungan tertentu tempat menyuusun. Yang harus dilakukan adalah mengunakan saat menentukan fungsinya. \par
	\noindent 
	\hspace*{0.5in} Fungsi inialisasi C umumnya memiliki strktur keseluruhan berikut : \par
	\noindent 
	PyMODINIT $  \_  $FUNC init\textit{Module}()  $  \{  $ \par
	\noindent 
	~~ Py $  \_  $InitModule3(\textit{func}, \textit{module} $  \_  $methods, "docstring..."); \par
	\noindent 
	$  \}  $ \par
	\vspace{12pt}
	\noindent 
	Berikut adalah penjelasan fugsi Py $  \_  $IntiModule : \par
	\noindent 
	\item Func  \par
	\noindent 
	Ini adalah fungsi yang akan diekspor \par
	\noindent 
	\item Module \par
	\noindent 
	Ini adalah nama tabel pemetaan yang didefinisikan diatas \par
	\noindent 
	\item Docstring\end{itemize}
\par
\noindent 
Ini adalah komentar yang ingin diberikan diekstensi \par
\vspace{12pt}
\noindent 
\hspace*{0.5in} Menempatkan ini semua bersama-sama terlihat sebagai berikut : \par
\noindent 
$  \#  $include <Python.h> \par
\vspace{12pt}
\noindent 
static PyObject *\textit{module $  \_  $func}(PyObject *self, PyObject *args)  $  \{  $ \par
\noindent 
~~ /* Do your stuff here. */ \par
\noindent 
~~ Py $  \_  $RETURN $  \_  $NONE; \par
\noindent 
$  \}  $ \par
\vspace{12pt}
\noindent 
static PyMethodDef \textit{module} $  \_  $methods[] =  $  \{  $ \par
\noindent 
~~  $  \{  $ "\textit{func}", (PyCFunction)\textit{module $  \_  $func}, METH $  \_  $NOARGS, NULL  $  \}  $, \par
\noindent 
~~  $  \{  $ NULL, NULL, 0, NULL  $  \}  $ \par
\noindent 
$  \}  $; \par
\vspace{12pt}
\noindent 
PyMODINIT $  \_  $FUNC init\textit{Module}()  $  \{  $ \par
\noindent 
~~ Py $  \_  $InitModule3(\textit{func}, \textit{module} $  \_  $methods, "docstring..."); \par
\noindent 
$  \}  $ \par
\vspace{12pt}
\noindent 
Contoh : \par
\noindent 
$  \#  $include <Python.h> \par
\vspace{12pt}
\noindent 
static PyObject* helloworld(PyObject* self) \par
\noindent 
$  \{  $ \par
\noindent 
~~~ return Py $  \_  $BuildValue("s", "Hello, Python extensions!!"); \par
\noindent 
$  \}  $ \par
\vspace{12pt}
\noindent 
static char helloworld $  \_  $docs[] = \par
\noindent 
~~~ "helloworld( ): Any message you want to put here!! $  \textbackslash  $n"; \par
\vspace{12pt}
\noindent 
static PyMethodDef helloworld $  \_  $funcs[] =  $  \{  $ \par
\noindent 
~~~  $  \{  $"helloworld", (PyCFunction)helloworld,  \par
\noindent 
~~~~ METH $  \_  $NOARGS, helloworld $  \_  $docs $  \}  $, \par
\noindent 
~~~  $  \{  $NULL $  \}  $ \par
\noindent 
$  \}  $; \par
\vspace{12pt}
\noindent 
void inithelloworld(void) \par
\noindent 
$  \{  $ \par
\noindent 
~~~ Py $  \_  $InitModule3("helloworld", helloworld $  \_  $funcs, \par
\noindent 
~~~~~~~~~~~~~~~~~~ "Extension module example!"); \par
\noindent 
$  \}  $ \par
\vspace{12pt}
\vspace{12pt}
\noindent 
\hspace*{0.5in} Disini fungsi Py $  \_  $BuildValue digunakan untuk membangun nilai Python.  \par
\vspace{12pt}
\vspace{12pt}
\noindent 

\section{Membangun dan Menginstal Ekstensi}
\par
\vspace{12pt}
\vspace{12pt}
\noindent 
\hspace*{0.5in} Distutils paket membuatnya sangat mudah mendistribusikan modul Python, baik Python murni dan modul ekstensi dengan cara standar. Modul didistribusikan dalam bentuk sumber dan dibangun dan diinstal melalui skrip setup yang iasa disebut seup.py sebagai berikut : \par
\noindent 
from distutils.core import setup, Extension \par
\noindent 
setup(name='helloworld', version='1.0',~  $  \textbackslash  $ \par
\noindent 
~~~~~ ext $  \_  $modules=[Extension('helloworld', ['hello.c'])]). \par
\vspace{12pt}
\noindent 
\hspace*{0.5in} Sekarang gunakan perintah berikut yang aka melalakukan semua kompilasi dan langkah penghubunh yang diperlukan dengan perintah dan bendera penyusun dan penghubung yang benar dan menyalin perpustakaan dinamis yang dihasilkan ke dalam direktori yang sesuai . \par
\vspace{12pt}
\noindent 
Contoh : \par
\noindent 
$  \$  $ python setup.py install \par
\vspace{12pt}
\noindent 
\hspace*{0.5in} Pada sistem berbasis Unix kemungkinan besar perlu menjalankan perintah ini sebagai root agar meminta izin untuk menulis ke direktori paket situs. Ini biasanya tidak menjadi masalah pada window. \par
Setelah menginstal ekstensi, akan dapat mengimpor dan memanggil ekstensi tersebut di skrip Python  sebagai berikut : \par
\noindent 
$  \#  $!/usr/bin/python \par
\noindent 
import helloworld \par
\vspace{12pt}
\noindent 
print helloworld.helloworld() \par
\vspace{14pt}
\noindent 
Ini akan menghasilkan hasil sebagai berikut  : \par
\noindent 
Hello, Python extensions!! \par
\vspace{12pt}
Seperti kemungkinan besar ingin mendefinisikan fungsi yang menerima argumen, dapat menggunakan salah satu tanda tangan lain untuk fungsi C. Sebagai contoh, fungsi berikut yang menerima beberapa parameter akan didefinisikan seperti ini : \par
\noindent 
static PyObject *\textit{module $  \_  $func}(PyObject *self, PyObject *args)  $  \{  $ \par
\noindent 
~~ /* Parse args and do something interesting here. */ \par
\noindent 
~~ Py $  \_  $RETURN $  \_  $NONE; \par
\noindent 
$  \}  $ \par
\vspace{12pt}
Tabel metode yang berisi entri untuk fungsi baru akan terlihat seperti ini : \par
\noindent 
static PyMethodDef \textit{module} $  \_  $methods[] =  $  \{  $ \par
\noindent 
~~  $  \{  $ "\textit{func}", (PyCFunction)\textit{module $  \_  $func}, METH $  \_  $NOARGS, NULL  $  \}  $, \par
\noindent 
~~  $  \{  $ "\textit{func}", \textit{module $  \_  $func}, METH $  \_  $VARARGS, NULL  $  \}  $, \par
\noindent 
~~  $  \{  $ NULL, NULL, 0, NULL  $  \}  $ \par
\noindent 
$  \}  $; \par
\vspace{12pt}
Menggunakan fungsi API PyArg $  \_  $ParseTuple untuk mengekstrak argumen dari satu pointer PyObject yang dikirimkan ke fungsi C. Argumen pertama untuk PyArg $  \_  $ParseTuple adalah args argumen. Ini adalah objek yang akan parsing. Argumen kedua adalah string format yang menggambarkan argumen saat mengharapkannya muncul. Setiap argumen diwakili oleh satu atau lebih karakter dalam format string sebagai berikut : \par
\noindent 
static PyObject *\textit{module $  \_  $func}(PyObject *self, PyObject *args)  $  \{  $ \par
\noindent 
~~ int i; \par
\noindent 
~~ double d; \par
\noindent 
~~ char *s; \par
\vspace{12pt}
\noindent 
~~ if (!PyArg $  \_  $ParseTuple(args, "ids",  $  \&  $i,  $  \&  $d,  $  \&  $s))  $  \{  $ \par
\noindent 
~~~~~ return NULL; \par
\noindent 
~~  $  \}  $ \par
\noindent 
~~  \par
\noindent 
~~ /* Do something interesting here. */ \par
\noindent 
~~ Py $  \_  $RETURN $  \_  $NONE; \par
\noindent 
$  \}  $ \par
\vspace{12pt}
Mengkompilasi versi baru dari modul dan mengimpornya memungkinkan untuk memanggil fungsi baru dengan sejumlah argumen dari jenis apa pun : \par
\noindent 
module.func(1, s="three", d=2.0) \par
\noindent 
module.func(i=1, d=2.0, s="three") \par
\noindent 
module.func(s="three", d=2.0, i=1) \par
\vspace{12pt}
\vspace{12pt}
\vspace{12pt}
\vspace{12pt}
\noindent 
\hspace*{0.5in} Berikut adalah tanda tangan standar untuk fungsi PyArg $  \_  $ParseTuple: \par
\noindent 
int PyArg $  \_  $ParseTuple(PyObject* tuple,char* format,...) \par
\vspace{12pt}
\noindent 
\hspace*{0.5in} Fungsi ini mengembalikan 0 untuk kesalahan, dan nilai tidak sama dengan 0 untuk kesuksesan. Tuple adalah PyObject * yang merupakan argumen kedua dari fungsi C. Format berikut adalah string C yang menggambarkan argumen wajib dan opsional.\vspace{\baselineskip}
~~~~~~ Berikut adalah daftar kode format untuk fungsi PyArg $  \_  $ParseTuple: \par


%%%%%%%%%%%%  Table No:1 Here %%%%%%%%%%%%%%


\begin{table}[H]
	\centering
	\begin{adjustbox}{width=\textwidth}
		\begin{tabular}{ p{0.52in}p{1.51in}p{2.2in} }
			\hhline{---}
			\multicolumn{1}{|p{0.52in}}{\Centering \textbf{Code}} & \multicolumn{1}{|p{1.51in}}{\Centering \textbf{C type}} & \multicolumn{1}{|p{2.2in}|}{\Centering \textbf{Meaning}} & \hhline{---}
			\multicolumn{1}{|p{0.52in}}{c} & \multicolumn{1}{|p{1.51in}}{char} & \multicolumn{1}{|p{2.2in}|}{String Python dengan panjang 1 menjadi huruf C.} & \hhline{---}
			\multicolumn{1}{|p{0.52in}}{d} & \multicolumn{1}{|p{1.51in}}{double} & \multicolumn{1}{|p{2.2in}|}{Pelampung Python menjadi C ganda.} & \hhline{---}
			\multicolumn{1}{|p{0.52in}}{f} & \multicolumn{1}{|p{1.51in}}{float} & \multicolumn{1}{|p{2.2in}|}{Pelampung Python menjadi pelampung C.} & \hhline{---}
			\multicolumn{1}{|p{0.52in}}{i} & \multicolumn{1}{|p{1.51in}}{int} & \multicolumn{1}{|p{2.2in}|}{Int Python menjadi int int} & \hhline{---}
			\multicolumn{1}{|p{0.52in}}{l} & \multicolumn{1}{|p{1.51in}}{long} & \multicolumn{1}{|p{2.2in}|}{Sebuah int Python menjadi panjang C.} & \hhline{---}
			\multicolumn{1}{|p{0.52in}}{L} & \multicolumn{1}{|p{1.51in}}{long long} & \multicolumn{1}{|p{2.2in}|}{Sebuah int Python menjadi C panjang panjang} & \hhline{---}
			\multicolumn{1}{|p{0.52in}}{O} & \multicolumn{1}{|p{1.51in}}{PyObject*} & \multicolumn{1}{|p{2.2in}|}{Gets non-NULL meminjam referensi ke argumen Python.} & \hhline{---}
			\multicolumn{1}{|p{0.52in}}{s} & \multicolumn{1}{|p{1.51in}}{char*} & \multicolumn{1}{|p{2.2in}|}{Python string tanpa nulls tertanam ke C char *.} & \hhline{---}
			\multicolumn{1}{|p{0.52in}}{s $  \#  $} & \multicolumn{1}{|p{1.51in}}{char*+int} & \multicolumn{1}{|p{2.2in}|}{Setiap string Python ke alamat dan panjang C.} & \hhline{---}
			\multicolumn{1}{|p{0.52in}}{t $  \#  $} & \multicolumn{1}{|p{1.51in}}{char*+int} & \multicolumn{1}{|p{2.2in}|}{Read-only penyangga segmen tunggal ke alamat C dan panjangnya.} & \hhline{---}
			\multicolumn{1}{|p{0.52in}}{u} & \multicolumn{1}{|p{1.51in}}{Py $  \_  $UNICODE*} & \multicolumn{1}{|p{2.2in}|}{Python Unicode tanpa nulls tertanam ke C.} & \hhline{---}
			\multicolumn{1}{|p{0.52in}}{u $  \#  $} & \multicolumn{1}{|p{1.51in}}{Py $  \_  $UNICODE*+int} & \multicolumn{1}{|p{2.2in}|}{Setiap alamat dan panjang Python Unicode C.} & \hhline{---}
			\multicolumn{1}{|p{0.52in}}{w $  \#  $} & \multicolumn{1}{|p{1.51in}}{char*+int} & \multicolumn{1}{|p{2.2in}|}{Membaca / menulis penyangga segmen tunggal ke alamat dan panjang C.} & \hhline{---}
			\multicolumn{1}{|p{0.52in}}{z} & \multicolumn{1}{|p{1.51in}}{char*} & \multicolumn{1}{|p{2.2in}|}{Seperti s, juga menerima None (set C char * ke NULL).} & \hhline{---}
			\multicolumn{1}{|p{0.52in}}{z $  \#  $} & \multicolumn{1}{|p{1.51in}}{char*+int} & \multicolumn{1}{|p{2.2in}|}{Seperti s  $  \#  $, juga menerima None (set C char * ke NULL).} & \hhline{---}
			\multicolumn{1}{|p{0.52in}}{(...)} & \multicolumn{1}{|p{1.51in}}{as per ...} & \multicolumn{1}{|p{2.2in}|}{Urutan Python diperlakukan sebagai satu argumen per item.} & \hhline{---}
			\multicolumn{1}{|p{0.52in}}{ $  \vert  $} & \multicolumn{1}{|p{1.51in}}{ $  $} & \multicolumn{1}{|p{2.2in}|}{Argumen berikut bersifat opsional.} & \hhline{---}
			\multicolumn{1}{|p{0.52in}}{:} & \multicolumn{1}{|p{1.51in}}{ $  $} & \multicolumn{1}{|p{2.2in}|}{Format akhir, diikuti dengan nama fungsi untuk pesan error.} & \hhline{---}
			\multicolumn{1}{|p{0.52in}}{;} & \multicolumn{1}{|p{1.51in}}{ $  $} & \multicolumn{1}{|p{2.2in}|}{Format akhir, diikuti oleh seluruh pesan kesalahan teks.} & \hline
		\end{tabular}
	\end{adjustbox}
\end{table}


%%%%%%%%%%%%  Table No:1 Ends Here %%%%%%%%%%%%%%


\vspace{12pt}
\vspace{14pt}
Py $  \_  $BuildValue mengambil format string seperti PyArg $  \_  $ParseTuple. Alih-alih menyampaikan~alamat nilai yang sedang  bangun, melewati nilai sebenarnya. Berikut adalah contoh yang menunjukkan bagaimana menerapkan fungsi tambah : \par
\noindent 
static PyObject *foo $  \_  $add(PyObject *self, PyObject *args)  $  \{  $ \par
\noindent 
~~ int a; \par
\noindent 
~~ int b; \par
\vspace{12pt}
\noindent 
~~ if (!PyArg $  \_  $ParseTuple(args, "ii",  $  \&  $a,  $  \&  $b))  $  \{  $ \par
\noindent 
~~~~~ return NULL; \par
\noindent 
~~  $  \}  $ \par
\noindent 
~~ return Py $  \_  $BuildValue("i", a + b); \par
\noindent 
$  \}  $ \par
\vspace{14pt}
Ini adalah apa yang akan terlihat seperti jika diimplementasikan dengan Python : \par
\noindent 
def add(a, b): \par
\noindent 
~~ return (a + b) \par
\vspace{16pt}
Mengembalikan dua nilai dari fungsi sebagai berikut, ini akan dipicu menggunakan daftar dengan Python : \par
\noindent 
static PyObject *foo $  \_  $add $  \_  $subtract(PyObject *self, PyObject *args)  $  \{  $ \par
\noindent 
~~ int a; \par
\noindent 
~~ int b; \par
\vspace{12pt}
\noindent 
~~ if (!PyArg $  \_  $ParseTuple(args, "ii",  $  \&  $a,  $  \&  $b))  $  \{  $ \par
\noindent 
~~~~~ return NULL; \par
\noindent 
~~  $  \}  $ \par
\noindent 
~~ return Py $  \_  $BuildValue("ii", a + b, a - b); \par
\noindent 
$  \}  $ \par
\vspace{16pt}
Ini adalah apa yang akan terlihat seperti jika diimplementasikan dengan Python : \par
\noindent 
def add $  \_  $subtract(a, b): \par
\noindent 
~~ return (a + b, a - b) \par
\vspace{14pt}
Berikut adalah tanda tangan standar untuk fungsi Py $  \_  $BuildValue  : \par
\noindent 
PyObject* Py $  \_  $BuildValue(char* format,...) \par
\vspace{14pt}
Format berikut adalah string C yang menggambarkan objek Python untuk dibangun. Argumentasi berikut Py $  \_  $BuildValue adalah nilai C dari mana hasilnya dibuat. Hasil PyObject * adalah referensi baru.  \par
Berikut daftar tabel string kode yang umum digunakan, yang nol atau lebihnya digabungkan ke dalam format string : \par


%%%%%%%%%%%%  Table No:2 Here %%%%%%%%%%%%%%


\begin{table}[H]
	\centering
	\begin{adjustbox}{width=\textwidth}
		\begin{tabular}{ p{0.51in}p{1.48in}p{1.83in} }
			\hhline{---}
			\multicolumn{1}{|p{0.51in}}{\Centering \textbf{Code}} & \multicolumn{1}{|p{1.48in}}{\Centering \textbf{Tipe C }} & \multicolumn{1}{|p{1.83in}|}{\textbf{Meaning}} & \hhline{---}
			\multicolumn{1}{|p{0.51in}}{c} & \multicolumn{1}{|p{1.48in}}{char} & \multicolumn{1}{|p{1.83in}|}{Sebuah char C menjadi string Python dengan panjang 1.} & \hhline{---}
			\multicolumn{1}{|p{0.51in}}{d} & \multicolumn{1}{|p{1.48in}}{double} & \multicolumn{1}{|p{1.83in}|}{C ganda menjadi float Python.} & \hhline{---}
			\multicolumn{1}{|p{0.51in}}{f} & \multicolumn{1}{|p{1.48in}}{float} & \multicolumn{1}{|p{1.83in}|}{Pelampung C menjadi float Python.} & \hhline{---}
			\multicolumn{1}{|p{0.51in}}{i} & \multicolumn{1}{|p{1.48in}}{int} & \multicolumn{1}{|p{1.83in}|}{C int menjadi int Python.} & \hhline{---}
			\multicolumn{1}{|p{0.51in}}{l} & \multicolumn{1}{|p{1.48in}}{long} & \multicolumn{1}{|p{1.83in}|}{Sebuah C panjang menjadi int Python.} & \hhline{---}
			\multicolumn{1}{|p{0.51in}}{N} & \multicolumn{1}{|p{1.48in}}{PyObject*} & \multicolumn{1}{|p{1.83in}|}{Melewati objek Python dan mencuri referensi.} & \hhline{---}
			\multicolumn{1}{|p{0.51in}}{O} & \multicolumn{1}{|p{1.48in}}{PyObject*} & \multicolumn{1}{|p{1.83in}|}{Melewati objek Python dan MENINGKATKANNYA seperti biasa.} & \hhline{---}
			\multicolumn{1}{|p{0.51in}}{O $  \&  $} & \multicolumn{1}{|p{1.48in}}{convert+void*} & \multicolumn{1}{|p{1.83in}|}{Konversi sewenang-wenang} & \hhline{---}
			\multicolumn{1}{|p{0.51in}}{s} & \multicolumn{1}{|p{1.48in}}{char*} & \multicolumn{1}{|p{1.83in}|}{C 0-diakhiri char * ke string Python, atau NULL to None.} & \hhline{---}
			\multicolumn{1}{|p{0.51in}}{s $  \#  $} & \multicolumn{1}{|p{1.48in}}{char*+int} & \multicolumn{1}{|p{1.83in}|}{C char * dan panjang ke string Python, atau NULL to None.} & \hhline{---}
			\multicolumn{1}{|p{0.51in}}{u} & \multicolumn{1}{|p{1.48in}}{Py $  \_  $UNICODE*} & \multicolumn{1}{|p{1.83in}|}{String C-wide, null-terminated menjadi Python Unicode, atau NULL to None.} & \hhline{---}
			\multicolumn{1}{|p{0.51in}}{u $  \#  $} & \multicolumn{1}{|p{1.48in}}{Py $  \_  $UNICODE*+int} & \multicolumn{1}{|p{1.83in}|}{String dan panjang lebar C ke Unicode Python, atau NULL to None.} & \hhline{---}
			\multicolumn{1}{|p{0.51in}}{w $  \#  $} & \multicolumn{1}{|p{1.48in}}{char*+int} & \multicolumn{1}{|p{1.83in}|}{Membaca / menulis penyangga segmen tunggal ke alamat dan panjang C.} & \hhline{---}
			\multicolumn{1}{|p{0.51in}}{z} & \multicolumn{1}{|p{1.48in}}{char*} & \multicolumn{1}{|p{1.83in}|}{Seperti s, juga menerima None (set C char * ke NULL).} & \hhline{---}
			\multicolumn{1}{|p{0.51in}}{z $  \#  $} & \multicolumn{1}{|p{1.48in}}{char*+int} & \multicolumn{1}{|p{1.83in}|}{Seperti s  $  \#  $, juga menerima None (set C char * ke NULL).} & \hhline{---}
			\multicolumn{1}{|p{0.51in}}{(...)} & \multicolumn{1}{|p{1.48in}}{as per ...} & \multicolumn{1}{|p{1.83in}|}{Membangun tuple Python dari nilai C.} & \hhline{---}
			\multicolumn{1}{|p{0.51in}}{[...]} & \multicolumn{1}{|p{1.48in}}{as per ...} & \multicolumn{1}{|p{1.83in}|}{Membangun daftar Python dari nilai C.} & \hhline{---}
			\multicolumn{1}{|p{0.51in}}{ $  \{  $... $  \}  $} & \multicolumn{1}{|p{1.48in}}{as per ...} & \multicolumn{1}{|p{1.83in}|}{Bangun kamus Python dari nilai C, kunci dan nilai bergantian.} & \hline
		\end{tabular}
	\end{adjustbox}
\end{table}


%%%%%%%%%%%%  Table No:2 Ends Here %%%%%%%%%%%%%%


\vspace{12pt}
Kode  $  \{  $... $  \}  $ membangun kamus dari sejumlah nilai C, kunci dan nilai bergantian. Misalnya, Py $  \_  $BuildValue (" $  \{  $issi $  \}  $", 23, "zig", "zag", 42) mengembalikan kamus seperti  $  \{  $23: 'zig', 'zag': 42 $  \}  $ Python. \par
Setiap blok memori yang dialokasikan dengan malloc () pada akhirnya harus dikembalikan ke genangan memori yang tersedia dengan satu panggilan untuk membebaskan (). Penting untuk menelepon gratis () pada waktu yang tepat. Jika alamat blok dilupakan tapi gratis () tidak dipanggil untuk itu, memori yang ditempatinya tidak dapat digunakan kembali sampai program berakhir. Ini disebut kebocoran memori. Di sisi lain, jika sebuah program memanggil gratis () untuk satu blok dan kemudian terus menggunakan blok tersebut, itu menciptakan konflik dengan penggunaan ulang blok melalui panggilan malloc () yang lain. Ini disebut dengan menggunakan memori yang dibebaskan. Ini memiliki konsekuensi buruk yang sama seperti merujuk pada data yang tidak diinisiasi - dump inti, hasil yang salah, crash misterius. \par
Karena Python membuat penggunaan malloc () dan gratis (), dibutuhkan strategi untuk menghindari kebocoran memori dan juga penggunaan memori yang bebas. Metode yang dipilih disebut penghitungan referensi. Prinsipnya sederhana: setiap objek berisi sebuah counter, yang bertambah saat referensi ke objek disimpan di suatu tempat, dan yang dikurangi saat referensi itu dihapus. Saat counter mencapai nol, referensi terakhir ke objek telah dihapus dan objeknya dibebaskan. \par
\vspace{12pt}

\appendix{Alternate Reference Styles}

\begin{references}{3.}
\bibitem{kilby}J. S. Kilby,
``Invention of the Integrated Circuit,'' {\it IEEE Trans. Electron Devices,}
{\bf ED-23,} 648 (1976).

\bibitem{hamming}R. W. Hamming,
                 {\it Numerical Methods for Scientists and 
                 Engineers}, Chapter N-1, McGraw-Hill, 
                 New York, 1962.

\bibitem{Hu}J. Lee, K. Mayaram, and C. Hu, ``A Theoretical
               Study of Gate/Drain Offset in LDD MOSFETs''
                     {\it IEEE Electron Device Lett.,} {\bf EDL-7}(3). 152 
                     (1986).

\bibitem{beren}A. Berenbaum, 
B. W. Colbry, D.R. Ditzel, R. D Freeman, and 
K.J. O'Connor, ``A Pipelined 32b Microprocessor with 13 kb of Cache Memory,''
{it Int. Solid State Circuit Conf., Dig. Tech. Pap.,} p. 34 (1987).
\end{references}


\begin{references}{Ham62}
\bibitem[Kil76]{kilb}J. S. Kilby,
``Invention of the Integrated Circuit,'' {\it IEEE Trans. Electron Devices,}
{\bf ED-23,} 648 (1976).

\bibitem[Ham62]{hamm}R. W. Hamming,
                 {\it Numerical Methods for Scientists and 
                 Engineers}, Chapter N-1, McGraw-Hill, 
                 New York, 1962.

\bibitem[Hu86]{lee}J. Lee, K. Mayaram, and C. Hu, ``A Theoretical
               Study of Gate/Drain Offset in LDD MOSFETs''
                     {\it IEEE Electron Device Lett.,} {\bf EDL-7}(3). 152 
                     (1986).

\bibitem[Ber87]{berm}A. Berenbaum, 
B. W. Colbry, D.R. Ditzel, R. D Freeman, and 
K.J. O'Connor, ``A Pipelined 32b Microprocessor with 13 kb of Cache Memory,''
{it Int. Solid State Circuit Conf., Dig. Tech. Pap.,} p. 34 (1987).

\end{references}



%%%%%%%%%%%%%%%
%%  The default LaTeX Index
%%  Don't need to add any commands before \begin{document}
\printindex

%%%% Making an index
%% 
%% 1. Make index entries, don't leave any spaces so that they
%% will be sorted correctly.
%% 
%% \index{term}
%% \index{term!subterm}
%% \index{term!subterm!subsubterm}
%% 
%% 2. Run LaTeX several times to produce <filename>.idx
%% 
%% 3. On command line, type  makeindx <filename> which
%% will produce <filename>.ind 
%% 
%% 4. Type \printindex to make the index appear in your book.
%% 
%% 5. If you would like to edit <filename>.ind 
%% you may do so. See docs.pdf for more information.
%% 
%%%%%%%%%%%%%%%%%%%%%%%%%%%%%%

%%%%%%%%%%%%%% Making Multiple Indices %%%%%%%%%%%%%%%%
%% 1. 
%% \usepackage{multind}
%% \makeindex{book}
%% \makeindex{authors}
%% \begin{document}
%% 
%% 2.
%% % add index terms to your book, ie,
%% \index{book}{A term to go to the topic index}
%% \index{authors}{Put this author in the author index}
%% 
%% \index{book}{Cows}
%% \index{book}{Cows!Jersey}
%% \index{book}{Cows!Jersey!Brown}
%% 
%% \index{author}{Douglas Adams}
%% \index{author}{Boethius}
%% \index{author}{Mark Twain}
%% 
%% 3. On command line type 
%% makeindex topic 
%% makeindex authors
%% 
%% 4.
%% this is a Wiley command to make the indices print:
%% \multiprintindex{book}{Topic index}
%% \multiprintindex{authors}{Author index}

\end{document}

