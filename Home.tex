
\sloppy
{\fontsize{14pt}{14pt}\selectfont HOME \\} \par
\noindent 
{\fontsize{14pt}{14pt}\selectfont Python adalah bahas pemrograman interpretatif multiguna dengan filosofi perancangan yang berfokus pada tingkat keterbacaan kode. Python yang diklaim sebagai bahasa yang menggabungkan kapabilitas, kemampuan, dengan sintaksis kode yang sangat jelas, dan dilengkapi dengan fungsionalitas pustaka standar yang besar serta komprehensif. \\} \par
\noindent 
{\fontsize{14pt}{14pt}\selectfont \vspace{\baselineskip}
Python mendukung multi paradigma pemrograman, utamanya; namun tidak dibatasi; pada pemrograman berorientasi objek, pemrograman imperatif, dan pemrograman fungsional. Salah satu fitur yang tersedia pada python adalah sebagai bahasa pemrograman dinamis yang dilengkapi dengan manajemen memori otomatis. Seperti halnya pada bahasa pemrograman dinamis lainnya, python umumnya digunakan sebagai bahasa skrip meski pada praktiknya penggunaan bahasa ini lebih luas mencakup konteks pemanfaatan yang umumnya tidak dilakukan dengan menggunakan bahasa skrip. Python dapat digunakan untuk berbagai keperluan pengembangan perangkat lunak dan dapat berjalan di berbagai platform sistem operasi. \\} \par
\noindent 
{\fontsize{14pt}{14pt}\selectfont Python didistribusikan dengan beberapa lisensi yang berbeda dari beberapa versi. Namun pada prinsipnya Python dapat diperoleh dan digunakan secara bebas, bahkan untuk kepentingan komersial. Lisensi python tidak bertentangan baik menurut definisi Open Source maupun General Public License (GPL).\vspace{\baselineskip}
Bebrapa fitur yang dimiliki Python adalah: \\} \par
\noindent 
{\fontsize{14pt}{14pt}\selectfont Memiliki kepustakaan yang luas dalam distribusi Python telah disediakan modul modul 'siap pakai; untuk berbagai keperluan \\} \par
\noindent 
{\fontsize{14pt}{14pt}\selectfont Memiliki tata bahasa yang jernih dan mudah dipelajari \\} \par
\noindent 
{\fontsize{14pt}{14pt}\selectfont Memiliki aturan layout kode sumber yang memudahkan pengecekan, pembacaan kembali dan penulisan ulang kode \\} \par
\vspace{14pt}
\noindent 
{\fontsize{14pt}{14pt}\selectfont Memiliki sistem pengolahan memori otomatis (garbage collectuion, seperti java) \\} \par
\noindent 
{\fontsize{14pt}{14pt}\selectfont Modular, mudah dikembangkan dengan menciptakan modul-modul baru, modul-modul tersebut dapat dibangun dengan bahasa Python maupun C/C++ \\} \par
\noindent 
{\fontsize{14pt}{14pt}\selectfont Memiliki fasilitas pengumpulan sampah otomatis, seperti halnya pada bahasa pemrograman Java, Python memiliki fasilitas pengaturan penggunaan ingatan komputer sehingga para pemrogram tidak perlu melakukan pengaturan ingatan komputer secara langsung \\} \par
\noindent 
{\fontsize{14pt}{14pt}\selectfont Memiliki banyak fasilitas pendukung sehingga mudah dalam pengoprasiannya \\} \par
\vspace{14pt}
\noindent 
{\fontsize{14pt}{14pt}\selectfont Sejarah \\} \par
\noindent 
{\fontsize{14pt}{14pt}\selectfont Bahasa pemrograman Python adalah bahasa yang dibuat oleh seorang keturunan Belanda yaitu Guido van Rossum. Awalnya, pembuatan bahasa pemrograman ini adalah untuk membuat skrip bahasa tingkat tinggi pada sebuah sistem operasi yang terdistribusi Amoeba. Python telah digunakan oleh beberapa pengembang dan bahkan digunakan oleh beberapa perusahaan untuk pembuatan perangkat lunak komersial. \\} \par
\noindent 
{\fontsize{14pt}{14pt}\selectfont Pemrograman bahasa python ini adalah pemrogram gratis atau freeware, sehingga dapat dikembangkan, dan tidak ada batasan dalam penyalinannya dan mendistribusikan. \\} \par
\vspace{14pt}
\noindent 
{\fontsize{14pt}{14pt}\selectfont Dukungan Komunitas yang Aktif \\} \par
\noindent 
{\fontsize{14pt}{14pt}\selectfont Python adalah salah satu pemrograman yang terus berkembang dan bertahan dikarenakan dukungan komunitas yang aktif diseluruh dunia. Banyak forum-forum ataupun blogger-blogger yang sering membagi pengalaman dalam menggunakan python. Hal ini memudahkan bagi pengguna pemula maupun pengembang untuk bertanya dan sharing tentang ilmu pemrograman ini. \\} \par
\noindent 
{\fontsize{14pt}{14pt}\selectfont Kelebihan dan Kekurangan \\} \par
\noindent 
{\fontsize{14pt}{14pt}\selectfont Kelebihan : \\} \par
\noindent 
{\fontsize{14pt}{14pt}\selectfont Tidak ada tahapan kompilasi dan penyambungan (link) sehingga kecepatan perubahan pada masa pembuatan sistem aplikasi meningkat. \\} \par
\noindent 
{\fontsize{14pt}{14pt}\selectfont Tidak ada deklarasi tipe data yang merumitkan sehingga program menjadi lebih sederhana, singkat, dan fleksible. \\} \par
\noindent 
{\fontsize{14pt}{14pt}\selectfont Manajemen memori otomatis yaitu kumpulan sampah memori sehingga dapat menghindari pencacatan kode. \\} \par
\noindent 
{\fontsize{14pt}{14pt}\selectfont Tipe data dan operasi tingkat tinggi yaitu kecepatan pembuatan sistem aplikasi menggunakan tipe objek yang telah ada. \\} \par
\noindent 
{\fontsize{14pt}{14pt}\selectfont Pemrograman berorientasi objek. \\} \par
\noindent 
{\fontsize{14pt}{14pt}\selectfont Pelekatan dan perluasan dalam C. \\} \par
\noindent 
{\fontsize{14pt}{14pt}\selectfont Terdapat kelas, modul, eksepsi sehingga terdapat dukungan pemrograman skala besar secara modular. \\} \par
\noindent 
{\fontsize{14pt}{14pt}\selectfont Pemuatan dinamis modul C sehingga ekstensi menjadi sederhana dan berkas biner yang kecil \\} \par
\noindent 
{\fontsize{14pt}{14pt}\selectfont Pemuatan kembali secara dinamis modul phyton seperti memodifikasi aplikasi tanpa menghentikannya. \\} \par
\noindent 
{\fontsize{14pt}{14pt}\selectfont Model objek universal kelas Satu. \\} \par
\noindent 
{\fontsize{14pt}{14pt}\selectfont Konstruksi pada saat aplikasi berjalan. \\} \par
\noindent 
{\fontsize{14pt}{14pt}\selectfont Interaktif, dinamis dan alamiah. \\} \par
\noindent 
{\fontsize{14pt}{14pt}\selectfont Akses hingga informasi interpreter. \\} \par
\noindent 
{\fontsize{14pt}{14pt}\selectfont Portabilitas secara luas seperti pemrograman antar platform tanpa ports. \\} \par
\noindent 
{\fontsize{14pt}{14pt}\selectfont Kompilasi untuk portable kode byte sehingga kecepatan eksekusi bertambah dan melindungi kode sumber. \\} \par
\noindent 
{\fontsize{14pt}{14pt}\selectfont Antarmuka terpasang untuk pelayanan keluar seperti perangkat Bantu system, GUI, persistence, database, dll. \\} \par
\noindent 
{\fontsize{14pt}{14pt}\selectfont Kekurangan : \\} \par
\noindent 
{\fontsize{14pt}{14pt}\selectfont Beberapa penugasan terdapat diluar dari jangkauan python, seperti bahasa pemrograman dinamis lainnya, python tidak secepat atau efisien sebagai statis, tidak seperti bahasa pemrograman kompilasi seperti bahasa C. \\} \par
\noindent 
{\fontsize{14pt}{14pt}\selectfont Disebabkan python merupakan interpreter, python bukan merupakan perangkat bantu terbaik untuk pengantar komponen performa kritis. \\} \par
\noindent 
{\fontsize{14pt}{14pt}\selectfont Python tidak dapat digunakan sebagai dasar bahasa pemrograman implementasi untuk beberapa komponen, tetapi dapat bekerja dengan baik sebagai bagian depan skrip antarmuka untuk mereka. \\} \par
\noindent 
{\fontsize{14pt}{14pt}\selectfont Python memberikan efisiensi dan fleksibilitas tradeoff by dengan tidak memberikannya secara menyeluruh. Python menyediakan bahasa pemrograman optimasi untuk kegunaan, bersama dengan perangkat bantu yang dibutuhkan untuk diintegrasikan dengan bahasa pemrograman lainnya. \\} \par
\noindent 
{\fontsize{14pt}{14pt}\selectfont Banyak terdapat referensi lama terutama dari pencarian google, python adalah pemrograman yang sangat lambat. Namun belum lama ini ditemukan bahwa Google, Youtube, DropBox dan beberapa software sistem banyak menggunakan Python. \\} \par
\noindent 
{\fontsize{14pt}{14pt}\selectfont Kini Python menjadi salah satu bahasa pemrograman yang populer digunakan oleh pengembangan $  $web, aplikasi $  $web, aplikasi perkantoran, simulasi, dan masih banyak lagi. $  $ Hal ini disebabkan karena Python bahasa pemrograman yang dinamis dan mudah dipahami. \\} \par
\noindent 
{\fontsize{14pt}{14pt}\selectfont Selain itu, sekarang telah tersedia berbagai situs kursus yang bagus untuk mempelajari bahasa pemrograman Python ini sehingga pembaca maupun developer pemula yang akan mempelajari bahasa ini akan menjadi lebih mudah karena dapat berlatih dimanapun dan kapanpun selama terhubung dengan Internet. \\} \par
\noindent 
{\fontsize{14pt}{14pt}\selectfont Menariknya, berbagai situs kursus gratis ini menawarkan metode pembelajaran yang interaktif sehingga mudah dimengerti oleh pesertanya. \\} \par
\vspace{14pt}
\noindent 
{\fontsize{14pt}{14pt}\selectfont Learn Python \\} \par
\noindent 
{\fontsize{14pt}{14pt}\selectfont Bagi pembaca yang tertarik untuk belajar bahasa pemrograman Python secara gratis, situs ini menjadi salah satu pilihan yang bijak. Dalam situs Learn Python ini pembaca akan dihadapkan pada halaman utama yang berisi penjelasan, tutorial, dan kolom pembelajaran interaktif. \\} \par
\noindent 
{\fontsize{14pt}{14pt}\selectfont Disini, pembaca dapat belajar bahasa pemrograman Python mulai dari dasar hingga tingkat lanjut dengan berbagai penjelasan serta tutorial dasar untuk memahami bahasa pemrograman Python. Developer dapat langsung memasukkan kode-kode latihan pada kolom pembelajaran interaktif yang nantinya dapat dijalankan untuk melihat apakah kode tersebut bisa berjalan atau terjadi kesalahan. \\} \par
\noindent 
{\fontsize{14pt}{14pt}\selectfont Selain itu, pembaca dapat juga mengetahui keinginan dari hasil-hasil kode latihan yang diberikan oleh Learn Python pada kolom pembelajaran interaktif. \\} \par
\vspace{14pt}
\noindent 
{\fontsize{14pt}{14pt}\selectfont Codecademy \\} \par
\noindent 
{\fontsize{14pt}{14pt}\selectfont Bisa dibilang ini salah satu situs yang menawarkan pembelajaran dan latihan untuk beberapa bahasa situs yang populer di dunia seperti Python, JavaScript, jQuery, Ruby, PHP, Python, HTML, dan CSS dengan tingkatan level yang disesuaikan. \\} \par
\noindent 
{\fontsize{14pt}{14pt}\selectfont Developer dapat mempelajar bahasa pemrograman Python di situs ini dengan interaktif dan baik. Nantinya developer akan diberikan halaman latihan dua kolom yang terdiri dari pengenalan pada kolom kiri dan latihan pada kolom kanan. \\} \par
\noindent 
{\fontsize{14pt}{14pt}\selectfont Pada kolom kanan ini developer dapat langsung mengetikan baris kode untuk pemrograman dan dapat langsung dijalankan secara langsung. Nantinya developer dapat melihat persentase mengenai tingkatan bahasa pemrograman yang telah dipelajar. \\} \par
\noindent 
{\fontsize{14pt}{14pt}\selectfont Treehouse \\} \par
\noindent 
{\fontsize{14pt}{14pt}\selectfont Treehouse merupakan salah satu situs yang mengajarkan beragam bahasa pemrograman mulai dari web hingga aplikasi mobile. Beberapa materi kursus yang ditawarkan oleh situs ini di antaranya Learn Python, Android Development, Web Designer, dan masih banyak lagi. \\} \par
\noindent 
{\fontsize{14pt}{14pt}\selectfont Dalam situs ini, pembaca maupun developer akan disuguhkan jalur latihan bahasa pemrograman Python mulai dari dasar hingga tingkat lanjut. Dengan menggunakan situs ini, developer dapat melakukan latihan pemrograman secara interaktif. \\} \par
\noindent 
{\fontsize{14pt}{14pt}\selectfont Nantinya developer akan meraih skor dari kemampuannya dalam menghadapi latihan yang ditawarkan Treehouse. \\} \par
\noindent 
{\fontsize{14pt}{14pt}\selectfont Trinket \\} \par
\noindent 
{\fontsize{14pt}{14pt}\selectfont Situs ini menghadirkan kursus bahasa pemrograman Python yang bisa dibilang lengkap.  $  $Pembaca maupun developer akan diberikan berbagai materi dan tutorial yang interaktif. Dalam halaman materi yang disampaikan akan terdapat kolom interaktif yang akan membuat developer dapat mempelajari Python dengan lebih mudah. \\} \par
\noindent 
{\fontsize{14pt}{14pt}\selectfont Untuk materi bahasa pemrograman Python yang disampaikan akan bertahap mulai dari dasar hingga tingkat lanjut. Developer juga dapat belajar bersama visual kura-kura yang akan disajikan pada kolom interaktif setiap developer menjalankan kode-kode yang telah diselesaikan. \\} \par
\noindent 
{\fontsize{14pt}{14pt}\selectfont Kura-kura tersebut akan bergerak yang disertai dengan hasil akhir dari kode-kode yang dijalankan. Bisa dibilang metode pembelajaran yang ditawarkan oleh situs Trinket ini menyenangkan. \\} \par
\vspace{14pt}
\noindent 
{\fontsize{14pt}{14pt}\selectfont Python Tutor \\} \par
\noindent 
{\fontsize{14pt}{14pt}\selectfont Dari namanya telah terlihat jelas bahwa situs ini membuka kursus bahasa pemrogramn Python bagi pembaca maupun developer. Hampir sama dengan situs lainnya, Pythontutor ini menawarkan beragam materi tentang bahasa pemrograman Python mulai dari dasar hingga tingkat lanjut. \\} \par
\noindent 
{\fontsize{14pt}{14pt}\selectfont Menariknya, dalam situs ini pembaca maupun developer akan diajarkan dan dijelaskan mengenai hasil kode-kode pada kolom yang disediakan. Selanjutnya, satu per satu setiap baris dari kode tersebut akan dijelaskan secara interaktif oleh situs ini. \\} \par
\noindent 
{\fontsize{14pt}{14pt}\selectfont Python adalah bahasa script tingkat tinggi, ditafsirkan, interaktif dan berorientasi objek. Python dirancang agar mudah dibaca. Ini menggunakan kata kunci bahasa Inggris sering di mana bahasa lainnya menggunakan tanda baca, dan memiliki konstruksi sintaksis lebih sedikit daripada bahasa lainnya. \\} \par
\vspace{14pt}
\noindent 
{\fontsize{14pt}{14pt}\selectfont Python diinterpretasikan: Python diproses pada saat runtime oleh interpreter. Anda tidak perlu mengkompilasi program Anda sebelum menjalankannya. Ini mirip dengan PERL dan PHP. \\} \par
\vspace{14pt}
\noindent 
{\fontsize{14pt}{14pt}\selectfont Python adalah Interaktif: Anda dapat benar-benar duduk dengan perintah Python dan berinteraksi dengan penerjemah secara langsung untuk menulis program Anda. \\} \par
\vspace{14pt}
\noindent 
{\fontsize{14pt}{14pt}\selectfont Python adalah Object-Oriented: Python mendukung gaya Berorientasi Objek atau teknik pemrograman yang mengenkapsulasi kode di dalam objek. \\} \par
\vspace{14pt}
\noindent 
{\fontsize{14pt}{14pt}\selectfont Python adalah bahasa Pemula: Python adalah bahasa yang bagus untuk para pemrogram tingkat pemula dan mendukung pengembangan berbagai aplikasi mulai dari pemrosesan teks sederhana hingga browser WWW hingga game. \\} \par
\vspace{14pt}
\noindent 
{\fontsize{14pt}{14pt}\selectfont Sejarah Python \\} \par
\noindent 
{\fontsize{14pt}{14pt}\selectfont Python dikembangkan oleh Guido van Rossum pada akhir tahun delapan puluhan dan awal tahun sembilan puluhan di National Research Institute for Mathematics and Computer Science di Belanda. \\} \par
\vspace{14pt}
\noindent 
{\fontsize{14pt}{14pt}\selectfont Python berasal dari banyak bahasa lain, termasuk ABC, Modula-3, C, C ++, Algol-68, SmallTalk, dan shell Unix dan bahasa script lainnya. \\} \par
\vspace{14pt}
\noindent 
{\fontsize{14pt}{14pt}\selectfont Python memiliki hak cipta. Seperti Perl, kode sumber Python sekarang tersedia di bawah GNU General Public License (GPL). \\} \par
\vspace{14pt}
\noindent 
{\fontsize{14pt}{14pt}\selectfont Python sekarang dikelola oleh tim pengembangan inti di institut tersebut, walaupun Guido van Rossum masih memegang peran penting dalam mengarahkan kemajuannya. \\} \par
\vspace{14pt}
\noindent 
{\fontsize{14pt}{14pt}\selectfont Fitur Python \\} \par
\noindent 
{\fontsize{14pt}{14pt}\selectfont Fitur Python meliputi: \\} \par
\vspace{14pt}
\noindent 
{\fontsize{14pt}{14pt}\selectfont Mudah dipelajari: Python memiliki beberapa kata kunci, struktur sederhana, dan sintaks yang jelas. Hal ini memungkinkan siswa untuk mengambil bahasa dengan cepat. \\} \par
\vspace{14pt}
\noindent 
{\fontsize{14pt}{14pt}\selectfont Mudah dibaca: kode Python lebih jelas dan terlihat oleh mata. \\} \par
\vspace{14pt}
\noindent 
{\fontsize{14pt}{14pt}\selectfont Mudah dipelihara: kode sumber Python cukup mudah untuk dipelihara. \\} \par
\vspace{14pt}
\noindent 
{\fontsize{14pt}{14pt}\selectfont Perpustakaan standar yang luas: sebagian besar perpustakaan Python sangat portabel dan kompatibel dengan platform cross-platform di UNIX, Windows, dan Macintosh. \\} \par
\noindent 
{\fontsize{14pt}{14pt}\selectfont Mode Interaktif: Python memiliki dukungan untuk mode interaktif yang memungkinkan pengujian interaktif dan debugging dari cuplikan kode. \\} \par
\vspace{14pt}
\noindent 
{\fontsize{14pt}{14pt}\selectfont Portable: Python dapat berjalan di berbagai platform perangkat keras dan memiliki antarmuka yang sama pada semua platform. \\} \par
\vspace{14pt}
\noindent 
{\fontsize{14pt}{14pt}\selectfont Dapat diperpanjang: Anda dapat menambahkan modul tingkat rendah ke penerjemah Python. Modul ini memungkinkan programmer untuk menambahkan atau menyesuaikan alat mereka agar lebih efisien. \\} \par
\vspace{14pt}
\noindent 
{\fontsize{14pt}{14pt}\selectfont Database: Python menyediakan antarmuka untuk semua database komersial utama. \\} \par
\vspace{14pt}
\vspace{14pt}
\noindent 
{\fontsize{14pt}{14pt}\selectfont Pengenal Python Pengenal Python adalah nama yang digunakan untuk mengidentifikasi variabel, fungsi, kelas, modul atau objek lainnya. Pengenal dimulai dengan huruf A sampai Z atau huruf a sampai z atau garis bawah ( $  \_  $) diikuti oleh nol atau lebih huruf, garis bawah dan angka (0 sampai 9). Python tidak mengizinkan karakter tanda baca seperti @,  $  \$  $, dan $  \%  $ dalam pengenal. Python adalah bahasa pemrograman yang sensitif. Dengan demikian, Tenaga Kerja dan Tenaga Kerja adalah dua pengidentifikasi yang berbeda dengan Python. Berikut adalah konvensi penamaan untuk pengenal Python - Nama kelas dimulai dengan huruf besar. Semua pengenal lainnya mulai dengan huruf kecil. Memulai pengenal dengan satu garis bawah terkemuka menunjukkan bahwa pengenal bersifat pribadi. Memulai pengenal dengan dua garis bawah terkemuka menunjukkan pengenal yang sangat pribadi. Jika pengenal juga diakhiri dengan dua tanda garis bawah, identifier adalah nama khusus yang ditentukan bahasa. \\} \par
\vspace{14pt}
\noindent 
{\fontsize{14pt}{14pt}\selectfont \vspace{\baselineskip}
Bahasa Python memiliki banyak kesamaan dengan Perl, C, dan Java. Namun, ada beberapa perbedaan yang pasti antara bahasa. Program Python Pertama Mari kita jalankan program dalam mode pemrograman yang berbeda. Pemrograman Mode Interaktif Memohon interpreter tanpa melewatkan file script sebagai parameter menampilkan prompt berikut  \\} \par
\vspace{14pt}
\noindent 
{\fontsize{14pt}{14pt}\selectfont Python \\} \par
\vspace{14pt}
\noindent 
{\fontsize{14pt}{14pt}\selectfont Python 2.4.3 ( $  \#  $1, Nov 11 2010, 13:34:43) \\} \par
\noindent 
{\fontsize{14pt}{14pt}\selectfont [GCC 4.1.2 20080704 (Red Hat 4.1.2-48)] on linux2 \\} \par
\noindent 
{\fontsize{14pt}{14pt}\selectfont Type "help", "copyright", "credits" or "license" for more information. \\} \par
\vspace{14pt}
\noindent 
{\fontsize{14pt}{14pt}\selectfont Ketik teks berikut pada prompt Python dan tekan Enter: \\} \par
\vspace{14pt}
\noindent 
{\fontsize{14pt}{14pt}\selectfont print "Hello, Python!" \\} \par
\vspace{14pt}
\noindent 
{\fontsize{14pt}{14pt}\selectfont Jika Anda menjalankan versi baru Python, Anda perlu menggunakan pernyataan cetak dengan tanda kurung seperti pada cetak ("Halo, Python!") ;. Namun dengan versi Python 2.4.3, ini menghasilkan hasil sebagai berikut: \\} \par
\vspace{14pt}
\noindent 
{\fontsize{14pt}{14pt}\selectfont Hello, Pyhton! \\} \par
\vspace{14pt}
\noindent 
{\fontsize{14pt}{14pt}\selectfont Pemrograman Mode Script \\} \par
\noindent 
{\fontsize{14pt}{14pt}\selectfont Memohon interpreter dengan parameter script memulai eksekusi script dan berlanjut sampai script selesai. Saat skrip selesai, juru bahasa tidak lagi aktif. \\} \par
\vspace{14pt}
\noindent 
{\fontsize{14pt}{14pt}\selectfont Mari kita tuliskan program Python sederhana dalam sebuah naskah. File Python memiliki ekstensi .py. Ketik kode sumber berikut di file \\} \par
\vspace{14pt}
\noindent 
{\fontsize{14pt}{14pt}\selectfont Objek Dengan Python, seperti semua bahasa berorientasi objek, ada kumpulan kode dan data yang disebut objek, yang biasanya mewakili potongan dalam model konseptual suatu sistem. \\} \par
\vspace{14pt}
\noindent 
{\fontsize{14pt}{14pt}\selectfont Objek dengan Python dibuat (yaitu, instantiated) dari template yang disebut kelas (yang akan dibahas kemudian, sebanyak bahasa dapat digunakan tanpa memahami kelas). Mereka memiliki atribut, yang mewakili berbagai potongan kode dan data yang membentuk objek. Untuk mengakses atribut, seseorang menuliskan nama objek yang diikuti oleh suatu periode (selanjutnya disebut titik), diikuti dengan nama atribut. \\} \par
\vspace{14pt}
\noindent 
{\fontsize{14pt}{14pt}\selectfont Contohnya adalah atribut 'atas' dari string, yang mengacu pada kode yang mengembalikan salinan string di mana semua huruf adalah huruf besar. Untuk mendapatkan ini, perlu untuk memiliki cara untuk merujuk ke objek (dalam contoh berikut, jalan adalah string literal yang membangun objek). \\} \par
\vspace{14pt}
\noindent 
{\fontsize{14pt}{14pt}\selectfont Interaktif: Anda dapat benar-benar duduk dengan perintah Python dan berinteraksi dengan penerjemah secara langsung untuk menulis program Anda. \\} \par
\vspace{14pt}
\noindent 
{\fontsize{14pt}{14pt}\selectfont Python adalah Object-Oriented: Python mendukung gaya Berorientasi Objek atau teknik pemrograman yang mengenkapsulasi kode di dalam objek. \\} \par
\vspace{14pt}
\noindent 
{\fontsize{14pt}{14pt}\selectfont Python adalah bahasa Pemula: Python adalah bahasa yang bagus untuk para pemrogram tingkat pemula dan mendukung pengembangan berbagai aplikasi mulai dari pemrosesan teks \\} \par
\vspace{14pt}
\noindent 
{\fontsize{14pt}{14pt}\selectfont Mudah dipelajari: Python memiliki beberapa kata kunci, struktur sederhana, dan sintaks yang jelas. Hal ini memungkinkan siswa untuk mengambil bahasa dengan cepat. \\} \par
\vspace{14pt}
\noindent 
{\fontsize{14pt}{14pt}\selectfont Mudah dibaca: kode Python lebih jelas dan terlihat oleh mata. \\} \par
\vspace{14pt}
\noindent 
{\fontsize{14pt}{14pt}\selectfont Mudah dipelihara: kode sumber Python cukup mudah untuk dipelihara. \\} \par
\vspace{14pt}
\noindent 
{\fontsize{14pt}{14pt}\selectfont Perpustakaan standar yang luas: sebagian besar perpustakaan Python sangat portabel dan kompatibel dengan platform cross-platform di UNIX, Windows, dan Macintosh. \\} \par
\noindent 
{\fontsize{14pt}{14pt}\selectfont Mode Interaktif: Python memiliki dukungan untuk mode interaktif yang memungkinkan pengujian interaktif dan debugging dari cuplikan kode. \\} \par
\vspace{14pt}
\vspace{14pt}
\noindent 
{\fontsize{14pt}{14pt}\selectfont Scalable: Python menyediakan struktur dan dukungan yang lebih baik untuk program besar daripada skrip shell. \\} \par
\vspace{14pt}
\noindent 
{\fontsize{14pt}{14pt}\selectfont Paradigma : Multi-paradigm: object-oriented, imperative, functional, procedural, reflective\vspace{\baselineskip}
Muncul Tahun : 1991\vspace{\baselineskip}
Perancang : Guido van Rossum\vspace{\baselineskip}
Pengembang : Python Software Foundation\vspace{\baselineskip}
Rilis terbaru : 3.2.3 / 11 April 2012; 46 hari lalu 2.7.3 / 11 April 2012; 46 hari lalu /\vspace{\baselineskip}
Sistem pengetikan: duck, dynamic, strong\vspace{\baselineskip}
Implementasi : CPython, IronPython, Jython, Python for S60, PyPy\vspace{\baselineskip}
Dialek : Cython, RPython, Stackless Python\vspace{\baselineskip}
Terpengaruh oleh : ABC,ALGOL 68,C,C++,Dylan Haskell,Icon,Java,Lisp,Modula-3,Perl\vspace{\baselineskip}
Mempengaruhi : Boo, Cobra, D, Falcon, Groovy, JavaScript, Ruby\vspace{\baselineskip}
Sistem operasi : Cross-platform\vspace{\baselineskip}
Lisensi : Python Software Foundation License,GNU GPL\vspace{\baselineskip}
Situs web : python.org \\} \par
\noindent 
{\fontsize{14pt}{14pt}\selectfont Apakah itu Python ?\vspace{\baselineskip}
Python adalah sebuah bahasa pemrograman dinamik yang telah banyak\vspace{\baselineskip}
digunakan diseluruh dunia. Pembuat aslinya Guido Van Rossum senang sekali dengan\vspace{\baselineskip}
acara televisi Monty Python Flying Circus dan dari judul acara tersebut lah Guido\vspace{\baselineskip}
memberi nama bahasa ciptaannya itu. Python merupakan kelanjutan dari bahasa\vspace{\baselineskip}
pemrograman ABC, guido merupakan salah satu pengembang bahasa ini (ABC).\vspace{\baselineskip}
Tahun 1995, Guido pindah ke CNRI sambil terus melanjutkan pengembangan\vspace{\baselineskip}
Python. Versi terakhir yang dikeluarkan adalah 1.6. Tahun 2000, Guido dan para\vspace{\baselineskip}
pengembang inti Python pindah ke BeOpen.com yang merupakan sebuah perusahaan\vspace{\baselineskip}
komersial dan membentuk BeOpen PythonLabs. Python 2.0 dikeluarkan oleh BeOpen.\vspace{\baselineskip}
Setelah mengeluarkan Python 2.0, Guido dan beberapa anggota tim PythonLabs\vspace{\baselineskip}
pindah ke DigitalCreations.\vspace{\baselineskip}
Saat ini pengembangan Python terus dilakukan oleh sekumpulan pemrogram\vspace{\baselineskip}
yang dikoordinir Guido dan Python Software Fondation. Python Software Fondation\vspace{\baselineskip}
adalah sebuah organisasi non-profit yang dibentuk sebagai pemegang hak cipta\vspace{\baselineskip}
intelektual Python sejak versi 2.1 dan dengan demikian mencegah Python dimiliki oleh\vspace{\baselineskip}
perusahaan komersial. Saat ini distribusi Python sudah mencapai versi 2.7 dan versi 3.2. \\} \par
\noindent 
{\fontsize{14pt}{14pt}\selectfont Python mendukung multi paradigma pemrograman, utamanya namun tidak\vspace{\baselineskip}
dibatasi pada pemrograman berorientasi objek, pemrograman imperatif, dan\vspace{\baselineskip}
pemrograman fungsional. Salah satu fitur yang tersedia pada python adalah sebagai\vspace{\baselineskip}
bahasa pemrograman dinamis yang dilengkapi dengan manajemen memori otomatis.\vspace{\baselineskip}
Seperti halnya pada bahasa pemrograman dinamis lainnya, pyhton umumnya\vspace{\baselineskip}
digunakan sebagai bahasa skrip meski pada prakteknya penggunaan bahasa ini lebih\vspace{\baselineskip}
luas mencakup konteks pemanfaatan yang umumnya tidak dilakungan dengan\vspace{\baselineskip}
menggunakan bahasa skrip. Python dapat digunakan untuk berbagai keperluan\vspace{\baselineskip}
pengembangan perangkat lunak dan dapat berjalan di berbagai platform sistem\vspace{\baselineskip}
operasi.\vspace{\baselineskip}
Penggunaan python sangat luas saat ini, bahkan NASA dan Google sangat\vspace{\baselineskip}
bergantung pada bahasa pemrograman yang satu ini. Anda dapat menemukan python\vspace{\baselineskip}
dimana mulai dari web, aplikasi mobile, desktop sampai embeded device\vspace{\baselineskip}
menggunakan python. \\} \par
\noindent 
{\fontsize{14pt}{14pt}\selectfont Fitur-Fitur Python\vspace{\baselineskip}
Python memiliki beberapa fitur yang menjadikan bahasa pemrograman ini\vspace{\baselineskip}
berbeda dari bahasa lain antara lain : \\} \par
\noindent 
{\fontsize{14pt}{14pt}\selectfont  Memiliki kepustakaan yang luas, dalam distribusi Python telah disediakan modul-\vspace{\baselineskip}
modul ‘siap pakai’ untuk berbagai keperluan. \\} \par
\noindent 
{\fontsize{14pt}{14pt}\selectfont Memiliki tata bahasa yang jernih dan mudah dipelajari. \\} \par
\noindent 
{\fontsize{14pt}{14pt}\selectfont Memiliki aturan layout kode sumber yang memudahkan pengecekan, pembacaan\vspace{\baselineskip}
kembali dan penulisan ulang kode sumber. \\} \par
\noindent 
{\fontsize{14pt}{14pt}\selectfont Berorientasi obyek. \\} \par
\noindent 
{\fontsize{14pt}{14pt}\selectfont Memiliki sistem pengelolaan memori otomatis (garbage collection, seperti java)\vspace{\baselineskip}
Modular, mudah dikembangkan dengan menciptakan modul-modul baru; modul-\vspace{\baselineskip}
modul tersebut dapat dibangun dengan bahasa Python maupun C/C++. \\} \par
\noindent 
{\fontsize{14pt}{14pt}\selectfont Memiliki fasilitas pengumpulan sampah otomatis, seperti halnya pada bahasa\vspace{\baselineskip}
pemrograman Java, python memiliki fasilitas pengaturan penggunaan memory\vspace{\baselineskip}
komputer sehingga para pemrogram tidak perlu melakukan pengaturan memory\vspace{\baselineskip}
komputer secara langsung. \\} \par
\noindent 
{\fontsize{14pt}{14pt}\selectfont Memiliki banyak faslitas pendukung sehingga mudah dalam pengoprasiannya. \\} \par
\vspace{14pt}
\vspace{14pt}
\vspace{14pt}
\vspace{12pt}
