
\sloppy

\begin{figure}[ht]
	\centerline{\includegraphics[width=0.70\textwidth]{figures/python}}
	\caption{Logo}
	\label{Logo} 
\end{figure}

\section{python}

Python adalah bahasa pemrograman interpretatif multiguna. tidak seperti bahasa lainnya yang susah untuk dibaca dan dipahami, python lebih menekankan pada keterbacaan kode agar lebih mudah untuk memahami sintaks. hal ini membuat python sangat mudah dipelajari baik untuk pemula maupun untuk yang sudah menguasai bahasa pemrograman lain.

Bahasa Python mendukung hampir semua sistem operasi, bahkan untuk sistem operasi Linux, hampir semua distronya sudah menyertakan Python didalamnya. Dengan kode yang simpel dan mudah diimplementasikan, seorang programmer dapat lebih mengutamakan pengembangan aplikasi yang dibuat, bukan malah sibuk mencari syntax error.

\begin{verbatim}
print("Python sangat simpel")
\end{verbatim}

Hanya dengan menuliskan kode \textit{print} seperti yang diatas, anda sudah bisa mencetak apapun yang anda inginkan didalam tanda kurung (). Dibagian akhir kode pun anda tidak harus mengakhirinya dengan tanda semicolon ;.

Python dapat digunakan secara bebas, bahkan untuk kepentingan komersial sekalipun. Python diklaim mampu memberikan kecepatan dan kualitas untuk membangun aplikasi bertingkat \textit{Rapid Application Development}. Hal ini didukung oleh adanya \textit{library} dengan modul-modul baik standar maupun tambahan misalnya NumPy, SciPy, dan lain-lain.


\begin{figure}[!htbp]
\centerline{\includegraphics[width=0.70\textwidth]{figures/tiobe.PNG}}
\caption{Tiobe Index}
\label{tiobe}
\end{figure}

Menurut \href{https://www.tiobe.com/tiobe-index/}{TIOBE}, python merupakan bahasa pemrograman terpopuler di tahun 2018. Python juga memiliki sintaks atau aturan penulisan kode pemrogramannya sendiri. Pada bagian \textit{home} kita akan mempelajari pengantar untuk mempelajari python. Sebelum berlanjut ketahapan yang baru, pembaca memerlukan pemahaman yang lain seperti \textit{environtment setup}, sintaks dan lain sebagainya. Yang pertama penulis jelaskan yaitu pengertian tentang \textit{class} pada python untuk mengarahkan logika dan pengetahuan tentang apa itu \textit{class}.

\textit{Class} merupakan suatu kelas yang di dalamnya mempunyai suatu \textit{method} yang sesuai dengan fungsinya. Contoh \textit{class} pada kehidupan nyata seperti kelas belajar yang diibaratkan sebagai \textit{class}-nya dan isi dari kelas itu diantaranya seperti: bangku, spidol, dan lain lain. Isi dari kelas tersebut merupakan metode dari \textit{class} dan \textit{method} itu berfungsi seperti fungsi metode itu sendiri. Contohnya seperti spidol yang berfungsi untuk menulis.
\subsection{Pembuatan class pada python}
Kita awali dengan sebuah kata kunci. Yaitu  ``class`` yang kemudian diikuti dengan ``nama class nya`` dan terakhir membuat kurung buka dan tutup serta membuat tanda titik dua  ``()`` dan ``:``jika sintaksnya seperti ini.


Pemrograman python adalah bahasa pemrograman terpopuler di tahun 2016 menurut tiobe. Python juga memiliki sintaks atau aturan penulisan \textit{code} pemrograman. Salah satu bagian Home merupakan halaman pengantar untuk mempelajari python. Sebelum ketahapan yang baru selain home ini pembaca memerlukan pengertian yang lain yaitu seperti enverinmoment setup, syntax dan lain lain, awal untuk penulis jelakan yaitu pengertian tentang \textit{class} pada python untuk mengantarkan logika dan pengetahuan apa itu \textit{class}.

\subsection{Pembuatan class pada python}
Kita awali dengan sebuah kata kunci. Yaitu ``class`` yang kemudian diikuti dengan ``nama class nya`` terakhir membuat kurung buka dan tutup serta membuat tanda titik dua  ``()`` dan ``:`` kalo synax nya, seperti :


\subsection{Beberapa aplikasi yang menggunakan Python}
Python dapat digunakan diberbagai bidang untuk membangun aplikasi-aplikasi yang berjalan pada banyak fungsi. Diantaranya sebagai berikut :
\begin{enumerate}
\item Website dan Internet. dimana python dapat digunakan sebagai server side yang diintegrasikan dengan berbagai internet protokol misalnya HTML, JSON, Email Processing, FTP, dan IMAP. Selain itu, python juga mempunyai library untuk pengembangan internet.
\item Penelitian Ilmiah dan Numerik, dimana python dapat digunakan untuk melakukan riset ilmiah untuk mempermudah perhitungan numerik. Misalnya penerapan algoritma KNN, Naive Bayes, Decision Tree, dan lain-lain.
\item Data Science dan Big Data, dimana python dapat memungkinkan untuk melakukan analisis data dari database big data.
\item Media Pembelajaran Pemrograman.
\item Graphical User Interface (GUI).
\item Pengembangan Software.
\item Aplikasi Bisnis.
\end{enumerate}

Bahasa pemrograman python adalah bahasa pemrograman terpopuler di tahun 2016 menurut tiobe\cite{https://www.tiobe.com/tiobe-index/}. Python juga memiliki sintak atau aturan penulisan \textit{code} pemrograman. Sintak elegan dan \textit{dynamic typing} yang dimiliki oleh python, bersama dengan \textit{interpreted nature} dari python, menjadikannya bahasa pemrograman yang ideal untuk melakukan \textit{'scripting'} dan pengembangan aplikasi yang sangat pesat dalam area pada kebanyakan platform.

Pada bagian \textit{home} kita akan mempelajari pengantar untuk mempelajari python. Sebelum berlanjut ketahapan yang baru, pembaca memerlukan pemahaman yang lain seperti \textit{environtment setup}, sintaks dan lain sebagainya. Yang pertama penulis jelaskan yaitu pengertian tentang \textit{class} pada python untuk mengarahkan logika dan pengetahuan tentang apa itu \textit{class}.

\subsection{Pembuatan class pada python}
\textit{Class} itu merupakan suatu kelas yang didalamnya mempunyai \textit{method} yang sesuai dengan fungsinya. Contoh dikehidupan nyata yaitu; seperti kelas tempat belajar yang diibaratkan sebagai \textit{class}-nya, dan adapun isi dari kelas itu seperti bangku, sepidol, dan lain-lain, isi dari kelas itu merupakan metode dari \textit{class} dan \textit{method} itu berfungsi seperti fungsinya sendiri seperti spidol yang berfungsi untuk menulis, masing-masing dari benda tersebut disesuaikan dengan fungsi.

\textit{Class} merupakan sebuah objek yang didalamnya terdapat beberapa \textit{method} yang juga merupakan isi dari subuah \textit{class}. \textit{Class} dan \textit{method} juga disebut sebagai OOP \textit{(Object Oriented programing)}. OOP mempunyai fungsi yang dapat memudahkan suatu proses atau kegiatan \textit{programing} yang kita lakukan.

Jadi, untuk membuat sebuah \textit{class} kita awali dengan sebuah kata kunci. Yaitu ``class`` yang kemudian diikuti dengan ``nama class nya`` terakhir membuat kurung buka dan tutup serta membuat tanda titik dua  ``()`` dan ``:`` kalo synax nya, seperti :


\verb|namaClass()|

untuk memanggil metode kita cukup menggunakan kata kunci "class" yang kemudian diikuti dengan memanggil nama metode yang tersedia didalam class tersebut dengan dipisahkan oleh tanda titik seperti:

\begin{verbatim}
namaClsass().namaMetode()
\end{verbatim}

ingin lebih mudah kita tampung classnya dulu ke variable 'penampung = namaClass()'.

\begin{verbatim}
penampung.namaMetode()
\end{verbatim}

Didalam sebuah class yang dibuat biasanya terdapat init itu disediakan langsung oleh pythonnya, seperti :


\begin{verbatim}

 class namaClass():
def init (self,parameter):
itu code program yang pertama kali kalian buat
def metode 1 (self,parameter):
isi metode
def metode 2 (self):
 isi metode

 \end{verbatim}

Setelah menjelaskan \textit{class} kita akan menjelaskan variable seperti tadi penampung \textit{class}. \textit{Variable} bisa diartikan sebagai huruf atau kata, tujuannya untuk mempermudah proses penulisan sebuah program. Contohnya dalam kehidupan nyata seperti gelas atau ember, ambil contoh ember kita ketahui bahwa ember bisa diisi dengan air, pasir, tanah dan lain-lain, ember itu sebagai variable dan isi variablenya itu adalah air, tanah, pasir dan lain-lain.
Contoh variablenya seperti ini :

\begin{verbatim}
 Variable=  "ini string atau teks"
\end{verbatim}

\begin{verbatim}
Print( "nilai isi dari variable1 adalah: ",variable1)
Print( "nilai atau isi dari variable2 adalah:",variable2)
\end{verbatim}

Ada beberapa hal yang harus diketahui seperti input, data operation dan lain-lain seperti:

\begin{itemize}
	\item input
	\item data
	\item opration
	\item output
	\item condutional
	\item looping
	\item subroutine
\end{itemize}

Tahapan ini merupakan proses memasukan data kedalam proses komputer melalui peralatan input. Pada bahasa Python, untuk menerima masukan dari pengguna yaitu dengan menggunakan method 'input()' dan 'raw $_$input()'.
Data adalah bahan mentah yang akan diolah menjadi informasi sehingga dapat berguna dan dimanfaatkan oleh pengguna. Data dapat berupa variabel, konstanta, atau yang berisi bilangan, kalimat, dan lainnya. Tipe data berupa string, number, list, tuple, dan lainnya.

\subsection {Operation}
Operation adalah yang akan mengubah suatu nilai menjadi nilai lain. Yang termasuk operation atau yang biasa disebut dengan operator adalah operator aritmatika, operator assignment, dan lainnya.

\subsection {Output}
Output adalah menuliskan informasi yang ditampilkan dilayar, disk, atau ke salah satu unit I/O. Pada Python 2.0, untuk menampilkan output dengan menulis sintax print. Sedangkan pada Python 3.0 dengan menggunakan fungsi print().

\subsection {Conditional}
Conditional merupakan jumlah perintah yang akan dijalankan jika kondisi tertentu sudah terpenuhi. Jika \textit{username} dan \textit{password} yang dimasukan benar, maka akan menampilkan halaman utama. Hal ini bisa disebut \textit{conditional}. Pada \textit{conditional}, Python menggunakan pernyataan if, else, dan elif.

\subsection {Looping}
Perintah yang akan berjalan beberapa kali, selama kondisi yang ditentukan atau kondisi yang terpenuhi. Pada \textit{looping} ini, Python menggunakan pernyataan \textit{for} dan \textit{while} untuk melakukan perulangan.

\subsection {Subroutine}
Perintah yang bisa dijalankan dengan cara memanggil namanya. Sering disebut sebagai functionatau method. Pada bahasa pemrograman Python, untuk menggunakan function atau method yaitu dengan menggunakan pernyatan def nama _
$function().
Fungsi def dalam python,
Penggunaan fungsi tanpa parameter
Command=fungsi()
Deklarasi command= def fungsi()
Pemanggilan fungsi, parameter sesuai dengan kata kunci seperti tadi class
Command= fungsi(arg=1, arg2=2)
Deklarasi command – def fungsi (arg2,arg2)
Pemanggilan fungsi, parameter sesuai dengan posisi
Command= fungsi()
Deklarasi command – def fungsi (x)  Pemanggilan fungsi parameter sesuai dengan argument posisional tuple
Command= fungsi((1,2).(1,3))
Deklarasi command – def fungsi (*args)
Pemanggilan fungsi, parameters ssesuai argument kata kunci dictionary
Command= fungsi (bahasa= ‘python’,versi=’2.2’)
Deklarasi command = def fungsi (**args)
Itu adalah cara memanggil dalam code python pemrograman jadi ada nenerapa fungsi yang dibutuhkan penulisan dengan tepat maka sebab itu dengan sebab itu buaat lah penulisan yang mudah. Karena pemrograma python sangan sensitive bila ada kesalahan sedikit di penulisan atau symbol yang tertinggal.

\subsection {Sejarah}
Bahasa pemrograman Python adalah bahasa yang dibuat oleh seorang keturunan Belanda yaitu Guido van Rossum. Sampai saat ini Python masih dikembangkan oleh \textit{Python Software Foundation}. Awalnya, pembuatan bahasa pemrograman ini adalah untuk membuat skrip bahasa tingkat tinggi pada sebuah sistem operasi yang terdistribusi Amoeba. Python telah digunakan oleh beberapa pengembang dan bahkan digunakan oleh beberapa perusahaan untuk pembuatan perangkat lunak komersial. Pemrograman bahasa python ini adalah pemrogram gratis atau \textit{freeware}, sehingga dapat dikembangkan, dan tidak ada batasan dalam penyalinannya dan mendistribusikan.

\subsection{Dukungan Komunitas yang Aktif}
Python adalah salah satu pemrograman yang terus berkembang dan bertahan dikarenakan dukungan komunitas yang aktif diseluruh dunia. Banyak forum-forum ataupun blogger-blogger yang sering membagi pengalaman dalam menggunakan python. Hal ini memudahkan bagi pengguna pemula maupun pengembang untuk bertanya dan sharing tentang ilmu pemrograman ini.

\subsection{Kelebihan dan Kekurangan}
Kelebihan :
\begin{enumerate}
\item Tidak ada tahapan kompilasi dan penyambungan (link) sehingga kecepatan perubahan pada masa pembuatan sistem aplikasi meningkat.
\item Tidak ada deklarasi tipe data yang merumitkan sehingga program menjadi lebih sederhana, singkat, dan fleksible.
\item Manajemen memori otomatis yaitu kumpulan sampah memori sehingga dapat menghindari pencacatan kode.
\item Tipe data dan operasi tingkat tinggi yaitu kecepatan pembuatan sistem aplikasi menggunakan tipe objek yang telah ada.
\item Pemrograman berorientasi objek.
\item Pelekatan dan perluasan dalam C.
\item Terdapat kelas, modul, eksepsi sehingga terdapat dukungan pemrograman skala besar secara modular.
\item Pemuatan dinamis modul C sehingga ekstensi menjadi sederhana dan berkas biner yang kecil
\item Pemuatan kembali secara dinamis modul phyton seperti memodifikasi aplikasi tanpa menghentikannya.
\item Model objek universal kelas Satu.
\item Konstruksi pada saat aplikasi berjalan.
\item Interaktif, dinamis dan alamiah.
\item Akses hingga informasi interpreter.
\item Portabilitas secara luas seperti pemrograman antar platform tanpa ports
\item Kompilasi untuk portable kode byte sehingga kecepatan eksekusi bertambah dan melindungi kode sumber.
\item Antarmuka terpasang untuk pelayanan keluar seperti perangkat Bantu \textit{system}, GUI, persistence, database, dll.
\end{enumerate}
Kekurangan:
\begin{enumerate}
\item Beberapa penugasan terdapat diluar dari jangkauan python, seperti bahasa pemrograman dinamis lainnya, python tidak secepat atau efisien sebagai statis, tidak seperti bahasa pemrograman kompilasi seperti bahasa C.
\item Disebabkan python merupakan interpreter, python bukan merupakan perangkat bantu terbaik untuk pengantar komponen performa kritis.
\item Python tidak dapat digunakan sebagai dasar bahasa pemrograman implementasi untuk beberapa komponen, tetapi dapat bekerja dengan baik sebagai bagian depan skrip antarmuka untuk mereka.
\item Python memberikan efisiensi dan fleksibilitas tradeoff by dengan tidak memberikannya secara menyeluruh. Python menyediakan bahasa pemrograman optimasi untuk kegunaan, bersama dengan perangkat bantu yang dibutuhkan untuk diintegrasikan dengan bahasa pemrograman lainnya.
\item Banyak terdapat referensi lama terutama dari pencarian google, python adalah pemrograman yang sangat lambat. Namun belum lama ini ditemukan bahwa Google, Youtube, DropBox dan beberapa software sistem banyak menggunakan Python.
\item Kini Python menjadi salah satu bahasa pemrograman yang populer digunakan oleh pengembangan web, aplikasi web, aplikasi perkantoran, simulasi, dan masih banyak lagi. Hal ini disebabkan karena Python bahasa pemrograman yang dinamis dan mudah dipahami.
\item Selain itu, sekarang telah tersedia berbagai situs kursus yang bagus untuk mempelajari bahasa pemrograman Python ini sehingga pembaca maupun developer pemula yang akan mempelajari bahasa ini akan menjadi lebih mudah karena dapat berlatih dimanapun dan kapanpun selama terhubung dengan Internet.
\item Menariknya, berbagai situs kursus gratis ini menawarkan metode pembelajaran yang interaktif sehingga mudah dimengerti oleh pesertanya.
\end{enumerate}

Python merupakan pemograman yang tidak pernah dicompile secara full. Jika kamu sudah menyelesaikan programnya dan kamu ingin mengirim ke teman atau dibagikan ke internet maka teman atau orang lain dapat mengubah kode diprogram kamu karena program dibuka dinotepad, python akan tetap berbentuk kode yang sama tidak acak acakan sehingga orang lain dapat memahami pemograman yang kamu buat.

$>$$>$$>$ Python 2.4.3 ( $  \#  $1, Nov 11 2010, 13:34:43) [GCC 4.1.2 20080704 (Red Hat 4.1.2-48)] on linux2 Type "help", "copyright", "credits" or "license" for more information.

Ketik teks berikut pada prompt Python dan lalu tekan Enter:

\begin{verbatim}
print ``Hello, Python!``
\end{verbatim}

Jika Anda menjalankan versi baru Python, Anda perlu menggunakan pernyataan cetak dengan tanda kurung seperti pada cetak (``Halo, Python!``). Namun dengan versi Python 2.4.3, ini menghasilkan hasil sebagai berikut:

Hello, Pyhton!

\subsection{Pemrograman Mode Script}
Memohon interpreter dengan parameter script memulai eksekusi script dan berlanjut sampai script selesai. Saat skrip selesai, juru bahasa tidak lagi aktif. Mari kita tuliskan program Python sederhana dalam sebuah naskah. File Python memiliki ekstensi .py. Ketik kode sumber berikut di file.

Objek Dengan Python, seperti semua bahasa berorientasi objek, ada kumpulan kode dan data yang disebut objek, yang biasanya mewakili potongan dalam model konseptual suatu sistem. Objek dengan Python dibuat (yaitu, instantiated) dari template yang disebut kelas (yang akan dibahas kemudian, sebanyak bahasa dapat digunakan tanpa memahami kelas). Mereka memiliki atribut, yang mewakili berbagai potongan kode dan data yang membentuk objek. Untuk mengakses atribut, seseorang menuliskan nama objek yang diikuti oleh suatu periode (selanjutnya disebut titik), diikuti dengan nama atribut.

Contohnya adalah atribut 'atas' dari string, yang mengacu pada kode yang mengembalikan salinan string di mana semua huruf adalah huruf besar. Untuk mendapatkan ini, perlu untuk memiliki cara untuk merujuk ke objek (dalam contoh berikut, jalan adalah string literal yang membangun objek).

Paradigma : Multi-paradigm: object-oriented, imperative, functional, procedural, reflective
Muncul Tahun : 1991
Perancang : Guido van Rossum
Pengembang : Python Software Foundation
Rilis terbaru : 3.2.3 / 11 April 2012; 46 hari lalu 2.7.3 / 11 April 2012; 46 hari lalu /
Sistem pengetikan: duck, dynamic, strong
Implementasi : CPython, IronPython, Jython, Python for S60, PyPy
Dialek : Cython, RPython, Stackless Python
Terpengaruh oleh : ABC,ALGOL 68,C,C++,Dylan Haskell,Icon,Java,Lisp,Modula-3,Perl
Mempengaruhi : Boo, Cobra, D, Falcon, Groovy, JavaScript, Ruby
Sistem operasi : Cross-platform
Lisensi : Python Software Foundation License,GNU GPL
Situs web : python.org 
