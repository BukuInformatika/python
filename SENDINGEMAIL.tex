% kelompok 2 TUGAS 3 GIS (Database Acces)
%Tiara Rizki Wulansari (1154026)
%Mohammad Agung Deomartha (1154032)

\section{Sending Email}
	Mail Server adalah perangkat lunak program yang mendistribusikan file atau informasi sebagai respons atas permintaan yang dikirim via email, mail server juga digunakanpadabitnetuntukmenyediakanlayananserupaftp. Selainitumailserverjuga dapat dikatakan sebagai aplikasi yang digunakan untuk penginstalan email.
	Tak hanya sebagai sebuah program mail server juga bisa berupa sebuah komputer yang memang dikhususkan untuk menjalankan sebuah aplikasi perangkat lunak program ini. nah komputer ini di ibaratkan sebagai jantung dari system sebuah email. Program ini biasanya dikelola oleh programer yang disebut dengan post master.
	Mail server ini dikelola oleh seorang post master yang memiliki beberapa tugas pokok yaitu mengelola kaun, memonitor bagaimana kinerja server dan melaksanakan tugas administrative lainnya. Biasanya program ini menggunakan protocol antara lain smtp, pop3 dan imap.
	
	Mail Server juga bisa disebut sebagai sebuah komputer yang didedikasikan untuk menjalankan jenis aplikasi perangkat lunak komputer, hal ini dianggap sebagai bagian terpenting dari setiap email sistem. Mail Server biasanya dikelola oleh seorang yang biasanya dipanggil post master.
 
	Tugas Post Master:
	\begin{enumerate}
		\item Mengelola Account  
		\item Memonitor Kinerja Server 
		\item Tugas Administratif Lainnya
	\end{enumerate}

\subsection {Protokol Pada Mail Server}
	Protokol yang umum digunakan antara lain protokol SMTP, POP3 dan IMAP.
	
	\begin{enumerate}
		\item  SMTP (Simple Mail Transfer Protocol) 
		 	SMTP (Simple Mail Transfer Protocol) digunakan sebagai standar untuk menampung dan mendistribusikan email. Simple Mail Transfer Protocol atau SMTP digunakan untuk berkomunikasi dengan server guna mengirimkan email dari lokal email ke server, sebelum akhirnya dikirimkan ke server email penerima. Proses ini dikontrol dengan Mail Transfer Agent (MTA) yang ada dalam server email Anda. Port SMTP Default:
			   
			\begin{itemize}
				\item Port~25 –  Port tanpa dienkripsi
				\item Port 426 – Port SSL/TLS, nama lainnya SMTPS
			\end{itemize}
 
		\item {POP3 Post Office Protocol v3}
			POP3 (Post Office Protocol v3) dan IMAP (Internet Mail Application Protocol) digunakan agar user dapat mengambil dan membaca email secara remote yaitu tidak perlu login ke dalam sistem shelll mesin mail server tetapi cukup menguhubungi port tertentu dengan mail client yang mengimplementasikan protocol POP3 dan IMAP.
			POP3 (Post Office Protocol 3) adalah versi terbaru dari protokol standar untuk menerima email. POP3 merupakan protokol client/server dimana email dikirimkan dari server ke email lokal. Digunakan untuk berkomunikasi dengan email server dan mengunduh semua email ke email lokal (seperti Outlook, Thunderbird, Windows Mail, Mac Mail, dan sebagainya), tanpa menyimpan salinannya di server. Biasanya, dalam aplikasi email terdapat pilihan untuk tetap menyimpan salinan email yang diunduh pada server atau tidak.

			Apabila kita mengakses akun email yang sama dari perangkat berbeda, akan sangat direkomendasikan untuk menyimpan backup. Hal ini perlu dilakukan sebagai langkah antisipasi apabila perangkat kedua tidak bisa mengunduh email, sementara perangkat pertama sudah menghapusnya. 

			POP3 adalah protokol komunikasi satu arah, yang artinya data diambil dari server dan dikirimkan ke email lokal di perangkat komputer Anda. 
Port POP3 Default:
		\begin{itemize}
			\item Port 110 Port tanpa dienkripsi
			\item Port 995 Port SSL/TLS, nama lainnya POP3S
		\end{itemize}
\end{enumerate}

\subsubsection {Kelebihan Menggunakan POP3}
	\begin{itemize}
		\item Ketika email sudah diunduh melalui aplikasi local mail di komputer, Anda tidak perlu terhubung ke internet apabila Anda ingin membukanya kembali. 
		\item Kebanyakan tidak ada ukuran limit untuk email yang dikirim dan diterima. 
		\item Dapat membuka file attachment dengan cepat.
		\item Tidak ada ukuran maksimal untuk mailbox, kecuali harddisk komputer Anda penuh.
	\end{itemize}
	
\textbf{Kekurangan Menggunakan POP3} \par
\noindent 
\begin{itemize}
\item Jika JavaScript pada email reader diaktifkan, email phishing dengan embed JavaScript dapat terbaca di email. \par
\noindent 
\item Semua pesan akan disimpan di komputer. Hal ini dapat mengurangi space pada harddisk komputer. \par
\noindent 
\item Semua file attachment diunduh dan disimpan dalam komputer. Karenanya, potensi komputer terinfeksi virus dari email lebih besar. \par
\noindent 
\item Folder email terkadang hilang. Jika ini yang terjadi, upaya restore cukup sulit dilakukan.\end{itemize}
 \par
\noindent 
\item IMAP (Internet Message Access Protocol) \end{enumerate} \par
IMAP (Internet Message Access Protocol), seperti halnya POP3, juga digunakan untuk mengirimkan email ke local mail, hanya saja terdapat sedikit perbedaan cara kerja. \par
\vspace{12pt}
IMAP merupakan protokol komunikasi dua arah sebagai perubahan yang dibuat pada local mail yang dikirimkan ke server. Pada dasarnya, isi email tetap berada di server. Protokol IMAP lebih direkomendasikan oleh penyedia email seperti Gmail dibandingkan menggunakan POP3. \par
\vspace{12pt}
Dalam IMAP, email disimpan di server. ketika Anda akan mengecek email, local mail akan menghubungi server untuk menampilkan pesan email. Sehingga untuk file pesan email tetap berada di server dan tidak didownload ke email lokal. Port IMAP Default: \par
\noindent 
\begin{itemize}
\item Port 143 – Port tanpa dienkripsi \par
\noindent 
\item Port 993 – Port SSL/TLS, nama lainnya IMAPS\end{itemize}
 \par
\vspace{12pt}
