% Nama Kelompok : 3
% 1. Kezia Tirza Naramessakh (1154093)
% 2. Dimas Mathovani (1154101)

\section(Python Basic Operator)
Peran operator dalam proses perhitungan matematika sangatlah penting.
Setiap bahasa pemrograman, pasti memiliki sebuah operator. Walaupun kadang ada beberapa contoh operator yang memiliki perbedaan fungsi atau simbol. Namun hanya sebagian saja, bukan seluruhnya. Operator adalah konstruksi yang dapat memanipulasi nilai operan. 
Selain operator Aritmatika, Python juga mendukung operator berkondisi yang berfungsi untuk membandingkan suatu nilai dengan nilai yang lain. Operator-operator yang didukung oleh Python yaitu operator Unari \$( + dan – ) dan operator Binari ( +, -, *, /). Pada ekspresi Aritmatika berikut:

x = y + z
 
y dan z disebut sebagai operan dari operator +.  
Sebagai contoh operasi 3 + 2 = 5. Disini 3 dan 2 adalah operan dan + adalah operator.

Bahasa pemrograman Python mendukung berbagai macam operator, diantaranya :
\begin{enumerate}
	\item Operator Aritmatika(Arithmetic Operators)
	\item Operator Perbandingan(Comparison Relational Operators)
	\item Operator Penugasan (Assignment Operators)
	\item Operator Logika (Logical Operators)
	\item Operator Bitwise (Bitwise Operators)
	\item Operator Keanggotaan (Membership Operators)
	\item Operator Identitas (Identity Operators)
\end{enumerate}

Bitwise Operator : Operator bitwise (Bitwise Operators) adalah operator khusus yang disediakan PHP untuk menangani proses logika untuk bilangan biner. Bilangan biner atau binary adalah jenis bilangan yang hanya terdiri dari 2 jenis angka, yakni 0 dan 1. Jika operand (sebuah objek yang ada pada operasi matematika yang dapat digunakan untuk melakukan operasi) yang digunakan bukan bilangan biner, maka akan dikonversi secara otomatis oleh PHP menjadi bilangan biner.
