% Nama Kelompok : 3
% 1. Kezia Tirza Naramessakh (1154093)
% 2. Dimas Mathovani (1154101)
% 3. Mariani Rospilinda Siki (1154107)
% 4. Doli Jonviter NT Simbolon (1154016)
% 5. Benedictus Simatupang (1154116)

\section{Python Basic Operator}
\section{Pengenalan Python Programming}
Python adalah bahasa pemrograman yang menggabungkan kapabilitas, kemampuan, dengan sintaksis kode yang sangat jelas, dan dilengkapi dengan fungsionalitas pustaka standar yang besar serta komprehensif. Python mendukung multi paradigma pemrograman, utamanya namun tidak dibatasi pada pemrograman berorientasi objek, pemrograman imperatif, dan pemrograman fungsional. Salah satu fitur yang tersedia pada python adalah sebagai bahasa pemrograman dinamis yang dilengkapi dengan manajemen memori otomatis. Seperti halnya pada bahasa pemrograman dinamis lainnya, python umumnya digunakan sebagai bahasa skrip meski pada praktiknya penggunaan bahasa ini lebih luas mencakup konteks pemanfaatan yang umumnya tidak dilakukan dengan menggunakan bahasa skrip. Python dapat digunakan untuk berbagai keperluan pengembangan perangkat lunak dan dapat berjalan di berbagai platform sistem operasi. Manfaat besar dari sistem operasi seperti pyton adalah portabilitas aplikasi, setidaknya aplikasi yang dibuat dengan Python murni. Seluruh Sistem Operasi akan berjalan di mana pun dapat mengkompilasi Kernel Linux dan Python, dengan sedikit usaha daripada mengatakan bahwa porting seluruh Linux ke chipset yang berbeda. Aplikasi Python tidak perlu dikompilasi ulang untuk CPU target, dan jika sebagian besar Sistem Operasi dibuat dengan Python, port dibuat dengan mudah.

\subsection{Python Basic Operator}
Peran operator dalam proses perhitungan matematika sangatlah penting.
Setiap bahasa pemrograman, pasti memiliki sebuah operator. Walaupun kadang ada beberapa contoh operator yang memiliki perbedaan fungsi atau simbol. Namun hanya sebagian saja, bukan seluruhnya. Operator adalah konstruksi yang dapat memanipulasi nilai operan.
Operator merupakan simbol khusus yang mempresentasikan perhitungan seperti penambahan dan perkalian. Nilai yang digunakan oleh opertor sering disebut operand. Ekspresi merupakan kombinasi dari operator dan operandnya. Dalam sebuah eksekusi program, suatu ekspresi akan di evaluasi sehingga menghasilkan suatu nilai tunggal. Di dalam bahasa pemograman Phython ada beberapa tipe oprator seperti operator Aritmatika, Perbandingan, Penugasan, Logika, Bitwise, Keanggotaan dan Identitas.
Memberi nilai dari bagian kanan operator kebagian kiri operator. Operator penunjukan dalam bahasa phython menggunakan tanda sama dengan, termasuk:
+=, -=, *=, /=. 
Instruksi x = x+1 sama artinya dengan x +=1
Operator membandingkan kesamaan dua nilai digunakan tanda == dan menghasilkan sebuah ekspresi boolean 
Selain operator Aritmatika, Python juga mendukung operator berkondisi yang berfungsi untuk membandingkan suatu nilai dengan nilai yang lain. Operator-operator yang didukung oleh Python yaitu operator Unari \$( + dan – ) dan operator Binari ( +, -, *, /). Pada ekspresi Aritmatika berikut:

x = y + z
 
y dan z disebut sebagai operan dari operator +.  
Sebagai contoh operasi 3 + 2 = 5. Disini 3 dan 2 adalah operan dan + adalah operator.

Bahasa pemrograman Python mendukung berbagai macam operator, diantaranya :
\begin{enumerate}
	\item Operator Aritmatika(Arithmetic Operators)
	\item Operator Perbandingan(Comparison Relational Operators)
	\item Operator Penugasan (Assignment Operators)
	\item Operator Logika (Logical Operators)
	\item Operator Bitwise (Bitwise Operators)
	\item Operator Keanggotaan (Membership Operators)
	\item Operator Identitas (Identity Operators)
\end{enumerate}  


\subsubsection{Bitwise Operator}
Operator bitwise (Bitwise Operators) adalah operator khusus yang disediakan PHP untuk menangani proses logika untuk bilangan biner. Bilangan biner atau binary adalah jenis bilangan yang hanya terdiri dari 2 jenis angka, yakni 0 dan 1. Jika operand (sebuah objek yang ada pada operasi matematika yang dapat digunakan untuk melakukan operasi) yang digunakan bukan bilangan biner, maka akan dikonversi secara otomatis oleh PHP menjadi bilangan biner.

\subsubsection{Operator aritmatika}
Operator aritmatika sendiri merupakan operator yang dipergunakan dalam melakukan operasi matematika, seperti pengurangan, pembagian, penambahan, perkalian, pangkat, modulus dll. Berikut contoh penggunaan operator aritmatika pada bahasa pemrograman python:
print 3 + 2 * 3 - 10 / 5 
print (3 + 2) * (3 - 10 / 5)
print 3 ** 2 #pangkat
print 100 % 3 #modulus / sisa bagi

perintah tersebut akan menghasilkan nilai
7
5
9
1

Perhatikan perintah pada baris pertama tanpa harus menuliskan tanda kurung pun python sudah melakukan operasi matematika sesuai dengan aturan prioritas matematika. Operator yang biasa dipergunakan pada aritmatika ini pun beberapa dapat dipergunakan dalam melakukan manipulasi string contohnya :
print 'Pintarcoding.com! ' * 3
print 'hallo' + ' ' + 'Pintarcoding.com!'

\subsubsection{Operator Pembanding}
Ada beberapa operator pembanding antara lain, seperti contoh berikut:
print 10 > 2 # lebih besar, bernilai True
print 10 < 2 # lebih kecil, bernilai False
print 10 != 2 # tidak sama dengan, bernilai True
print 5 >= 5 # lebih besar atau sama dengan, bernilai True
print 5 <= 4 # lebih kecil atau sama dengan, bernilai False
print 5 == 5 # sama dengan, bernilai True

\subsubsection{Operator Logika}
Pada bahasa python terdapat 3 operator logika antara lain and, or dan not. Adapun cara penggunaan dari operator-operator tersebut adalah sebagai berikut:
print 10 > 2 and 10 > 5 # bernilai True
print True or False # bernilai True
print not False # bernilai True
perintah diatas digunakan untuk ketika dijalankan maka akan menghasilkan nilai True pada semua baris perintah diatas.

Perintah tersebut akan mengashilkan
Pintarcoding.com! Pintarcoding.com! Pintarcoding.com! 
hallo Pintarcoding.com!

