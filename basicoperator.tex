% Nama Kelompok : 3
% 1. Kezia Tirza Naramessakh (1154093)
% 2. Dimas Mathovani (1154101)
% 3. Mariani Rospilinda Siki (1154107)
% 4. Doli Jonviter NT Simbolon (1154016)
% 5. Benedictus Simatupang (1154116)

Python Programming
Python adalah bahasa pemrograman yang menggabungkan kapabilitas, kemampuan, dengan sintaksis kode yang sangat jelas, dan dilengkapi dengan fungsionalitas pustaka standar yang besar serta komprehensif. Python mendukung multi paradigma pemrograman, utamanya namun tidak dibatasi pada pemrograman berorientasi objek, pemrograman imperatif, dan pemrograman fungsional. Salah satu fitur yang tersedia pada python adalah sebagai bahasa pemrograman dinamis yang dilengkapi dengan manajemen memori otomatis. Seperti halnya pada bahasa pemrograman dinamis lainnya, python umumnya digunakan sebagai bahasa skrip meski pada praktiknya penggunaan bahasa ini lebih luas mencakup konteks pemanfaatan yang umumnya tidak dilakukan dengan menggunakan bahasa skrip. Python dapat digunakan untuk berbagai keperluan pengembangan perangkat lunak dan dapat berjalan di berbagai platform sistem operasi.Manfaat besar dari sistem operasi seperti pyton adalah portabilitas aplikasi, setidaknya aplikasi yang dibuat dengan Python murni. Seluruh Sistem Operasi akan berjalan di mana pun Anda dapat mengkompilasi Kernel Linux dan Python menjadi, dengan sedikit usaha daripada mengatakan bahwa porting seluruh Linux ke chipset yang berbeda. Aplikasi Python tidak perlu dikompilasi ulang untuk CPU target, dan jika sebagian besar Sistem Operasi dibuat dengan Python, port dibuat dengan mudah.

Jika proyek semacam itu dimulai, saya akan membayangkan terlebih dahulu akan menjadi lebih dari Sistem Operasi Riset tanpa penggunaan mutlak. Ini akan memungkinkan pengembang Python menjalankan kode mereka sangat dekat dengan logam kosong dari CPU dengan sedikit usaha, dan pengembang tertanam jenis proyek ini sebenarnya dapat mempercepat proses pengembangan sistem yang ada.


\section{Python Basic Operator}
Peran operator dalam proses perhitungan matematika sangatlah penting.
Setiap bahasa pemrograman, pasti memiliki sebuah operator. Walaupun kadang ada beberapa contoh operator yang memiliki perbedaan fungsi atau simbol. Namun hanya sebagian saja, bukan seluruhnya. Operator adalah konstruksi yang dapat memanipulasi nilai operan.
Operator merupakan simbol khusus yang mempresentasikan perhitungan seperti penambahan dan perkalian. Nilai yang digunakan oleh opertor sering disebut operand. Ekspresi merupakan kombinasi dari operator dan operandnya. Dalam sebuah eksekusi program, suatu ekspresi akan di evaluasi sehingga menghasilkan suatu nilai tunggal. Di dalam bahasa pemograman Phython ada beberapa tipe oprator seperti operator penunjukan, aritmatika, relasional, logika, dan pernyataan.
Memberi nilai dari bagian kanan operator kebagian kiri operator. Operator penunjukan dalam bahasa phython menggunakan tanda sama dengan, termasuk:
+=, -=, *=, /=, %=, **=, &=, |=, ^=, >>=, <<=.
Instruksi x = x+1 sama artinya dengan x +=1
Operator membandingkan kesamaan dua nilai digunakan tanda == dan menghasilkan sebuah ekspresi boolean 
Selain operator Aritmatika, Python juga mendukung operator berkondisi yang berfungsi untuk membandingkan suatu nilai dengan nilai yang lain. Operator-operator yang didukung oleh Python yaitu operator Unari \$( + dan – ) dan operator Binari ( +, -, *, /). Pada ekspresi Aritmatika berikut:

x = y + z
 
y dan z disebut sebagai operan dari operator +.  
Sebagai contoh operasi 3 + 2 = 5. Disini 3 dan 2 adalah operan dan + adalah operator.

Bahasa pemrograman Python mendukung berbagai macam operator, diantaranya :
\begin{enumerate}
	\item Operator Aritmatika(Arithmetic Operators)
	\item Operator Perbandingan(Comparison Relational Operators)
	\item Operator Penugasan (Assignment Operators)
	\item Operator Logika (Logical Operators)
	\item Operator Bitwise (Bitwise Operators)
	\item Operator Keanggotaan (Membership Operators)
	\item Operator Identitas (Identity Operators)
\end{enumerate}  


Bitwise Operator
Operator bitwise (Bitwise Operators) adalah operator khusus yang disediakan PHP untuk menangani proses logika untuk bilangan biner. Bilangan biner atau binary adalah jenis bilangan yang hanya terdiri dari 2 jenis angka, yakni 0 dan 1. Jika operand (sebuah objek yang ada pada operasi matematika yang dapat digunakan untuk melakukan operasi) yang digunakan bukan bilangan biner, maka akan dikonversi secara otomatis oleh PHP menjadi bilangan biner.

\subsubsection{operator aritmatika}
Operator Aritmatika adalah operator matematis yang terdiri dari operator penambahan, pengurangan, perkalian, pembagian, modulus, plus, dan minus.
Arithmetic Operator (operator aritmatika) adalah operator yang digunakan untuk melaksanakan operasi aritmatika.
Beberapa operator aritmatika antara lain:
\begin{enumerate}
	\item * : untuk perkalian
	\item + : untuk penjumlahan
	\item - : untuk pengurangan
	\item / : untuk pembagian
	\item \% : untuk sisa pembagian (modulus)
\end{enumerate}
