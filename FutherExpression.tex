%%%%%%%%%%%%  Generated using docx2latex.pythonanywhere.com  %%%%%%%%%%%%%%


\documentclass[a4paper,12pt]{report}

% Other options in place of 'report' are 1)article 2)book 3)letter
% Other options in place of 'a4paper' are 1)a5paper 2)b5paper 3)letterpaper 4)legalpaper 5)executivepaper


 %%%%%%%%%%%%  Include Packages  %%%%%%%%%%%%%%


\usepackage{amsmath}
\usepackage{latexsym}
\usepackage{amsfonts}
\usepackage{amssymb}
\usepackage{graphicx}
\usepackage{txfonts}
\usepackage{wasysym}
\usepackage{enumitem}
\usepackage{adjustbox}
\usepackage{ragged2e}
\usepackage{tabularx}
\usepackage{changepage}
\usepackage{setspace}
\usepackage{hhline}
\usepackage{multicol}
\usepackage{float}
\usepackage{multirow}
\usepackage{makecell}
\usepackage{fancyhdr}
\usepackage[toc,page]{appendix}
\usepackage[utf8]{inputenc}
\usepackage[T1]{fontenc}
\usepackage{hyperref}


 %%%%%%%%%%%%  Define Colors For Hyperlinks  %%%%%%%%%%%%%%


\hypersetup{
colorlinks=true,
linkcolor=blue,
filecolor=magenta,
urlcolor=cyan,
}
\urlstyle{same}


 %%%%%%%%%%%%  Set Depths for Sections  %%%%%%%%%%%%%%

% 1) Section
% 1.1) SubSection
% 1.1.1) SubSubSection
% 1.1.1.1) Paragraph
% 1.1.1.1.1) Subparagraph


\setcounter{tocdepth}{5}
\setcounter{secnumdepth}{5}


 %%%%%%%%%%%%  Set Page Margins  %%%%%%%%%%%%%%


\usepackage[a4paper,bindingoffset=0.2in,headsep=0.5cm,left=1.0in,right=1.0in,bottom=2cm,top=2cm,headheight=2cm]{geometry}
\everymath{\displaystyle}


 %%%%%%%%%%%%  Set Depths for Nested Lists created by \begin{enumerate}  %%%%%%%%%%%%%%


\setlistdepth{9}
\newlist{myEnumerate}{enumerate}{9}
	\setlist[myEnumerate,1]{label=\arabic*)}
	\setlist[myEnumerate,2]{label=\alph*)}
	\setlist[myEnumerate,3]{label=(\roman*)}
	\setlist[myEnumerate,4]{label=(\arabic*)}
	\setlist[myEnumerate,5]{label=(\Alph*)}
	\setlist[myEnumerate,6]{label=(\Roman*)}
	\setlist[myEnumerate,7]{label=\arabic*}
	\setlist[myEnumerate,8]{label=\alph*}
	\setlist[myEnumerate,9]{label=\roman*}

\renewlist{itemize}{itemize}{9}
	\setlist[itemize]{label=$\cdot$}
	\setlist[itemize,1]{label=\textbullet}
	\setlist[itemize,2]{label=$\circ$}
	\setlist[itemize,3]{label=$\ast$}
	\setlist[itemize,4]{label=$\dagger$}
	\setlist[itemize,5]{label=$\triangleright$}
	\setlist[itemize,6]{label=$\bigstar$}
	\setlist[itemize,7]{label=$\blacklozenge$}
	\setlist[itemize,8]{label=$\prime$}



 %%%%%%%%%%%%  Header here  %%%%%%%%%%%%%%


\pagestyle{fancy}
\fancyhf{}


 %%%%%%%%%%%%  Footer here  %%%%%%%%%%%%%%




 %%%%%%%%%%%%  Print Page Numbers  %%%%%%%%%%%%%%


\rfoot{\thepage}


 %%%%%%%%%%%%  This sets linespacing (verticle gap between Lines) Default=1 %%%%%%%%%%%%%%


\setstretch{1.08}


 %%%%%%%%%%%%  Document Code starts here %%%%%%%%%%%%%%


\begin{document}
\sloppy
\begin{center}{\fontsize{14pt}{14pt}\selectfont \textbf{BAB IV} \\}\end{center} \par
\noindent 
\begin{center}{\fontsize{14pt}{14pt}\selectfont \textbf{FURTHER EXPRESSION} \\}\end{center} \par
\vspace{14pt}
\noindent 
 \hspace*{0.5in} Stiap kode yang dituliskan menggunakan bahasa yang dikompilasi seperti C, C++ atau Java dapat diintegrasikan ke skrip Python lainnya. Kode ini diagnggap sebagai ektensi. \par
\noindent 
 \hspace*{0.5in} Modul ekstensi Python tidak lebih dari sekedar perpustakaan C biasa. Pada mesin Unix, perpustakaan ini biasanya diakhiri dengan .so (untuk objek bersama). Pada mesin windows, biasanya melihat .dll (untuk perpusatkaan yang terhubung secara dinamis).  \par
\vspace{12pt}
\noindent 
\textbf{4.1 Pra-Persyaratan untuk Menulis Ekstensi} \par
\noindent 
 \hspace*{0.5in} Untuk memulai ektensi, memerlukan file header Python. Pada mesin Unix, biasanya memerlukan instalasi paket khusus pengembang seperti python 2-5. \par
\noindent 
 \hspace*{0.5in} Pengguna window mendapatkan header ini sebagai bagian dari paket saat menggunakan pemasang Python biner. \par
\noindent 
 \hspace*{0.5in} Harus memiliki pengetahuan yang baik tentang C atau C++ untuk menulis ekstensi Python menggunakan pemrograman C. \par
\noindent 
 \hspace*{0.5in} Untuk melihat modul ekstensi Python, perlu mengelompokkan kode menjadi empat bagian : \par
\noindent 
\begin{itemize}
\item File header Python h \par
\noindent 
\item Fungsi C yang ingin ditampilkan sebagai antarmuka dari modul \par
\noindent 
\item Sebuah tabel memetakan nama-nama fungsi saat pengembang Python melihat ke fungsi C didalam modul ekstensi \par
\noindent 
\item Fungsi inilisasi\end{itemize}
 \par
\vspace{12pt}
Perlu menyertakan file header Python.h di file sumber C memberi akses ke API Python internal digunakan untuk menghitung modul ke penerjamah. \par
Menyertakan header Python.h sebelum header lain yang mungkin dibutuhkan. Mengikuti termasuk dengan fungsi yang ingin dipanggil dari Python. \par
Tanda tangan penerapan C fungsi selalu mengambil salah satu dari tiga bentuk berikut : \par
\noindent 
{\fontsize{10pt}{10pt}\selectfont static PyObject *MyFunction( PyObject *self, PyObject *args );} \par
\noindent 
\vspace{10pt}
\noindent 
{\fontsize{10pt}{10pt}\selectfont static PyObject *MyFunctionWithKeywords(PyObject *self,} \par
\noindent 
{\fontsize{10pt}{10pt}\selectfont ~~~~~~~~~~~~~~~~~~~~~~~~~~~~ PyObject *args,} \par
\noindent 
{\fontsize{10pt}{10pt}\selectfont ~~~~~~~~~~~~~~~~~~~~~~~~~~~~ PyObject *kw);} \par
\noindent 
\vspace{10pt}
\noindent 
{\fontsize{10pt}{10pt}\selectfont static PyObject *MyFunctionWithNoArgs( PyObject *self );} \par
\vspace{12pt}
\noindent 
 \hspace*{0.5in} Masing-masing deklarasi seelumnya mengembalikan objek Python. Tidak ada yang namanya fungsi void dengan Python seperti ada di C. Jika ingin fungsi mengembalikan nilai, Python. Header Python mendefinisikan makro. Py $  \_  $Return $  \_  $None yang melakukan ini. \par
\noindent 
 \hspace*{0.5in} Nama-nama fungsi C bisa menjadi apapun yang disuka karena tidak pernah diluar modul ekstensi mendefinisikan sebagai statis. \par
\noindent 
 \hspace*{0.5in} Fungi Cbiasanya diberi nama dengan menggabnungkan modul dan fungsi Python bersama-sama yang ditunjukan disini : \par
\noindent 
static PyObject *\textit{module $  \_  $func}(PyObject *self, PyObject *args)  $  \{  $ \par
\noindent 
~~ /* Do your stuff here. */ \par
\noindent 
~~ Py $  \_  $RETURN $  \_  $NONE; \par
\noindent 
 $  \}  $ \par
\vspace{12pt}
\vspace{12pt}
\noindent 
 \hspace*{0.5in} Ini adalah fungsi Python yang disebut func didalam modul-modul. Memasukkan petunjuk ke fungsi C ke dalam tabel metode untuk modul yang biasanya muncul selanjutnya dikode sumber tael pemetaan metode. \par
\noindent 
 \hspace*{0.5in} Tabel metode ini adalah susunan sederhana dari struktur PyMethodSef. Struktur itu terlihat seperti ini : \par
\noindent 
struct PyMethodDef  $  \{  $ \par
\noindent 
~~ char *ml $  \_  $name; \par
\noindent 
~~ PyCFunction ml $  \_  $meth; \par
\noindent 
~~ int ml $  \_  $flags; \par
\noindent 
~~ char *ml $  \_  $doc; \par
\noindent 
 $  \}  $; \par
\vspace{12pt}
\noindent 
Inilai uraian anggota struktur ini : \par
\noindent 
\begin{itemize}
\item Ml $  \_  $name \par
Nama fungsi yang digunakan penafsir Python saat digunakan dalam program Python \par
\noindent 
\item Ml $  \_  $meth \par
Menjadi alamaat ke fungsi yang memiliki salah satu tanda tangan yang dijelaskan dalam penelusuran sebelumnya \par
\noindent 
\item Ml $  \_  $flags \par
Memberitahu penafsir yang mana dari tiga tanda tangan yang digunakan ml $  \_  $meth. Bendera ini biasanya mmiliki nilai meth $  \_  $varargs. Bendera ini dapat digandakan dengan or’ed dengan meth $  \_  $keywords jika ingin memiarkan argumen kata kunci masuk ke fungsi. Ini juga bisa memiliki nilai meth $  \_  $noargs yang menunjukan bahwa tidak ingin menerima argumen apa pun. \par
\noindent 
\item Ml $  \_  $doc \par
Ini adalah docstring untuk fungsi yang bisa jadi NULL jika tidak ingin menulisnya. \par
\vspace{12pt}
\vspace{12pt}
\noindent 
 \hspace*{0.5in} Tabel ini perlu diakhiri dengan sentinel yang terdiri dari NULL dan 0 untuk anggota yang sesuai. \par
\vspace{12pt}
\noindent 
Contoh: \par
\noindent 
static PyMethodDef \textit{module} $  \_  $methods[] =  $  \{  $ \par
\noindent 
~~  $  \{  $ "\textit{func}", (PyCFunction)\textit{module $  \_  $func}, METH $  \_  $NOARGS, NULL  $  \}  $, \par
\noindent 
~~  $  \{  $ NULL, NULL, 0, NULL  $  \}  $ \par
\noindent 
 $  \}  $; \par
\vspace{12pt}
\noindent 
 \hspace*{0.5in} Bagian terakhir dari modul ekstensi adalah fungsi inialisasi. Fungsi ini dipanggil oleh juru bahasa Python saat modul diisikan. Hal ini diperlukan agar fungsi diberi nama intiModule dimana modul adalah nama modul. \par
\noindent 
 \hspace*{0.5in} Fungsi inialisasi perlu diekspor dari perpustakaan yang akan dibangun. Header Python mendefinisikan PyMODINIT $  \_  $Func untuk memasukkan mantra yang sesuai agar terjadi pada lingkungan tertentu tempat menyuusun. Yang harus dilakukan adalah mengunakan saat menentukan fungsinya. \par
\noindent 
 \hspace*{0.5in} Fungsi inialisasi C umumnya memiliki strktur keseluruhan berikut : \par
\noindent 
PyMODINIT $  \_  $FUNC init\textit{Module}()  $  \{  $ \par
\noindent 
~~ Py $  \_  $InitModule3(\textit{func}, \textit{module} $  \_  $methods, "docstring..."); \par
\noindent 
 $  \}  $ \par
\vspace{12pt}
\noindent 
Berikut adalah penjelasan fugsi Py $  \_  $IntiModule : \par
\noindent 
\item Func  \par
\noindent 
Ini adalah fungsi yang akan diekspor \par
\noindent 
\item Module \par
\noindent 
Ini adalah nama tabel pemetaan yang didefinisikan diatas \par
\noindent 
\item Docstring\end{itemize}
 \par
\noindent 
Ini adalah komentar yang ingin diberikan diekstensi \par
\vspace{12pt}
\noindent 
 \hspace*{0.5in} Menempatkan ini semua bersama-sama terlihat sebagai berikut : \par
\noindent 
 $  \#  $include <Python.h> \par
\vspace{12pt}
\noindent 
static PyObject *\textit{module $  \_  $func}(PyObject *self, PyObject *args)  $  \{  $ \par
\noindent 
~~ /* Do your stuff here. */ \par
\noindent 
~~ Py $  \_  $RETURN $  \_  $NONE; \par
\noindent 
 $  \}  $ \par
\vspace{12pt}
\noindent 
static PyMethodDef \textit{module} $  \_  $methods[] =  $  \{  $ \par
\noindent 
~~  $  \{  $ "\textit{func}", (PyCFunction)\textit{module $  \_  $func}, METH $  \_  $NOARGS, NULL  $  \}  $, \par
\noindent 
~~  $  \{  $ NULL, NULL, 0, NULL  $  \}  $ \par
\noindent 
 $  \}  $; \par
\vspace{12pt}
\noindent 
PyMODINIT $  \_  $FUNC init\textit{Module}()  $  \{  $ \par
\noindent 
~~ Py $  \_  $InitModule3(\textit{func}, \textit{module} $  \_  $methods, "docstring..."); \par
\noindent 
 $  \}  $ \par
\vspace{12pt}
\vspace{12pt}
\noindent 
Contoh : \par
\noindent 
 $  \#  $include <Python.h> \par
\vspace{12pt}
\noindent 
static PyObject* helloworld(PyObject* self) \par
\noindent 
 $  \{  $ \par
\noindent 
~~~ return Py $  \_  $BuildValue("s", "Hello, Python extensions!!"); \par
\noindent 
 $  \}  $ \par
\vspace{12pt}
\noindent 
static char helloworld $  \_  $docs[] = \par
\noindent 
~~~ "helloworld( ): Any message you want to put here!! $  \textbackslash  $n"; \par
\vspace{12pt}
\noindent 
static PyMethodDef helloworld $  \_  $funcs[] =  $  \{  $ \par
\noindent 
~~~  $  \{  $"helloworld", (PyCFunction)helloworld,  \par
\noindent 
~~~~ METH $  \_  $NOARGS, helloworld $  \_  $docs $  \}  $, \par
\noindent 
~~~  $  \{  $NULL $  \}  $ \par
\noindent 
 $  \}  $; \par
\vspace{12pt}
\noindent 
void inithelloworld(void) \par
\noindent 
 $  \{  $ \par
\noindent 
~~~ Py $  \_  $InitModule3("helloworld", helloworld $  \_  $funcs, \par
\noindent 
~~~~~~~~~~~~~~~~~~ "Extension module example!"); \par
\noindent 
 $  \}  $ \par
\vspace{12pt}
\vspace{12pt}
\noindent 
 \hspace*{0.5in} Disini fungsi Py $  \_  $BuildValue digunakan untuk membangun nilai Python.  \par
\vspace{12pt}
\vspace{12pt}
\noindent 
\textbf{4.2 Membangun dan Menginstal Ekstensi} \par
\vspace{12pt}
\vspace{12pt}
\noindent 
 \hspace*{0.5in} Distutils paket membuatnya sangat mudah mendistribusikan modul Python, baik Python murni dan modul ekstensi dengan cara standar. Modul didistribusikan dalam bentuk sumber dan dibangun dan diinstal melalui skrip setup yang iasa disebut seup.py sebagai berikut : \par
\noindent 
from distutils.core import setup, Extension \par
\noindent 
setup(name='helloworld', version='1.0',~  $  \textbackslash  $ \par
\noindent 
~~~~~ ext $  \_  $modules=[Extension('helloworld', ['hello.c'])]). \par
\vspace{12pt}
\noindent 
 \hspace*{0.5in} Sekarang gunakan perintah berikut yang aka melalakukan semua kompilasi dan langkah penghubunh yang diperlukan dengan perintah dan bendera penyusun dan penghubung yang benar dan menyalin perpustakaan dinamis yang dihasilkan ke dalam direktori yang sesuai . \par
\vspace{12pt}
\noindent 
Contoh : \par
\noindent 
 $  \$  $ python setup.py install \par
\vspace{12pt}
\noindent 
 \hspace*{0.5in} Pada sistem berbasis Unix kemungkinan besar perlu menjalankan perintah ini sebagai root agar meminta izin untuk menulis ke direktori paket situs. Ini biasanya tidak menjadi masalah pada window. \par
Setelah~menginstal ekstensi, akan dapat mengimpor dan memanggil ekstensi tersebut di skrip Python  sebagai berikut : \par
\noindent 
 $  \#  $!/usr/bin/python \par
\noindent 
import helloworld \par
\vspace{12pt}
\noindent 
print helloworld.helloworld() \par
\vspace{14pt}
\noindent 
Ini~akan menghasilkan hasil sebagai berikut  : \par
\noindent 
Hello, Python extensions!! \par
\vspace{12pt}
Seperti kemungkinan besar ingin mendefinisikan fungsi yang menerima argumen, dapat menggunakan salah satu tanda tangan lain untuk fungsi C. Sebagai contoh, fungsi berikut yang menerima beberapa parameter akan didefinisikan seperti ini : \par
\noindent 
static PyObject *\textit{module $  \_  $func}(PyObject *self, PyObject *args)  $  \{  $ \par
\noindent 
~~ /* Parse args and do something interesting here. */ \par
\noindent 
~~ Py $  \_  $RETURN $  \_  $NONE; \par
\noindent 
 $  \}  $ \par
\vspace{12pt}
Tabel metode yang berisi entri untuk fungsi baru akan terlihat seperti ini : \par
\noindent 
static PyMethodDef \textit{module} $  \_  $methods[] =  $  \{  $ \par
\noindent 
~~  $  \{  $ "\textit{func}", (PyCFunction)\textit{module $  \_  $func}, METH $  \_  $NOARGS, NULL  $  \}  $, \par
\noindent 
~~  $  \{  $ "\textit{func}", \textit{module $  \_  $func}, METH $  \_  $VARARGS, NULL  $  \}  $, \par
\noindent 
~~  $  \{  $ NULL, NULL, 0, NULL  $  \}  $ \par
\noindent 
 $  \}  $; \par
\vspace{12pt}
Menggunakan fungsi API PyArg $  \_  $ParseTuple untuk mengekstrak argumen dari satu pointer PyObject yang dikirimkan ke fungsi C. Argumen pertama untuk PyArg $  \_  $ParseTuple adalah args argumen. Ini adalah objek yang akan parsing. Argumen kedua adalah string format yang menggambarkan argumen saat mengharapkannya muncul. Setiap argumen diwakili oleh satu atau lebih karakter dalam format string sebagai berikut : \par
\noindent 
static PyObject *\textit{module $  \_  $func}(PyObject *self, PyObject *args)  $  \{  $ \par
\noindent 
~~ int i; \par
\noindent 
~~ double d; \par
\noindent 
~~ char *s; \par
\vspace{12pt}
\noindent 
~~ if (!PyArg $  \_  $ParseTuple(args, "ids",  $  \&  $i,  $  \&  $d,  $  \&  $s))  $  \{  $ \par
\noindent 
~~~~~ return NULL; \par
\noindent 
~~  $  \}  $ \par
\noindent 
~~  \par
\noindent 
~~ /* Do something interesting here. */ \par
\noindent 
~~ Py $  \_  $RETURN $  \_  $NONE; \par
\noindent 
 $  \}  $ \par
\vspace{12pt}
Mengkompilasi versi baru dari modul dan mengimpornya memungkinkan untuk memanggil fungsi baru dengan sejumlah argumen dari jenis apa pun : \par
\noindent 
module.func(1, s="three", d=2.0) \par
\noindent 
module.func(i=1, d=2.0, s="three") \par
\noindent 
module.func(s="three", d=2.0, i=1) \par
\vspace{12pt}
\vspace{12pt}
\vspace{12pt}
\vspace{12pt}
\noindent 
 \hspace*{0.5in} Berikut adalah tanda tangan standar untuk fungsi PyArg $  \_  $ParseTuple: \par
\noindent 
int PyArg $  \_  $ParseTuple(PyObject* tuple,char* format,...) \par
\vspace{12pt}
\noindent 
 \hspace*{0.5in} Fungsi ini mengembalikan 0 untuk kesalahan, dan nilai tidak sama dengan 0 untuk kesuksesan. Tuple adalah PyObject * yang merupakan argumen kedua dari fungsi C. Format berikut adalah string C yang menggambarkan argumen wajib dan opsional.\vspace{\baselineskip}
~~~~~~ Berikut adalah daftar kode format untuk fungsi PyArg $  \_  $ParseTuple: \par


 %%%%%%%%%%%%  Table No:1 Here %%%%%%%%%%%%%%


\begin{table}[H]
\centering
\begin{adjustbox}{width=\textwidth}
\begin{tabular}{ p{0.52in}p{1.51in}p{2.2in} }
\hhline{---}
\multicolumn{1}{|p{0.52in}}{\Centering \textbf{Code}} & \multicolumn{1}{|p{1.51in}}{\Centering \textbf{C type}} & \multicolumn{1}{|p{2.2in}|}{\Centering \textbf{Meaning}} & \hhline{---}
\multicolumn{1}{|p{0.52in}}{c} & \multicolumn{1}{|p{1.51in}}{char} & \multicolumn{1}{|p{2.2in}|}{String Python dengan panjang 1 menjadi huruf C.} & \hhline{---}
\multicolumn{1}{|p{0.52in}}{d} & \multicolumn{1}{|p{1.51in}}{double} & \multicolumn{1}{|p{2.2in}|}{Pelampung Python menjadi C ganda.} & \hhline{---}
\multicolumn{1}{|p{0.52in}}{f} & \multicolumn{1}{|p{1.51in}}{float} & \multicolumn{1}{|p{2.2in}|}{Pelampung Python menjadi pelampung C.} & \hhline{---}
\multicolumn{1}{|p{0.52in}}{i} & \multicolumn{1}{|p{1.51in}}{int} & \multicolumn{1}{|p{2.2in}|}{Int Python menjadi int int} & \hhline{---}
\multicolumn{1}{|p{0.52in}}{l} & \multicolumn{1}{|p{1.51in}}{long} & \multicolumn{1}{|p{2.2in}|}{Sebuah int Python menjadi panjang C.} & \hhline{---}
\multicolumn{1}{|p{0.52in}}{L} & \multicolumn{1}{|p{1.51in}}{long long} & \multicolumn{1}{|p{2.2in}|}{Sebuah int Python menjadi C panjang panjang} & \hhline{---}
\multicolumn{1}{|p{0.52in}}{O} & \multicolumn{1}{|p{1.51in}}{PyObject*} & \multicolumn{1}{|p{2.2in}|}{Gets non-NULL meminjam referensi ke argumen Python.} & \hhline{---}
\multicolumn{1}{|p{0.52in}}{s} & \multicolumn{1}{|p{1.51in}}{char*} & \multicolumn{1}{|p{2.2in}|}{Python string tanpa nulls tertanam ke C char *.} & \hhline{---}
\multicolumn{1}{|p{0.52in}}{s $  \#  $} & \multicolumn{1}{|p{1.51in}}{char*+int} & \multicolumn{1}{|p{2.2in}|}{Setiap string Python ke alamat dan panjang C.} & \hhline{---}
\multicolumn{1}{|p{0.52in}}{t $  \#  $} & \multicolumn{1}{|p{1.51in}}{char*+int} & \multicolumn{1}{|p{2.2in}|}{Read-only penyangga segmen tunggal ke alamat C dan panjangnya.} & \hhline{---}
\multicolumn{1}{|p{0.52in}}{u} & \multicolumn{1}{|p{1.51in}}{Py $  \_  $UNICODE*} & \multicolumn{1}{|p{2.2in}|}{Python Unicode tanpa nulls tertanam ke C.} & \hhline{---}
\multicolumn{1}{|p{0.52in}}{u $  \#  $} & \multicolumn{1}{|p{1.51in}}{Py $  \_  $UNICODE*+int} & \multicolumn{1}{|p{2.2in}|}{Setiap alamat dan panjang Python Unicode C.} & \hhline{---}
\multicolumn{1}{|p{0.52in}}{w $  \#  $} & \multicolumn{1}{|p{1.51in}}{char*+int} & \multicolumn{1}{|p{2.2in}|}{Membaca / menulis penyangga segmen tunggal ke alamat dan panjang C.} & \hhline{---}
\multicolumn{1}{|p{0.52in}}{z} & \multicolumn{1}{|p{1.51in}}{char*} & \multicolumn{1}{|p{2.2in}|}{Seperti s, juga menerima None (set C char * ke NULL).} & \hhline{---}
\multicolumn{1}{|p{0.52in}}{z $  \#  $} & \multicolumn{1}{|p{1.51in}}{char*+int} & \multicolumn{1}{|p{2.2in}|}{Seperti s  $  \#  $, juga menerima None (set C char * ke NULL).} & \hhline{---}
\multicolumn{1}{|p{0.52in}}{(...)} & \multicolumn{1}{|p{1.51in}}{as per ...} & \multicolumn{1}{|p{2.2in}|}{Urutan Python diperlakukan sebagai satu argumen per item.} & \hhline{---}
\multicolumn{1}{|p{0.52in}}{ $  \vert  $} & \multicolumn{1}{|p{1.51in}}{ $  $} & \multicolumn{1}{|p{2.2in}|}{Argumen berikut bersifat opsional.} & \hhline{---}
\multicolumn{1}{|p{0.52in}}{:} & \multicolumn{1}{|p{1.51in}}{ $  $} & \multicolumn{1}{|p{2.2in}|}{Format akhir, diikuti dengan nama fungsi untuk pesan error.} & \hhline{---}
\multicolumn{1}{|p{0.52in}}{;} & \multicolumn{1}{|p{1.51in}}{ $  $} & \multicolumn{1}{|p{2.2in}|}{Format akhir, diikuti oleh seluruh pesan kesalahan teks.} & \hline
\end{tabular}
\end{adjustbox}
\end{table}


 %%%%%%%%%%%%  Table No:1 Ends Here %%%%%%%%%%%%%%


\vspace{12pt}
\vspace{14pt}
Py $  \_  $BuildValue~mengambil format string seperti PyArg $  \_  $ParseTuple. Alih-alih menyampaikan alamat nilai yang sedang  bangun, melewati nilai sebenarnya. Berikut adalah contoh yang menunjukkan bagaimana menerapkan fungsi tambah : \par
\noindent 
static PyObject *foo $  \_  $add(PyObject *self, PyObject *args)  $  \{  $ \par
\noindent 
~~ int a; \par
\noindent 
~~ int b; \par
\vspace{12pt}
\noindent 
~~ if (!PyArg $  \_  $ParseTuple(args, "ii",  $  \&  $a,  $  \&  $b))  $  \{  $ \par
\noindent 
~~~~~ return NULL; \par
\noindent 
~~  $  \}  $ \par
\noindent 
~~ return Py $  \_  $BuildValue("i", a + b); \par
\noindent 
 $  \}  $ \par
\vspace{14pt}
Ini adalah apa yang akan terlihat seperti jika diimplementasikan dengan Python : \par
\noindent 
def add(a, b): \par
\noindent 
~~ return (a + b) \par
\vspace{16pt}
Mengembalikan dua nilai dari fungsi sebagai berikut, ini akan dipicu menggunakan daftar dengan Python : \par
\noindent 
static PyObject *foo $  \_  $add $  \_  $subtract(PyObject *self, PyObject *args)  $  \{  $ \par
\noindent 
~~ int a; \par
\noindent 
~~ int b; \par
\vspace{12pt}
\noindent 
~~ if (!PyArg $  \_  $ParseTuple(args, "ii",  $  \&  $a,  $  \&  $b))  $  \{  $ \par
\noindent 
~~~~~ return NULL; \par
\noindent 
~~  $  \}  $ \par
\noindent 
~~ return Py $  \_  $BuildValue("ii", a + b, a - b); \par
\noindent 
 $  \}  $ \par
\vspace{16pt}
Ini adalah apa yang akan terlihat seperti jika diimplementasikan dengan Python : \par
\noindent 
def add $  \_  $subtract(a, b): \par
\noindent 
~~ return (a + b, a - b) \par
\vspace{14pt}
Berikut~adalah tanda tangan standar untuk fungsi Py $  \_  $BuildValue  : \par
\noindent 
PyObject* Py $  \_  $BuildValue(char* format,...) \par
\vspace{14pt}
Format berikut adalah string C yang menggambarkan objek Python untuk dibangun. Argumentasi berikut Py $  \_  $BuildValue adalah nilai C dari mana hasilnya dibuat. Hasil PyObject * adalah referensi baru.  \par
Berikut daftar tabel string kode yang umum digunakan, yang nol atau lebihnya digabungkan ke dalam format string : \par


 %%%%%%%%%%%%  Table No:2 Here %%%%%%%%%%%%%%


\begin{table}[H]
\centering
\begin{adjustbox}{width=\textwidth}
\begin{tabular}{ p{0.69in}p{1.51in}p{1.88in} }
\hhline{---}
\multicolumn{1}{|p{0.69in}}{\Centering \textbf{Code}} & \multicolumn{1}{|p{1.51in}}{\Centering \textbf{Tipe C }} & \multicolumn{1}{|p{1.88in}|}{\textbf{Meaning}} & \hhline{---}
\multicolumn{1}{|p{0.69in}}{c} & \multicolumn{1}{|p{1.51in}}{char} & \multicolumn{1}{|p{1.88in}|}{Sebuah char C menjadi string Python dengan panjang 1.} & \hhline{---}
\multicolumn{1}{|p{0.69in}}{d} & \multicolumn{1}{|p{1.51in}}{double} & \multicolumn{1}{|p{1.88in}|}{C ganda menjadi float Python.} & \hhline{---}
\multicolumn{1}{|p{0.69in}}{f} & \multicolumn{1}{|p{1.51in}}{float} & \multicolumn{1}{|p{1.88in}|}{Pelampung C menjadi float Python.} & \hhline{---}
\multicolumn{1}{|p{0.69in}}{i} & \multicolumn{1}{|p{1.51in}}{int} & \multicolumn{1}{|p{1.88in}|}{C int menjadi int Python.} & \hhline{---}
\multicolumn{1}{|p{0.69in}}{l} & \multicolumn{1}{|p{1.51in}}{long} & \multicolumn{1}{|p{1.88in}|}{Sebuah C panjang menjadi int Python.} & \hhline{---}
\multicolumn{1}{|p{0.69in}}{N} & \multicolumn{1}{|p{1.51in}}{PyObject*} & \multicolumn{1}{|p{1.88in}|}{Melewati objek Python dan mencuri referensi.} & \hhline{---}
\multicolumn{1}{|p{0.69in}}{O} & \multicolumn{1}{|p{1.51in}}{PyObject*} & \multicolumn{1}{|p{1.88in}|}{Melewati objek Python dan MENINGKATKANNYA seperti biasa.} & \hhline{---}
\multicolumn{1}{|p{0.69in}}{O $  \&  $} & \multicolumn{1}{|p{1.51in}}{convert+void*} & \multicolumn{1}{|p{1.88in}|}{Konversi sewenang-wenang} & \hhline{---}
\multicolumn{1}{|p{0.69in}}{s} & \multicolumn{1}{|p{1.51in}}{char*} & \multicolumn{1}{|p{1.88in}|}{C 0-diakhiri char * ke string Python, atau NULL to None.} & \hhline{---}
\multicolumn{1}{|p{0.69in}}{s $  \#  $} & \multicolumn{1}{|p{1.51in}}{char*+int} & \multicolumn{1}{|p{1.88in}|}{C char * dan panjang ke string Python, atau NULL to None.} & \hhline{---}
\multicolumn{1}{|p{0.69in}}{u} & \multicolumn{1}{|p{1.51in}}{Py $  \_  $UNICODE*} & \multicolumn{1}{|p{1.88in}|}{String C-wide, null-terminated menjadi Python Unicode, atau NULL to None.} & \hhline{---}
\multicolumn{1}{|p{0.69in}}{u $  \#  $} & \multicolumn{1}{|p{1.51in}}{Py $  \_  $UNICODE*+int} & \multicolumn{1}{|p{1.88in}|}{String dan panjang lebar C ke Unicode Python, atau NULL to None.} & \hhline{---}
\multicolumn{1}{|p{0.69in}}{w $  \#  $} & \multicolumn{1}{|p{1.51in}}{char*+int} & \multicolumn{1}{|p{1.88in}|}{Membaca / menulis penyangga segmen tunggal ke alamat dan panjang C.} & \hhline{---}
\multicolumn{1}{|p{0.69in}}{z} & \multicolumn{1}{|p{1.51in}}{char*} & \multicolumn{1}{|p{1.88in}|}{Seperti s, juga menerima None (set C char * ke NULL).} & \hhline{---}
\multicolumn{1}{|p{0.69in}}{z $  \#  $} & \multicolumn{1}{|p{1.51in}}{char*+int} & \multicolumn{1}{|p{1.88in}|}{Seperti s  $  \#  $, juga menerima None (set C char * ke NULL).} & \hhline{---}
\multicolumn{1}{|p{0.69in}}{(...)} & \multicolumn{1}{|p{1.51in}}{as per ...} & \multicolumn{1}{|p{1.88in}|}{Membangun tuple Python dari nilai C.} & \hhline{---}
\multicolumn{1}{|p{0.69in}}{[...]} & \multicolumn{1}{|p{1.51in}}{as per ...} & \multicolumn{1}{|p{1.88in}|}{Membangun daftar Python dari nilai C.} & \hhline{---}
\multicolumn{1}{|p{0.69in}}{ $  \{  $... $  \}  $} & \multicolumn{1}{|p{1.51in}}{as per ...} & \multicolumn{1}{|p{1.88in}|}{Bangun kamus Python dari nilai C, kunci dan nilai bergantian.} & \hline
\end{tabular}
\end{adjustbox}
\end{table}


 %%%%%%%%%%%%  Table No:2 Ends Here %%%%%%%%%%%%%%


\vspace{12pt}
Kode  $  \{  $... $  \}  $ membangun kamus dari sejumlah nilai C, kunci dan nilai bergantian. Misalnya, Py $  \_  $BuildValue (" $  \{  $issi $  \}  $", 23, "zig", "zag", 42) mengembalikan kamus seperti  $  \{  $23: 'zig', 'zag': 42 $  \}  $ Python. \par
Setiap blok memori yang dialokasikan dengan malloc () pada akhirnya harus dikembalikan ke genangan memori yang tersedia dengan satu panggilan untuk membebaskan (). Penting untuk menelepon gratis () pada waktu yang tepat. Jika alamat blok dilupakan tapi gratis () tidak dipanggil untuk itu, memori yang ditempatinya tidak dapat digunakan kembali sampai program berakhir. Ini disebut kebocoran memori. Di sisi lain, jika sebuah program memanggil gratis () untuk satu blok dan kemudian terus menggunakan blok tersebut, itu menciptakan konflik dengan penggunaan ulang blok melalui panggilan malloc () yang lain. Ini disebut dengan menggunakan memori yang dibebaskan. Ini memiliki konsekuensi buruk yang sama seperti merujuk pada data yang tidak diinisiasi - dump inti, hasil yang salah, crash misterius. \par
Karena Python membuat penggunaan malloc () dan gratis (), dibutuhkan strategi untuk menghindari kebocoran memori dan juga penggunaan memori yang bebas. Metode yang dipilih disebut penghitungan referensi. Prinsipnya sederhana: setiap objek berisi sebuah counter, yang bertambah saat referensi ke objek disimpan di suatu tempat, dan yang dikurangi saat referensi itu dihapus. Saat counter mencapai nol, referensi terakhir ke objek telah dihapus dan objeknya dibebaskan. \par
\end{document}
