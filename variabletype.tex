\section{Variable Type}
Variabel merupakan tempat menyimpan data, sedangkan tipe data adalah jenis data yang terseimpan dalam variabel. Variabel menyimpan data yang dilakukan selama program dieksekusi, yang natinya isi dari variabel tersebut dapat diubah oleh operasi - operasi tertentu pada program yang menggunakan variabel. Variabel juga dapat dikatakan sebagai entitas yang memiliki nilai dan berbeda satu dengan yang lain. Variabel mengalokasikan memori untuk menyimpan nilai.Hal ini berarti ketika membuat variabel, maka  memesan beberapa ruang di memori. 

Pada tahun 1970, sudah memperkenalkan gagasan tipe variabel, yaitu, a
skema abstraksi w.r.t, jenis Contoh tipikal adalah, misalnya, secara abstrak
fungsi identitas dari jenis apa pun atau, yaitu, Ax ~ .x ° "dari tipe o-, sehingga mendapatkan
'identitas universal' Aot.Ax ".x". Identitas universal ini pada gilirannya memiliki tipe Aa.a ~ a,
yang merupakan jenis fungsi (jika seseorang dapat memanggil mereka seperti itu) bergaul ke masing-masing
ketik cr sebuah objek dari tipe crier. Sebenarnya, formalisme itu cukup umum, karena
Pembentukan abstraksi tipe tidak terbatas sama sekali. Yang tentu saja bermasalah
adalah skema evaluasi Dari fungsi universal tipe Aa.cr [a], karena itu
adalah mungkin untuk menerapkan objek t jenis ini ke tipe apapun ~ ', menghasilkan tipe t {~'}
o '[¢ / a]: ini jelas memberi masalah sirkularitas.


Tipe data di bahasa pemrograman python dibagi menjadi dua kelompok yaitu :
\begin{enumerate}
	\item Immutable, merupakan tipe data yang tidak bisa diubah (string dan bilangan)
	\item Mutable, merupakan tipe data yang bisa diubah (list dan dictionary)
\end{enumerate}  

Variabel bisa digunakan untuk menyimpan bilangan bulat, desimal atau juga karakter.Variabel dapat menyimpan berbagai macam tipe data. Variabel bersifat mutable, artinya nilainya bisa berubah-ubah.Tidak seperti pemrograman lainnya, variabel pada Python tidak harus dideklarasikan secara eksplisit. Pendeklarasian variabel terjadi secara otomatis ketika kita memberikan sebuah nilai pada suatu variabel. Untuk pemberian nilai, bisa langsung dengan tanda =. Berdasarkan tipe data sebuah variabel, penafsir mengalokasikan memori dan memutuskan apa yang dapat disimpan dalam memori yang dipesan. Oleh karena itu, dengan menetapkan tipe data yang berbeda ke variabel, Anda dapat menyimpan bilangan bulat, desimal atau karakter dalam variabel ini.

\subsection{Menetapkan Nilai ke Variabel}
Variabel Python tidak memerlukan deklarasi eksplisit untuk memesan ruang memori. Deklarasi terjadi secara otomatis saat Anda menetapkan nilai ke variabel. Tanda sama (=) digunakan untuk menetapkan nilai pada variabel

Operan di sebelah kiri operator = adalah nama dari variabel dan operan di sebelah kanan operator = adalah nilai yang disimpan dalam variabel. Misalnya -
\begin{verbatim}
#!/usr/bin/python

counter = 100          # An integer assignment
miles   = 1000.0       # A floating point
name    = "John"       # A string

print counter
print miles
print name
\end{verbatim}

Di sini, 100, 1000,0 dan "John" adalah nilai-nilai ditugaskan untuk counter , miles , dan name variabel, masing-masing. Ini menghasilkan hasil berikut -
100
1000.0
John

\subsection{Aturan Penulisan Variabel}
Berikut ini merupakan aturan-aturan yang digunakan dalam penulisan variabel pada bahasa pemograman python, yaitu sebagai berikut :
\begin{enumerate}
	\item Nama variabel boleh diawali menggunakan huruf atau garis bawah (_), contoh: nama, _nama, namaKu, nama_variabel.
	\item Karakter selanjutnya dapat berupa huruf, garis bawah (_) atau angka, contoh: __nama, n2, nilai1.
	\item Karakter pada nama variabel bersifat sensitif (case-sensitif). Artinya huruf besar dan kecil dibedakan. Misalnya, variabel_Ku
	\item variabel_ku, keduanya adalah variabel yang berbeda.
	\item Nama variabel tidak boleh menggunakan kata kunci yang sudah ada dalam python seperti if, while, for, dsb.
\end{enumerate}  

\subsection{Membuat Variabel di Python}
Variabel di python dapat dibuat dengan format seperti ini:
\begin{verbatim}
nama_variabel = <nilai>
Contoh:
variabel_ku = "ini isi variabel"
variabel2 = 20
Kemudian untuk melihat isi variabel, kita dapat menggunakan fungsi print.
print variabel_ku
print variabel2
\end{verbatim}

\subsection{Menghapus Variabel}
Ketika sebuah variabel tidak dibutuhkan lagi, maka kita bisa menghapusnya dengan fungsi del().
Contoh:
\begin{verbatim}
>>> nama = "petanikode"
>>> print nama
petanikode
>>> del(nama)
>>> print nama
Traceback (most recent call last):
  File "<stdin>", line 1, in <module>
NameError: name 'nama' is not defined
\end{verbatim}

\subsection{penggunaan variabel}
contoh penggunaan variabel  dalam bahasa pemrograman Python adalah sebagai berikut :
\begin{verbatim}
#proses memasukan data ke dalam variabel
nama = "John Doe"
#proses mencetak variabel
print(nama)

#nilai dan tipe data dalam variabel  dapat diubah
umur = 20               #nilai awal
print(umur)             #mencetak nilai umur
type(umur)              #mengecek tipe data umur
umur = "dua puluh satu" #nilai setelah diubah
print(umur)             #mencetak nilai umur
type(umur)              #mengecek tipe data umur

namaDepan = "Budi"
namaBelakang = "Susanto"
nama = namaDepan + " " + namaBelakang
umur = 22
hobi = "Berenang"
print("Biodata\n", nama, "\n", umur, "\n", hobi)

#contoh variabel lainya
inivariabel = "Halo"
ini_juga_variabel = "Hai"
_inivariabeljuga = "Hi"
inivariabel222 = "Bye" 

panjang = 10
lebar = 5
luas = panjang * lebar
print(luas)
\end{verbatim}
