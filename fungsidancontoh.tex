Variabel

Variabel di Python tidak menyertakan tipe data secara eksplisit, berbeda dengan C++ atau Java. Semua tipe data sama cara memasukkan nilainya ke dalam variabel.

\begin{tabular}{|c|c|c|c|}
%%%%%%%%%%%%%%%%%%%%%%%%%%%%%%%%%%%%%%%%%%%%%
%					assek					%
%%%%%%%%%%%%%%%%%%%%%%%%%%%%%%%%%%%%%%%%%%%%%
\hline
$>>>$ integer = 1 \\ \hline
$>>>$ float = 2.5 \\ \hline
$>>>$ string = "aku adalah string" \\ \hline
$>>>$ boolean = False \\ \hline
$>>>$ integer \\ \hline
1 \\ \hline
$>>>$ float \\ \hline
2.5 \\ \hline
$>>>$ string \\ \hline
"aku adalah string" \\ \hline
$>>>$ boolean \\ \hline

\end{tabular}

False  
String dapat diapit dengan tkamu kutip tunggal ' ' atau tkamu kutip gkamu " ". Kita juga bisa mendeklarasikan string multiline dengan mengapit string dalam tkamu kutip tunggal atau gkamu sebanyak tiga kali (''' ''' atau """ """).
  
$>>>$ multiline = '''
$...$ Aku seorang kapiten
$...$ Mempunyai pedang panjang
$...$ Kalau berjalan prok-prok-prok
$...$ Aku seorang kapiten
$...$ '''
$...$
$>>>$ multiline
$...$%nAku seorang kapiten%nmempunyai pedang panjang\nkalau berjalan prok-prok-prok%nAku seorang kapiten%
Struktur Data
List
Pada Python, kita tidak dapat menemui konsep array. Sebagai gantinya, python menyediakan struktur data list. Inisialisasi untuk list sama seperti variabel biasa.

\begin{tabular}{|c|c|c|c|}
\hline
$>>>$ list = [1, 2, 3, 4] \\ \hline
$>>>$ list \\ \hline
$[1, 2, 3, 4]$
$>>>$ list = [1.5, 2.4, 0.5] \\ \hline
$>>>$ list \\ \hline
$[1.5, 2.4, 0.5]$ \\ \hline

$>>>$ list = range(5) \\ \hline
$>>>$ list \\ \hline
$[0, 1, 2, 3, 4]$ \\ \hline
$>>>$ list = range(0,5) \\ \hline
$>>>$ list \\ \hline
$[0, 1, 2, 3, 4]$ \\ \hline
$>>>$ list = range(0,10,2) \\ \hline
$[0, 2, 4, 6, 8]$ \\ \hline 

$>>>$ list = [1, 'dua', 3.0] \\ \hline
$>>>$ list \\ \hline
$[1, 'dua', 3.0]$
\end{tabular}

\begin{tabular}{|c|c|c|c|}
\hline
$>>>$ list = [1, [1, 2, 3.0], [1.5, 'dua']] \\ \linebreak
%Kita dapat menambahkan atau mengurangi elemen pada list yang sudah dibuat sebelumnya dengan menggunakan fungsi append(), insert() dan remove().
$>>>$ list = [0, 1] \\ \linebreak
$>>>$ list \\ \linebreak
$[0, 1]$ \\ \linebreak
$>>>$ list.append(2)\\ \linebreak % menambahkan elemen pada akhir list 
$>>>$ list \\ \linebreak
$[0, 1, 2]$ \\ \linebreak
$>>>$ list.insert(1, 'aku ditengah')\\ \linebreak % menambahkan elemen pada indeks 1
$>>>$ list \\ \linebreak
$[0, 'aku ditengah', 1, 2]$ \\ \linebreak
$>>>$ list.remove('aku ditengah')\\ \linebreak % menghapus elemen yang nilainya 'aku ditengah'
$>>>$ list \\ \linebreak
$[0, 1, 2]$ \\ \linebreak
%Untuk mengambil elemen pertama dari suatu list, kita dapat menggunakan pop.
$>>>$ list = [0, 1] \\ \linebreak
$>>>$ list.pop() \\ \linebreak
$0$ \\ \linebreak
%Dictionary
%Selain list, Python memiliki struktur data dictionary. Mirip seperti konsep hash, tiap elemen pada dictionary memiliki alias atau key.
$>>>$ dict = {'satu': 1, 'dua': 2, 'tiga': 3} \\ \linebreak
%Berbeda dengan list, cara akses nilai pada dictionary adalah dengan menyertakan nama key-nya.
$>>>$ dict['satu'] \\ \linebreak
$1$ \\ \linebreak
$>>>$ dict['dua'] \\ \linebreak
$2$ \\ \linebreak
%Elemen pada dictionary dapat terdiri dari tipe data yang berbeda-beda, bahkan nama key juga tidak harus selalu string. Kita juga bisa memasukkan list sebagai elemen pada dictionary.
$>>>$ dict = {'satu': 1.0, 'dua': [1, 'dua', [1, 2, 3]], 3: 'tiga'} \\ \linebreak
$>>>$ dict[3] \\ \linebreak
$'tiga'$ \\ \linebreak
$>>>$ dict['dua'] \\ \linebreak
$[1, 'dua', [1, 2, 3]]$ \\ \linebreak
$>>>$ dict['dua'][1] \\ \linebreak
$['dua']$ \\ \linebreak
$>>>$ dict['dua'][2] \\ \linebreak 
$[1, 2, 3]$ \\ \linebreak
$>>>$ dict['dua'][2][1] \\ \linebreak
$[2]$ \\ \linebreak

\end{tabular}

List dan dictionary adalah struktur data yang paling sering digunakan pada Python, namun struktur data pada Python tidak hanya terbatas pada itu saja. Untuk mempelajari struktur data lebih lanjut, kamu dapat membaca artikel ini.
If…elif…else…
Fungsi if pada Python lebih sederhana penulisannya dibandingkan C++ atau Java. Untuk mengapit pernyataan di dalam if tidak menggunakan kurung kurawal. Penulisan pernyataan yang masuk pada fungsi if lebih menjorok ke dalam (menggunakan whitespace).

\begin{tabular}{|c|c|c|c|}
\hline
$>>>$ var = 1 \\ \linebreak
$>>>$ if var < 2: \\ \linebreak
$...$ 	var = var + 1 \\ \linebreak
$...$ elif var < 3: \\ \linebreak
$...$ 	var = var + 2 \\ \linebreak
$...$ else: \\ \linebreak
$...$ 	var = 0 \\ \linebreak
$...$ \\ \linebreak
$>>>$ var \\ \linebreak
$2$ \\ \linebreak

\end{tabular}


For
Fungsi for sedikit berbeda dengan C++ atau Java, karena kita tidak menginisialisasi kondisi berhenti iterasi. Di Python, for akan menjalankan pernyataan di dalamnya sebanyak elemen pada list dan akan berhenti setelah nilai indeks elemen melebihi nilai indeks maksimal pada list.

\begin{tabular}{|c|c|c|c|}
\hline

$>>>$ list = ['maine coon', 'munchkin', 'siamese'] \\ \linebreak
$>>>$ var = 0  \\ \linebreak
$>>>$ for el in list: \\ \linebreak
$...$ 	var = var + 1 \\ \linebreak
$...$ \\ \linebreak
$>>>$ var \\ \linebreak
$2$ \\ \linebreak

\end{tabular}


Agar kita bisa menginisialisasi jumlah perulangan, kita bisa menggunakan range.

\begin{tabular}{|c|c|c|c|}
\hline

$>>>$ var = 0 \\ \linebreak
$>>>$ for el in range(1, 10): \\ \linebreak
$...$ 	var = var + 1 \\ \linebreak
$...$\\ \linebreak
$>>>$ var\\ \linebreak
$9$\\ \linebreak

\end{tabular}


Untuk menghentikan atau melanjutkan iterasi pada suatu kondisi tertentu, kita dapat menggunakan break dan continue.


 
\begin{tabular}{|c|c|c|c|}
\hline
 
$>>>$ var = 0 \\ \linebreak
$>>>$ for el in range(1, 10): \\ \linebreak
$...$ 	if var > 5: \\ \linebreak
$...$ 		break \\ \linebreak
$...$	elif var == 3: \\ \linebreak
$...$ 		continue \\ \linebreak
$...$ 	var = var + 1 \\ \linebreak
$...$ \\ \linebreak
$>>>$  \\ \linebreak
$3$ \\ \linebreak

\end{tabular}



While
Fungsi while akan terus secara iteratif menjalankan pernyataan di dalamnya selama kondisi pada while belum terpenuhi.




\begin{tabular}{|c|c|c|c|}
\hline

$>>>$ var = 10 \\ \linebreak
$>>>$ while var > 0: \\ \linebreak
$...$ 	var = var - 1 \\ \linebreak
$...$ \\ \linebreak
$>>>$ \\ \linebreak
$0$ \\ \linebreak

\end{tabular}


Print
Fungsi print digunakan untuk mencetak string pada terminal.
$>>>$ print 'Halo dunia!'
Halo dunia!
Kita juga bisa mencetak string multiline dengan menggunakan fungsi yang sama.


\begin{tabular}{|c|c|c|c|}
\hline

$>>>$ print ''' \\ \linebreak
$...$ Aku seorang kapiten \\ \linebreak
$...$ Mempunyai pedang panjang \\ \linebreak
$...$ Kalau berjalan prok-prok-prok \\ \linebreak
$...$ Aku seorang kapiten \\ \linebreak
$...$ ''' \\ \linebreak

\end{tabular}



Aku seorang kapiten
Mempunyai pedang panjang
Kalau berjalan prok-prok-prok
Aku seorang kapiten
Menyambungkan 2 atau lebih string dapat dilakukan dengan menggunakan operator 

$+.$
$>>>$ print 'Halo ' + 'dunia!'
Halo dunia!

$>>>$ print 'Kucing yang kami pelihara di rumah berjenis: ', ['main coon', 'siamese', 'munchkin']
Kucing yang kami pelihara di rumah berjenis: ['main coon', 'siamese', 'munchkin']
 (formatting) dengan menggunakan tkamu kurung kurawal {}.


\begin{tabular}{|c|c|c|c|}
\hline

>>> x = 1 \\ \linebreak
>>> y = 2 \\ \linebreak
>>> print '{} + {} = {}'.format(x, y, x + y) \\ \linebreak
1 + 2 = 3 \\ \linebreak

\end{tabular}

terimakasih isal.
