
\begin{document}


\textbf{GUI Programming}

Python menyediakan berbagai pilihan untuk mengembangkan antarmuka pengguna grafis (GUIs). 
Berikut dibawah ini merupakan berbagai pilihan yang disediakan oleh Python :
\begin{itemize}
\item Tkinter \par
Tkinter merupakan standar bahasa python yang ditetapkan untuk membangun suatu antarmuka pengguna grafik (GUI). 
\item wxPython \par
wxPython adalah toolkit antarmuka pengguna grafis (GUI) yang digunakan dalam skripsi ini dan ini adalah pembungkus untuk toolkit wxWidgets.
\item Jpython \par
Port Python untuk java yang memberikan Python script akses tanpa batas ke perpustakaan kelas java pada mesin lokal \par
\end{itemize}
\vspace{12pt}
\noindent 
\section{\textbf Tkinter Pemrograman}
Tkinter adalah perpustakaan GUI standar untuk Python. Python bila dikombinasikan dengan Tkinter menyediakan cara yang amat mudah dan cepat untuk membuat aplikasi GUI. Tkinter menyediakan antarmuka yang berorientasi objek yang kuat untuk toolkit Tk GUI.Membuat aplikasi GUI menggunakan Tkinter adalah tugas yang mudah. Yang diperlukan adalah melakukan langkah-langkah sebagai berikut : 
\begin{enumerate} 
	\item Mengimpor Tkinter modul 
	\item Buat jendela utama aplikasi GUI
	\item Tambahkan satu atau lebih dari widget tersebut diatas ke aplikasi GUI
	\item Masukkan acara loop utama untuk mengambil tindakan terhadap setiap peristiwa dipicu oleh pengguna
\end{enumerate}


antarmuka Python open-source untuk wxWindows 
 
\item Jpython 

Port Python untuk java yang memberikan Python script akses tanpa batas ke perpustakaan kelas java pada mesin lokal 

\subsection{Tkinter Pemrograman} 
\hspace*{0.5in} Tkinter adalah perpustakaan GUI standar untuk Python. Python bila dikombinasikan dengan Tkinter menyediakan cara yang mudah dan cepat untuk membuat aplikasi GUI. Tkinter menyediakan antarmuka berorientasi ojek yang kuat untuk toolkit Tk GUI. 
 
\hspace*{0.5in} Membuat aplikasi GUI menggunakan Tkinter adalah tugas yang mudah. Yang diperlukan adalah melakukan langkah-langkah sebagai berikut : 
 
\item Mengimpor Tkinter modul 
\item Buat jendela utama aplikasi GUI 
\item Tambahkan satu atau lebih dari widget tersebut diatas ke aplikasi GUI 
\item Masukkan acara loop utama untuk mengambil tindakan terhadap setiap peristiwa dipicu oleh pengguna\end{itemize}
 
\vspace{12pt} 
Contoh : 
\begin{verbatim}
$  \#  $!/usr/bin/python} 
import Tkinter} 

top = Tkinter.Tk()} 

 $  \#  $ Code to add widgets will go here...} 

top.mainloop()} 
\end{verbatim}


\vspace{12pt}
\vspace{12pt}
\noindent 
Contoh : 
\begin{verbatim}
#!/usr/bin/python 
import Tkinter 
top = Tkinter.Tk()
# Code to add widgets will go here...
top.mainloop()
\end{verbatim}


\vspace{10pt}
\subsection{Tkinter Widget} 
\hspace*{0.5in} Tkinter menyediakan berbagai kontrol seperti tombol, label dan kotak teks yang digunakan dalam aplikasi GUI. Kontrol ini biasanya disebut widget.  
 
\hspace*{0.5in} Saat ini ada 15 jenis widget di Tkinter. Menyajikan widget serta penjelasan singkat pada tabel berikut ini : 
\vspace{100pt}



 %%%%%%%%%%%%  Table No:1 Here %%%%%%%%%%%%%%


\begin{table}[h]
	\caption{Ukuran}
		\begin{center}
		\begin{tabular}{|l|l|}
			\hline
			Operator & Penjelasan \\
			\hline
			Button & Menampilkan tombol dalam aplikasi\\
			Canvas & Menggambar bentuk seperti garis, oval, 
			poligon dan persegi panjang dalam aplikasi\\
			Checkbutton & Menampilkan sejumlah pilihan 
			sebagai kotak centang. Pengguna dapat memilih beberapa pilihan pada suatu waktu\\
			Entry & Menampilkan bidang garis teks tunggal untuk menerima nilai-nilai dari pengguna\\
			Frame & Wadah untuk mengatur widget lainnya\\
			Label & Memberikan keterangan garis single untuk widget lainnya. Hal ini berisi gambar\\
			Listbox & Menyediakan daftar pilihan kepada pengguna\\
			Menubutton & Menampilkan menu dalam aplikasi\\
			Menu & Memberikan berbagai perintah untuk pengguna. Perintah-perintah ini terkandung di dalam MenuButton\\
			Message & Menampilkan bidang teks multiline untuk menerima nilai-nilai dari pengguna\\
			RadioButton & Menampilkan sejumah pilihan sebagai tombol radio. Pengguna dapat memilih hanya satu pilihan pada suatu waktu\\
			Scale & Menyediakan widget slide\\
			Scrollbar & Menambah kemampuan bergulir ke berbagai widget seperti kotak daftar\\
			Text & Menampilka teks dalam beberapa garis\\
			Toplevel & Menyediakan wajah jendela terpisah\\
			PanedWindow & Wadah yang mengandung sejumlah panel disusun horizontal atau vertikal\\
			LabelFrame & Wadah widget sederhana. Bertindak sebagai spacer atau wajah untuk layout jendela kompleks\\
			TkMessageBox & Menampilkan kotak pesan dalam aplikasi\\
			Spinbox & Memilih sejumlah tetap nilai-nilai\\
		\hline
		\end{tabular}
		\end{center}
\end{table}

	


 %%%%%%%%%%%%  Table No:1 Ends Here %%%%%%%%%%%%%%


\vspace{12pt}

\subsection{Atribut Tkinter}
\noindent 

\hspace*{0.5in} Beberapa atribut sebagai ukuran, warna dan font ditentukan. Berikut adalah beberapa atribut standar : 
 
\begin{itemize}
\item Ukuran 
 
Berbagai panjang, lebar, dan dimensi lain dari widget digambarkan dalam banyak unit yang berbeda seperti : 
 
\item Jika menetapkan dimensi ke integer diasumsikan dalam piksel 
 
\item Menentukan unit dengan menentukan dimensi untuk string yang berisi sejumlah diikuti oleh :
\end{itemize}
\vspace{100pt}

 


 %%%%%%%%%%%%  Table No:2 Here %%%%%%%%%%%%%%


\begin{table}[ht]
	\caption{Ukuran}
	\begin{center}
	\begin{tabular}{|c|c|}
		\hline
		Karakter&  Penjelasan \cr
		\hline
		c&Sentimeter\\
		i&Inci\\
		m&Milimeter\\
		p&Poin printer\\
		\hline
	\end{tabular}
	\end{center}
\end{table}


 %%%%%%%%%%%%  Table No:2 Ends Here %%%%%%%%%%%%%%


 
 \hspace*{0.5in} \vspace{12pt}
\subsection{Panjang Tkinter}
 \hspace*{0.5in} Tkinter mengungkapkan panjang sebagai integer jumlah piksel. Berikut ini adalah daftar pilihan panjang umum:
\noindent 
\begin{itemize}
	\item borderwidth
	Lebar batas yang memberikan tampilan tiga dimensi untuk widget
	\item highlightthickness
	Lebar puncak persegi panjang ketika widget memiliki fokus
 	\item padX padY
	Ruang tambahan widget dari manajer tata letak luar minimum widget perlu menampilkan isinya di x dan y arah
	\item selectborderwidth
	Lebar perbatasan tiga dimensi disekitar dipilih item widget
	\item wraplength \par
	Panjang garis maksimum untuk widget yang melakukan kata membungkus
	\item height
	Tinggi diinginkan widget
	\item underline
	Indeks karakter untuk menggarisawahi dalam teks widget 
	\item width
	\item Lebar diinginkan widget
\end{itemize}
 
\noindent 
subsection{Warna Tkinter}
\noindent 
Tkinter memiliki warna dengan string. Ada dua cara umum untuk menentukan sebuah warna di Tkiter, yaitu : \par
\noindent 
\begin{itemize}
	\item Menggunakan string menentukan proporsi merah, hijau dan biru didigit heksadesimal. Misalnya  `` \#ffff '' putih,  ``  \#000000 '' hitam dan  ``\#000fff000 '' hijau.
\noindent 
	\item Menggunakan lokal standar nama warna . warna-warna ``white'', ``black'',  ``green'' dan  ``magenta'' akan selalu tersedia.
\end{itemize}

\vspace{12pt}
Pilihan warna umum :
\noindent 
\begin{itemize}
	\item activebackground \par
	Warna latar berlakang untuk widget ketika widget aktif \par
	\noindent 
	\item activeforeground \par
	Warna depan untuk widget ketika widget aktif \par
	\noindent 
	\item background \par
	Merepresentasikan sebagai \textit{bg} \par
	\noindent 
	\item disableforeground \par
	Warna depan untuk widget ketika widget dinonaktifkan \par
	\noindent 
	\item foreground \par
	Merepresentasikan fg \par
	\noindent 
	\item highlightbackground \par
	Warna latar belakang dari daerah puncak ketika widget memiliki fokus \par
	\noindent 
	\item hightlightcolor \par
	Warna depan dari wilayah puncak ketika widget memiliki fokus \par
	\noindent 
	\item selectbackground \par
	Warna latar belakang untuk item yang dipilih dari widget \par
	\noindent 
	\item selectforeground \par
	Warna depan untuk item yang dipilih dari widget \par
	\noindent 
	\item Font \par
	\noindent 
	Sebagai tupel yang elemen pertama adalah keluarga font diikuti dengan string yang berisi satu atau lebih gaya pengubah tebal,miring, garis bawah dan overstrike. 
\end{itemize}
	\noindent 
Contoh : \par
	\begin{verbatim}
		("Helvetica","16") untuk 16 - point Helvetica biasa
		("Times","24","beranimiring") untuk 24 - point kali miring tebal
	\end{verbatim}
 		\par
\vspace{12pt}
Dapat membuat  ``font object'' dengan mengimpor modul tkFont dan menggunakan kelas konstruktor font nya : \par
Import tkFont \par
Font = tkFont.Font (option, ....) \par

\vspace{12pt}
Berikut adalah daftar pilihan : 
 
\begin{itemize}

	\item Family \par
	Font nama keluarga sebagai string \par
	\noindent 
	\item Size \par
	Font tinggi sebagai integer dalam poin \par
	\noindent 
	\item Weight \par
	Bold untuk teal, normal untuk berat badan secara teratur \par
	\noindent 
	\item Slant \par
	Italic untuk miring, roman untuk unstlanted \par
	\noindent 
	\item Underline \par
	1 untuk teks yang digarisbawahi, 0 untuk normal \par
	\noindent 
	\item Overstrike \par
	1 untuk teks telak, 0 untuk normal \par
	Jika berjalan di bawah X window system, dapat menggunakan salah satu nama font X. Sebagai contoh, font bernama  \verb|"-*lucidatypewriter-medium-r-*-*-*-140-*-*-*"| adalah favorit fixed-width font penulis untuk digunakan pada layar. \par
	\noindent 
	\item Jangkar \par
	\noindent 
	Jangkar digunakan untuk mendefinisikan mana teks diposisikan relatif terhadap titik acuan. Berikut adalah daftar kemungkinan konstanta yang dapat digunakan :
	\noindent
	\begin{itemize}
		\item NW  
		\item N
		\item NE
		\item W
		\item TENGAH
		\item E
		\item SW
		\item S
		\item SE
	\end{itemize}
\end{itemize}
 \par
\vspace{12pt}
Jika menggunakan tengah sebagai jangkar tek, tek akan ditengahkan horizontal dan vertikal disekitar titik referensi. \par
Jangkar NW akan posisi teks sehingga titik referensi bertepatan dengan laut sudut kotak berisi teks \par
Jangakr W akan pusat teks secara vertikal disekitar satu titik referensi dengan tepi kiri kotak teks yang melewati titik itu dan sebagainya. \par
Jika membuat widget kecil didalam bingkai besar dan menggunakan jangkar = SE pilihan, widget akan ditempatkan disudut kanan bawah gambar. Jika menggunakan anchor = N sebaliknya widget akan dipusatkan disepanjang tepi atas. \par
\noindent 

subsection{Gaya relief}

Widget mengacu pada efek 3-D simulasi terbaru disekitar bagian luar widget. Berikut adalah daftar konstanta yang mungkin dapat digunakan untuk atribut:
\begin{itemize}
	\item Datar
	\item Dibesarkan
	\item Cekung
	\item Alur
	\item Punggung bukit
\end{itemize}


\vspace{12pt}
Contoh : \par
\begin{verbatim}
from Tkinter import *
import Tkinter

top = Tkinter.Tk()

B1 = Tkinter.Button(top, text ="FLAT", relief=FLAT )
B2 = Tkinter.Button(top, text ="RAISED", relief=RAISED )
B3 = Tkinter.Button(top, text ="SUNKEN", relief=SUNKEN )
B4 = Tkinter.Button(top, text ="GROOVE", relief=GROOVE )
B5 = Tkinter.Button(top, text ="RIDGE", relief=RIDGE )

B1.pack()
B2.pack()
B3.pack()
B4.pack()
B5.pack()
top.mainloop()
\end{verbatim}

\noindent 
\subsection{Britmaps}
\noindent 
Ada beberapa jenis bitmap yang tersedia, diantaranya: \par
\noindent 
\begin{itemize}
	\item Kesalahan
	\item Gray75 
	\item Gray50 
	\item Gray12
	\item Jam Pasir 
	\item Info 
	\item Questhead
	\item Perantanyaan
	\item Peringatan
\end{itemize}

\vspace{12pt}
Contoh: \par
\begin{verbatim}
from Tkinter import *
import Tkinter

top = Tkinter.Tk()

B1 = Tkinter.Button(top, text ="error", relief=RAISED,\
                         bitmap="error")
B2 = Tkinter.Button(top, text ="hourglass", relief=RAISED,\
                         bitmap="hourglass")
B3 = Tkinter.Button(top, text ="info", relief=RAISED,\
                         bitmap="info")
B4 = Tkinter.Button(top, text ="question", relief=RAISED,\
                         bitmap="question")
B5 = Tkinter.Button(top, text ="warning", relief=RAISED,\
                         bitmap="warning")
B1.pack()
B2.pack()
B3.pack()
B4.pack()
B5.pack()
top.mainloop()
\end{verbatim}

\noindent 
\subsection{Kursor} \par
\noindent 
Berikut daftar menarik : \par
\noindent 
\begin{itemize}
	\item Panah 
	\item Lingkaran
	\item Jam 
	\item Menyebrang 
	\item Dotbox 
	\item Bertukar
	\item Fluer 
	\item Jantung 
	\item Manusia 
	\item Tikus
	\item Bajak laut 
	\item Tamah 
	\item Antar jemput 
	\item Perekat
	\item Laba-laba
	\item Kaleng semprot
	\item Bintang
	\item Target 
	\item Tcross 
	\item Melakukan perjalanan
	\item Menonton
\end{itemize}

\vspace{12pt}
Contoh : \par
\begin{verbatim}
from Tkinter import *
import Tkinter

top = Tkinter.Tk()

B1 = Tkinter.Button(top, text ="circle", relief=RAISED,\
                         cursor="circle")
B2 = Tkinter.Button(top, text ="plus", relief=RAISED,\
                         cursor="plus")
B1.pack()
B2.pack()
top.mainloop()

\end{verbatim}
\section{Manajemen Geometri}
\noindent 
 \hspace*{0.5in} Semua widget tkinter memiliki akses ke metode manajemen geometri tertentu, yang memiliki tujuan menggorganisir widget diseluruh wilayah widget induk. Tkinter mengekspos kelas manager geometri berikut : \par
\noindent 
\begin{itemize}
	\item Metode the \textit{pack()} \par
	\noindent 
	Manajer geometri ini mengatur widget diblok sebelum menempatkan mereka di widget induk \par
	\noindent 
	\item Metode the \textit{grid()} \par
	\noindent 
	Manajer geometri ini mengatur widget dalam struktur tabel seperti di widget induk \par
	\noindent 
	\item Metode the  \textit{place()}\end{itemize} \par
	\noindent 
	Manajer geometri ini mengatur widget dengan menempatkan dalam posisi tertentu dalam widget induk \par
	\vspace{12pt}
	\noindent 
\section{Manfaat Tkinter} \par
Tkinter sangat sederhana. Berikut manfaat Tkinter dibandingkan GUI toolkit : \par
\noindent 
\begin{itemize}
	\item Tkinter terdiri dari sejumlah modul.  (Tkinter)\vspace{\baselineskip} \par
	Antarmuka Tk terletak di modul biner bernama \_ tkinter (ini adalah tkinter di versi sebelumnya). Modul ini berisi lowlevel antarmuka ke Tk, dan tidak boleh digunakan langsung oleh pemrogram aplikasi. inibiasanya shared library (atau DLL), tapi mungkin dalam beberapa kasus secara statis terkait denganPenerjemah Python.
	\item Tkinter mudah diakses oleh siapa saja. (Accessibilty)\vspace{\baselineskip} \par
	Tkinter merupakan toolkit yang ringan dan satu-satunya solusi GUI yang paling sederhana untuk Python sampai saat ini. Cukup menuliskan beberapa baris kode Python untuk membuat aplikasi GUI sederhana dengan Tkinter. Untuk menambahkan komponen baru pada Tkinter, dapat membuatnya dalam kode Python atau menambahkan paket ekstensi seperti Pmw, Tix, atau ttk.
	
	\item widget root tk (widget root tk)\vspace{\baselineskip} \par
	Untuk menginisialisasi Tkinter, kita harus membuat widget root Tk. Ini adalah jendela biasa, dengan judul bar dan hiasan lainnya yang disediakan oleh window manager Anda. Anda seharusnya saja buat satu widget root untuk setiap program, dan itu harus dibuat sebelum widget lainnya. \par
\noindent 
	\item Tkinter mudah digunakan di semua platform (Portability)\vspace{\baselineskip} \par
	Sebuah program Python yang dibangun menggunakan Tkinter dapat berjalan dengan baik di semua platform sistem operasi seperti Microsoft Windows, Linux, dan Macintosh. Dan dari segi tampilan window, akan terlihat sama dengan standar platform yang digunakan. \par
\noindent 
	\item Tkinter selalu tersedia di Python (Availability)\vspace{\baselineskip} \par
	Tkinter merupakan modul standar pada pustaka Python. Sebagian besar paket instalasi Python sudah langsung berisi Tkinter. Khusus untuk beberapa distro Linux, perlu menambahkan paket Tkinter secara terpisah. Pada Windows, bisa langsung menggunakan Tkinter sesaat setelah menginstal paket instalasi Python. \par
\noindent 
	\item handles bagus di gunakan di Python (Handles)\vspace{\baselineskip} \par
	item kandles merupakan nilai integer yang digunakan untuk mengidentifikasi item tertentu pada kanvas. Tkinter secara otomatis menugaskan pegangan baru ke setiap item baru yang dibuat di atas kanvas. Item Handles dapat di lewatkan ke berbagai metode kanvas baik sebagai bilangan bulat atau sebagai string. Tag adalah nama simbolis yang dilekatkan pada item. Tag adalah string biasa, dan bisa berisi apa saja kecuali spasi (asalkan tidak sesuai dengan item pegangan). \par
\noindent 
	\item Modul Tkinter menyediakan kelas yang sesuai dengan berbagai jenis widget di Tk (Module)\vspace{\baselineskip} \par
	dan sejumlah mixin dan kelas pembantu lainnya (mixin adalah kelas yang dirancang untuk menjadi dikombinasikan dengan kelas lain menggunakan multiple inheritance). Bila Anda menggunakan Tkinter, Anda jangan pernah mengakses kelas mixin secara langsung. \par
\noindent 
	\item Dokumentasi Tkinter sangat LUAR BIASA (Documentation)\vspace{\baselineskip} \par
	Python (plus Tkinter) ini bersifat open-source, maka banyak sekali komunitas-komunitas yng membahas Python dan Tkinter dan bisa belajar dan bertanya langsung dengan para ahli.
\end{itemize}

\end{document}

